% test.tex -*-latex-*-
%
% Test file to check functionality.
%
% Copyright (c) 2024 Marcel Ilg
%
% Permission is hereby granted, free of charge, to any person obtaining
% a copy of this software and associated documentation files (the
% "Software"), to deal in the Software without restriction, including
% without limitation the rights to use, copy, modify, merge, publish,
% distribute, sublicense, and/or sell copies of the Software, and to
% permit persons to whom the Software is furnished to do so, subject to
% the following conditions:
%
% The above copyright notice and this permission notice shall be
% included in all copies or substantial portions of the Software.
%
% THE SOFTWARE IS PROVIDED "AS IS", WITHOUT WARRANTY OF ANY KIND,
% EXPRESS OR IMPLIED, INCLUDING BUT NOT LIMITED TO THE WARRANTIES OF
% MERCHANTABILITY, FITNESS FOR A PARTICULAR PURPOSE AND NONINFRINGEMENT.
% IN NO EVENT SHALL THE AUTHORS OR COPYRIGHT HOLDERS BE LIABLE FOR ANY
% CLAIM, DAMAGES OR OTHER LIABILITY, WHETHER IN AN ACTION OF CONTRACT,
% TORT OR OTHERWISE, ARISING FROM, OUT OF OR IN CONNECTION WITH THE
% SOFTWARE OR THE USE OR OTHER DEALINGS IN THE SOFTWARE.
\documentclass[parskip=half]{scrartcl}
\usepackage{scrhack} % hacks other packages

\KOMAoptions{}

% for identifying the compiler
\usepackage{iftex}

% check for the compiler and load fonts
\iftutex%
  \usepackage{fontspec}% Loads lmodern font
\else%
  \usepackage[T1]{fontenc}% Better encoding for european languages
  \usepackage{lmodern}% Better looking font
\fi%
% recalculate typearea after loading fonts
\recalctypearea%

% (multi-)language support
\usepackage[american]{babel}

\usepackage{amssymb}
\usepackage[bm]{conmath}

\usepackage{fancyvrb-ex}
\fvset{frame=single,numbers=left}


% a rule spanning the entire line
\NewDocumentCommand{\fulllinerule}{}{
  \rule{\linewidth}{0.4pt}
}

\begin{document}

\section{Test for Delimited Operators}


\DeclarePairedDelimiter\inparent{\lparen}{\rparen}


\begin{SideBySideExample}[xrightmargin=.4\textwidth]
\begin{gather*}
  \psin_{bljv}^{dkfjal}{a} \\
   \psin*_{p}^{b}{\frac{1}{2}}\\[0.5ex]
   \bsin*{\frac{1}{2}} \\[0.5ex]
   \Bsin*{\frac{1}{2}} \\[0.5ex]
   \vsin*{\frac{1}{2}} \\[0.5ex]
   \Vsin*{\frac{1}{2}} \\
   \psin[\Bigg]^{2}{a}\\
   \plim_{x\to\infty}{1+\frac{1}{x}}^{2}
   \qquad \lim_{0\to\infty}\\
   \plog_{10}{a}
\end{gather*}
\end{SideBySideExample}

\section{Tests for Delimited Vector Calculus Operators}

Standalone operators:

\begin{SideBySideExample}[xrightmargin=.3\textwidth,gobble=2]
  \begin{gather*}
    \grad f(x)\\
    \grad_{x} f(x)\\
    \grad[arrownabla=true] f(x)
    \quad \mathop{\nabla} f(x)\\
    \pgrad_{x}{f\left(\frac{x}{2}\right)}\\
    \pgrad*_{x}{f\left(\frac{x}{2}\right)}\\
    \pgrad[scale=\Bigg,arrownabla=true]_{x}{f(x)} \\
  \end{gather*}
\end{SideBySideExample}

\fulllinerule

\begin{SideBySideExample}[xrightmargin=.3\textwidth,gobble=2]
  \begin{gather*}
    \laplacian f(x)\\
    \laplacian_{x} f(x)\\
    \laplacian[delta-laplace] f(x)\\
    \laplacian[delta-laplace]_{x} f(x)\\
    \plaplacian*[arrowlaplace=true]{f(x)}\\
    \plaplacian*[arrowlaplace=true]_{x}{f(x)}\\
    \plaplacian*[delta-laplace=true]{f(x)}\\
    \plaplacian*[delta-laplace=true]_{x}{f(x)}\\
  \end{gather*}
\end{SideBySideExample}

\fulllinerule

\begin{SideBySideExample}[xrightmargin=.3\textwidth,gobble=2]
  \begin{gather*}
    \divergence f(x) \\
    \divergence_{x} f(x) \\
    \pdiv{f+g}\\
    \pdiv[arrownabla=true]_{x}{f+g}\\
  \end{gather*}
\end{SideBySideExample}

\fulllinerule

\begin{SideBySideExample}[xrightmargin=.3\textwidth,gobble=2]
  \begin{gather*}
    \curl f(x)\\
    \curl_{x} f(x)\\
    \pcurl{f+g}\\
    \pcurl[arrownabla=true]_{x}{f+g}
  \end{gather*}
\end{SideBySideExample}

\fulllinerule

\fulllinerule

\begin{Example}[label={Interactions with Other Stuff},gobble=2]
  \begin{gather*}
    \divergence_{x} f(x) + \sum_{a}^{N} g\\
    \grad_{x}\sum_{0} f(x)
  \end{gather*}
\end{Example}

\section{Vectors and Matrices}

\subsection{Row and Column Vectors}

\begin{Example}[gobble=2]
  \[
    \rvector[delimiter=p]{a_{1},a_{2},a_{3}}
    \times \cvector[align=r,delimiter=p]{-1,2,3}
  \]
\end{Example}

\fulllinerule

\begin{SideBySideExample}[xrightmargin=.5\textwidth,gobble=2]
  \[
    \cvector{1,2,3}
  \]
\end{SideBySideExample}

\fulllinerule

\begin{SideBySideExample}[xrightmargin=.45\textwidth]
\(\smallcvector[delimiter=p]{1,2,3}\)
\end{SideBySideExample}

\fulllinerule

\begin{SideBySideExample}[gobble=2,xrightmargin=.5\textwidth]
  \[
    \pcvector{1,2,3}
  \]

  \(\psmallcvector{1,2,3}\)
\end{SideBySideExample}

\fulllinerule

\subsection{(Anti-)diagonal Matrices}
\begin{Example}[gobble=2]
  \[
    \diagmat[delimiter=p,fill=\blacktriangle]{1,2,3,4} \times
    \antidiagmat[delimiter=p]{1,2,3,4}
  \]
\end{Example}

\fulllinerule

\begin{Example}[gobble=2]
  \(\smalldiagmat{1,2,3,4}\),
  \(\psmalldiagmat{1,2,3,4}\),
  \(\psmallantidiagmat{1,2,3,4}\)
\end{Example}

\nopagebreak\fulllinerule

\subsection{Identity Matrices}
\begin{SideBySideExample}[xrightmargin=.45\textwidth,gobble=2]
  \[
    \idmat[delimiter=p]{4}
  \]

  \(\smallidmat[delimiter=p,fill=0]{4}\)
\end{SideBySideExample}

\fulllinerule

\begin{SideBySideExample}[xrightmargin=.45\textwidth,gobble=2]
  \begin{gather*}
    \pidmat{3}\\
    \bidmat{2}\\
    \Bidmat[fill=\star]{4}\\
    \vidmat{3}\\
    \Vidmat{3}
  \end{gather*}
\end{SideBySideExample}

\nopagebreak\fulllinerule

\begin{Example}[gobble=2]
  Inline identity matrices look like this:
  \(\psmallidmat{3}\), \(\bsmallidmat{2}\), \(\Bsmallidmat{2}\),
  \(\vsmallidmat{3}\), \(\Vsmallidmat{2}\).
\end{Example}


\end{document}
