% \iffalse meta-comment
%
% Copyright (C) 2024 Marcel Ilg
%
% This work may be distributed and/or modified under the
% conditions of the LaTeX Project Public License, either version 1.3
% of this license or (at your option) any later version.
% The latest version of this license is in
%
%      https://www.latex-project.org/lppl.txt
%
% and version 1.3 or later is part of all distributions of LaTeX
% version 2005/12/01 or later.
%
% This work has the LPPL maintenance status ‘maintained’.
%
% The Current Maintainer of this work is Marcel Ilg.
%
% This work consists of the files listed in MANIFEST.md.
% \fi
%
% \iffalse
%
%<*driver>
% Generated by ctanbib
\begin{filecontents*}[overwrite]{\jobname.bib}
@preamble{ "\providecommand\ctanbibpkgname[1]{\textsl{#1}}" }
@manual{amsfonts,
  title = {The \ctanbibpkgname{amsfonts} package},
  subtitle = {TeX fonts from the American Mathematical Society},
  author = {{The American Mathematical Society}},
  url = {https://ctan.org/pkg/amsfonts},
  urldate = {2024-07-04},
  date = {},
  version = {3.04},
}

@preamble{ "\providecommand\ctanbibpkgname[1]{\textsl{#1}}" }
@manual{amsmath,
  title = {The \ctanbibpkgname{amsmath} package},
  subtitle = {AMS mathematical facilities for LaTeX},
  author = {{The LaTeX Project Team}},
  url = {https://ctan.org/pkg/amsmath},
  urldate = {2024-07-04},
  date = {2023-05-13},
  version = {2.17o},
}

@preamble{ "\providecommand\ctanbibpkgname[1]{\textsl{#1}}" }
@manual{bm,
  title = {The \ctanbibpkgname{bm} package},
  subtitle = {Access bold symbols in maths mode},
  author = {Carlisle, David and Mittelbach, Frank and {The LaTeX Project Team}},
  url = {https://ctan.org/pkg/bm},
  urldate = {2024-07-04},
  date = {2023-12-19},
  version = {1.2f},
}

@preamble{ "\providecommand\ctanbibpkgname[1]{\textsl{#1}}" }
@manual{mathtools,
  title = {The \ctanbibpkgname{mathtools} package},
  subtitle = {Mathematical tools to use with amsmath},
  author = {Høgholm, Morten and Madsen, Lars and Robertson, Will and Wright, Joseph and {The LaTeX Project Team}},
  url = {https://ctan.org/pkg/mathtools},
  urldate = {2024-07-04},
  date = {2024-03-11},
  version = {1.30},
}
\end{filecontents*}
\begin{filecontents*}[overwrite]{\jobname-ams-op-table.tex}
\begin{tabular}{rlrlrlrl}
  \cs[module=amsmath,replace=false]{arccos} &\(\arccos\)  &\cs[module=amsmath,replace=false]{deg}    &\( \deg \)  &\cs[module=amsmath,replace=false]{lg}      &\( \lg \)      &\cs[module=amsmath,replace=false]{projlim} &\( \projlim \)\\
  \cs[module=amsmath,replace=false]{arcsin} &\(\arcsin\)  &\cs[module=amsmath,replace=false]{det}    &\( \det \)  &\cs[module=amsmath,replace=false]{lim}     &\( \lim \)     &\cs[module=amsmath,replace=false]{sec}     &\( \sec \)\\
  \cs[module=amsmath,replace=false]{arctan} &\(\arctan\)  &\cs[module=amsmath,replace=false]{dim}    &\( \dim \)  &\cs[module=amsmath,replace=false]{liminf}  &\( \liminf \)  &\cs[module=amsmath,replace=false]{sin}     &\( \sin \)\\
  \cs[module=amsmath,replace=false]{arg}    &\(\arg\)     &\cs[module=amsmath,replace=false]{exp}    &\( \exp \)  &\cs[module=amsmath,replace=false]{limsup}  &\( \limsup \)  &\cs[module=amsmath,replace=false]{sinh}    &\( \sinh \)\\
  \cs[module=amsmath,replace=false]{cos}    &\(\cos\)     &\cs[module=amsmath,replace=false]{gcd}    &\( \gcd \)  &\cs[module=amsmath,replace=false]{ln}      &\( \ln \)      &\cs[module=amsmath,replace=false]{sup}     &\( \sup \)\\
  \cs[module=amsmath,replace=false]{cosh}   &\(\cosh\)    &\cs[module=amsmath,replace=false]{hom}    &\( \hom \)  &\cs[module=amsmath,replace=false]{log}     &\( \log \)     &\cs[module=amsmath,replace=false]{tan}     &\( \tan \)\\
  \cs[module=amsmath,replace=false]{cot}    &\(\cot\)     &\cs[module=amsmath,replace=false]{inf}    &\( \inf \)  &\cs[module=amsmath,replace=false]{max}     &\( \max \)     &\cs[module=amsmath,replace=false]{tanh}    &\( \tanh \)\\
  \cs[module=amsmath,replace=false]{coth}   &\(\coth\)    &\cs[module=amsmath,replace=false]{injlim} &\(\injlim\) &\cs[module=amsmath,replace=false]{min}     &\( \min \)     &                       &\\
  \cs[module=amsmath,replace=false]{csc}    &\(\csc\)     &\cs[module=amsmath,replace=false]{ker}    &\(\ker\)    &\cs[module=amsmath,replace=false]{Pr}      &\( \Pr \)      &                       &\\
\end{tabular}
\begin{center}
  \begin{tabular}{rlrl}
    \cs[module=amsmath,replace=false]{varinjlim}  &\(\varinjlim\)   &\cs[module=amsmath,replace=false]{varliminf} &\(\varliminf\)\\
    \cs[module=amsmath,replace=false]{varprojlim} &\(\varprojlim\)  &\cs[module=amsmath,replace=false]{varlimsup} &\(\varlimsup\)\\
  \end{tabular}
\end{center}
\end{filecontents*}


% custom commands for use inside the documentation
\ExplSyntaxOn
% Property list storing version date information
\prop_new:N \l__moremathdoc_version_date_prop

\cs_new_protected:Nn \moremathdoc_add_version:nn
{
  \prop_put_if_new:Nnn \l__moremathdoc_version_date_prop {#1} {#2}
}

\cs_new_protected:Nn \moremathdoc_get_version:n
{
  \group_begin:
  \prop_get:NnNT \l__moremathdoc_version_date_prop {#1} \l_tmpa_tl
  {
    \tl_use:N \l_tmpa_tl
  }
  \group_end:
}

\NewDocumentCommand \AddVersion { m m }
{
  \moremathdoc_add_version:nn {#1} {#2}
}

\NewDocumentCommand \GetVersionDate { m }
{
  \moremathdoc_get_version:n {#1}
}
\ExplSyntaxOff

\documentclass[%
%show-notes,%
]{l3doc}

\SetupDoc{reportchangedates=true}
\RecordChanges

\usepackage[american]{babel}
\AddBabelHook[american]{nofrench}{afterextras}{\frenchspacing}

\usepackage[style=alphabetic]{biblatex}
\addbibresource{\jobname.bib}

\usepackage{csquotes}
\SetCiteCommand{\autocite}

\usepackage{amssymb}
\usepackage[bm]{moremath}

% renew the ctanbibpkgname command
\NewDocumentCommand\ctanbibpkgname{m}{\pkg{#1}}

% Example operators and commands
\DeclareMathOperator{\glorb}{glorb}
\DeclarePairedDelimiter{\inparen}{\lparen}{\rparen}
\DeclareDelimitedOperator{\pglorb}{\glorb}{\inparen}

\begin{document}
  \DocInput{moremath.dtx}
\end{document}
%</driver>
% \fi
%
% ^^X Get the package version name etc.
% \GetFileInfo{moremath.sty}
%
% ^^X Register changes here
% \AddVersion{v0.1.0}{2024-06-28}
% \AddVersion{v0.2.0}{2024-07-04}
% \AddVersion{v0.3.0}{2024-07-08}
% \AddVersion{v0.4.0}{2024-07-15}
%
% ^^X Placeholder for unreleased versions
% \AddVersion{vUNRELEASED}{UNRELEASED}
%
% \begin{documentation}
%
% \changes{v0.1.0}{\GetVersionDate{v0.1.0}}{Initial development release}
% \changes{v0.3.0}{\GetVersionDate{v0.3.0}}{
%   Split the documentation into several parts.
% }
% \changes{v0.4.0}{\GetVersionDate{v0.4.0}}{
% \textbf{Breaking Change:} Rename the package
%  and the \meta{prefix} used for function names from
%  "conmath" to "moremath".
% }
%
% \title{The \pkg{moremath} package\thanks{This document coresponds to
%   \textsf{moremath}~\fileversion, dated \filedate.}}
%
% \author{Marcel Ilg\thanks{\texttt{mister01x (at) web (dot) de}}}
%
% \date{Released \filedate}
%
% \maketitle
%
% \begin{abstract}
%   The \pkg{moremath} package provides several document level commands
%   to ease the typesetting
%   and \LaTeX{} code readability of certain mathematical constructs.
%   It provides complementary commands to all operators defined by \pkg{amsmath},
%   the commands typeset delimiters that may be automatically, manually or
%   not scaled at all.
%   The commands also accept optional sub- and superscripts to the operators.
%
%   Furthermore it provides several commands to typeset gradient, divergence,
%   curl, and Laplace operators,
%   for which there are also versions with delimiters.
%   Those commands also accept an optional subscript
%   and their appearance can be modified using key-value options.
%
%   Additionally commands are provided for producing row and column vectors,
%   as well as (anti-)diagonal matrices,
%   utilizing \pkg{mathtools} \env{matrix*} family of environments.
%
%   Most of the document level commands defined by this package can also
%   be disabled using a package load-time option to avoid clashes with
%   commands defined by other packages.
% \end{abstract}
%
% \section*{Note: This Package is Still in its Initial Development Phase}
% Do not expect a stable interface until version 1.0.0 has been reached!
% While I try to keep the interface stable and backwards compatible,
% I am unable to promise this.
%
% \tableofcontents
%
% \part{Document Author Documentation}
% \label{pt:doc-auth}
%
% \section{Introduction}
% \label{sec:intro}
%
% When typesetting mathematics in \LaTeX{} it is very common to often encounter,
% code patterns such as |\sin \left( \frac{x}{2} \right)|.
% While this above code works as expected its not especially easy for human
% readers of this code to immediately understand what that code is going to
% do.
%
% This package tries to ease readability of such commonly used constructs
% by providing commands that (hopefully) increase readability,
% as well as speeding up writing math in \LaTeX{}.
% Using \pkg{moremath} the above code can be simplified to |\psin*{\frac{x}{2}}|,
% which is shorter and better \enquote{human readable}.
% The package also provides the possibility for users to define such delimited operators,
% on their own.
%
% There are also other things that can be improved when typesetting math.
% One prominent example is the typsetting of row and column vectors,
% which require the use of a \env{matrix} environment.
% For this case \pkg{moremath} provides commands which accept a comma separated
% list of vector entries,
% obliterating the need for a \env{matrix} environment to typeset vectors.
%
% The same goes for (anti-)diagonal matrices,
% only the elements of the diagonal are of importance here.
% Therefore the production of those matrices may also be simplified into a single command,
% which again improves readability of the code compared to a mostly empty \env{matrix} environment.
%
% Another feature of this package is the definition of several (delimited)
% vector calculus operators such as gradient, divergence and curl operators,
% this does not only improve the semantics of the math code
% (|\mathop{\nabla} f| vs.\ |\grad f|),
% but also tries to provide the correct spacing between the operator and its
% arguments.
%
% The \pkg{moremath} package is written using \LaTeX{}3
% and provides a small \LaTeX{}3 interface for class and package writers.
%
% The source code for \pkg{moremath} is hosted on GitHub at
% \begin{center}
%   \url{https://github.com/Mister00X/moremath}
% \end{center}
% If you have any issues or found a bug, feel free to register an issue there.
%
% \section{Dependencies}
% \label{sec:deps}
%
% The main dependency of this package is \pkg{mathtools}~\autocite{mathtools}.
% Optionally the \pkg{bm}~\autocite{bm} may be loaded.
%
% If the option "no-vector" has \emph{not} been given as a package load time option,
% this package also loads the \pkg{amssymb}~\autocite{amsfonts} package.
%
% \section{Package Options}
% \label{sec:pack-opts}
%
% ^^X SUBSECTION: Package Load Time Options
% \subsection{Package Load Time Options}
% \label{sec:pkg-load-time-opts}
% The options described in this section \emph{must} be given
% at package load time i.e.\ as package options.
% All options which are not described below will be passed to
% \pkg{mathtools}~\autocite{mathtools},
% see its documentation for more information.
%
% \DescribeOption{bm} The \verb|bm| option
% loads the \pkg{bm}~\autocite{bm} package which provides a better version of
% the \cs{boldsymbol} command.
%
%
%
% 
% The options
% \DescribeOption{no-vector} "no-vector",
% \DescribeOption{no-abs-shorthands} "no-abs-shorthands",
% \DescribeOption{no-operators} "no-operators",
% \DescribeOption{no-crvector} "no-crvector",
% and \DescribeOption{no-matrix} "no-matrix"
% disable the definition of the predefined commands described in sections~\ref{sec:vector-calculus},
% \ref{sec:shorthands},
% \ref{sec:delim-ops},
% \ref{sec:rc-vectors},
% and~\ref{sec:matrices}
% respectively.
%
% \DescribeOption{nopredef}
% The option |nopredef| achieves the same as the three
% |no-|\meta{functionality} options above.
% It accepts multiple values, valid option values are:
% |vector|,
% |abs|,
% |operators|,
% |crvector|,
% |matrix|,
% and |all|.
% The values
% |abs|,
% |operators|,
% and |vector|
% disable the predefined
% document level commands for
% delimited operators,
% vector calculus
% and the shorthands for absolute value and norm respectively.
% The values "crvector" and "matrix" disable the shorthand commands
% for row and column vectors, and simple matrices respectively.
% The special value |all| disables all of them.
%
% The option accepts multiple values which can be given as a comma separated
% list, or as multiple key-value options, like in the examples below:
% \begin{verbatim}
% \usepackage[nopredef={vector,abs,operators}]{moremath}
% \end{verbatim}
%   This is equivalent to
% \begin{verbatim}
% \usepackage[nopredef=all]{moremath}
% \end{verbatim}
%   and to:
% \begin{verbatim}
% \usepackage[nopredef=vector,nopredef=abs,nopredef=operators]{moremath}
% \end{verbatim}
%
% The command \cs{NewDelimitedOperator} is not affected by any of the above
% settings.
%
% ^^X SUBSECTION: General Options
% \subsection{General Options}
% \label{sec:pkg-general-options}
%
% The options described in this section
% \emph{must not} be given as package options,
% instead they should be set using \cs{moremathsetup}
% or given as optional argument to the commands described later.
% \begin{function}[updated = 2024-07-15]{\moremathsetup}
% \begin{syntax}
%   \cs{moremathsetup}\marg{kv~list}
% \end{syntax}
% Sets the options specified in the \meta{key-value list},
% the assignment is local to the current group.
% If a \meta{value} contains a comma it needs to be wrapped in braces.
% This command may be used anywhere in the document after \pkg{moremath}
% has been loaded.
% \end{function}
%
% \subsubsection{Options Affecting Vector Calculus Operators}
% \label{sec:options-vec-calc}
%
% \DescribeOption{nabla}
% The option |nabla| sets the symbol to use by the document level commands
% described in section~\ref{sec:vector-calculus} to use for the nabla.
% It accepts a list of \meta{tokens}. Its default value is |\nabla|.
%
% \DescribeOption{arrownabla}
% The option |arrownabla| puts a small arrow over the gradient operator symbol.
% Its default value is |false|.
%
% \DescribeOption{boldnabla}
% The option |boldnabla| makes the nabla symbol bold.
% If the |bm| package option has been given the |\boldsymbol| command from
% the \pkg{bm} package is used for the bold symbol,
% otherwise the \pkg{amsmath}~\autocite{amsmath} version is used.
%
% \DescribeOption{grad-op}
% The option |grad-op| may be used to overwrite,
% the built in version of the gradient operator, it accepts a \meta{token list}.
% Use at your own responsibility.
%
% \DescribeOption{laplacian-symb}
% The option |laplacian-symb| sets the symbol to use by the document level
% commands described in section~\ref{sec:vector-calculus} to use for the
% Laplace operator. It accepts a list of \meta{tokens}.
%
% \DescribeOption{delta-laplace}
% The option |delta-laplace| replaces the Laplace operator symbol
% (by default \(\laplacian\)) with a uppercase delta
% (\(\laplacian[delta-laplace=true]\)).
% Its default value is |false|.
%
% \DescribeOption{arrowlaplace}
% The option |arrowlaplace| if set to |true| makes the Laplace operator look
% like this: \(\laplacian[arrowlaplace=true]\).
%
% \DescribeOption{laplacian}
% Like the option |grad-op| above the option |laplacian| may be used to
% overwrite the built-in version of the Laplace operator.
% Use at your own responsibility.
%
% \DescribeOption{dalembert-symb}
% Like the option "nabla" this sets the symbol to use by the document level
% commands described in section~\ref{sec:dalembert-op-cmds} to use as symbol
% for the \foreignlanguage{french}{d'Alembert} operator. It accepts a list of
% \meta{tokens}. It's default value is \cs[module=amssymb,replace=false]{square}.
%
% \DescribeOption{vcenter}
% The option "vcenter" controls if certain mathematical symbols
% of the operators described in section~\ref{sec:vector-calculus} should be
% vertically centered along the math-axis.
% The default value of this option is "true".
%
%
%
% \subsubsection{Options Affecting Matrices and Vectors}
% \label{sec:options-matrix}
%
% The options in this section only affect the commands described in sections~\ref{sec:rc-vectors}
% and~\ref{sec:matrices}. To set them with \cs{moremathsetup} it is necessary
% to add the prefix |matrix / | to these options,
% so that the resulting command looks like \cs{moremathsetup}\texttt{\{matrix / \meta{option}\}}.
% When using these options inside the optional argument of the commands described
% in sections~\ref{sec:rc-vectors} and~\ref{sec:matrices}, the prefix "matrix /"
% must be omitted.
%
% \DescribeOption{delimiter}
% The option |delimiter| determines the delimiters used for the matrices,
% valid values are "p" for parenthesis, "b" for brackets, "B" for braces,
% "v" for single vertical lines (\enquote{\(\vert\)}),
% "V" for double vertical lines (\enquote{\(\Vert\)}) or empty for no delimiters.
% The default value is "{}" (empty).
%
% \DescribeOption{fill}
% The fill option determines the values an (anti-)diagonal matrix is filled with,
% outside the diagonal. The default is again empty.
%
% \DescribeOption{align}
% This option determines the alignment of the numbers inside the matrix.
% The value of this option gets passed to the optional argument of the \env{matrix*}
% or \env{smallmatrix*} family of environments defined by \pkg{mathtools}~\autocite{mathtools}.
% Valid values for both types of those environments are |l| for left alignment,
% |r| for right alignment and |c| for centered alignment. The default is |c|.
%
% \begin{texnote}
%   The non-"small" versions of the commands described in the sections~\ref{sec:rc-vectors}
%   and~\ref{sec:matrices}, accept \textcquote{mathtools}{\textelp{}
%   any column type valid in the usual \env{array} environment.}
% \end{texnote}
%
%
% \section{Delimited Operators}
% \label{sec:delim-ops}
%
% \subsection{Delimited Operators Predefined by \pkg{moremath}}
% \label{sec:delim-op-predef}
%
% \DescribeOption{no-operators}
% If the package load time option |no-operators| is not given this package defines several
% delimited mathematical operators.
%
% \begin{table}
%   \caption{%
%     Operator commands defined by the \pkg{amsmath}~\autocite{amsmath}
%     package.}
%  \label{tab:amsmath-operators}
%  \vspace{.5\baselineskip}
%   \input{\jobname-ams-op-table.tex}
% \end{table}
%
% \begin{function}{
%   \parccos, \barccos, \Barccos, \varccos, \Varccos,
% }
% \begin{syntax}
%   \cs{parccos} \oarg{size cmd} \marg{contents}

%   \cs{parccos} \oarg{size cmd} \textasciicircum\marg{superscript} \marg{contents}
%
%   \cs{parccos} \oarg{size cmd} \_\marg{subscript} \marg{contents}
%
%   \cs{parccos} \oarg{size cmd} \textasciicircum\marg{superscript} \_\marg{subscript} \marg{contents}
%
%   \cs{parccos}* \marg{contents}
%
%   \cs{parccos}* \textasciicircum\marg{superscript} \marg{contents}
%
%   \cs{parccos}* \_\marg{subscript} \marg{contents}
%   \cs{parccos}* \textasciicircum\marg{superscript} \_\marg{subscript} \marg{contents}
%
%
%   \cs[no-index]{\meta{prefix}\meta{op name}} \oarg{size cmd} \marg{contents}
%   \cs[no-index]{\meta{prefix}\meta{op name}} \oarg{size cmd} \textasciicircum\marg{superscript}  \marg{contents}
%   \cs[no-index]{\meta{prefix}\meta{op name}} \oarg{size cmd}  \_\marg{subscript} \marg{contents}
%   \cs[no-index]{\meta{prefix}\meta{op name}} \oarg{size cmd} \textasciicircum\marg{superscript} \_\marg{subscript} \marg{contents}
%   \cs[no-index]{\meta{prefix}\meta{op name}}* \marg{contents}
%   \cs[no-index]{\meta{prefix}\meta{op name}}* \textasciicircum\marg{superscript}  \marg{contents}
%   \cs[no-index]{\meta{prefix}\meta{op name}}* \_\marg{subscript} \marg{contents}
%   \cs[no-index]{\meta{prefix}\meta{op name}}* \textasciicircum\marg{superscript} \_\marg{subscript} \marg{contents}
% \end{syntax}
% For all of the operators predefined by \pkg{amsmath}~\autocite{amsmath},
% which are shown in table~\ref{tab:amsmath-operators},
% \pkg{moremath} declares delimited versions.
% The name of those commands follows the scheme \cs[no-index]{\meta{prefix}\meta{op}},
% where \meta{prefix} is one of |p|, |b|, |B|, |v|, or |V|
% and \meta{op} is the name of one of the operators shown in table~\ref{tab:amsmath-operators}.
%
% The \meta{prefix}es |p|, |b|, |B|, |v|, and |V| stand for
% parenthesis, brackets, braces, single vertical lines (\enquote{\(\vert\)}), and double vertical lines
% (\enquote{\(\Vert\)}) respectively.
%
% The commands accept a \meta{size cmd} optional argument, which is usually one of
% \cs{big}, \cs{Big}, \cs{bigg} and \cs{Bigg}.
% These \meta{size cmd}s are used to change the size of the delimiters.
%
% The commands also accept sub- and superscripts,
% which have to be issued \emph{after} the optional argument (if present),
% but before the mandatory argument \meta{contents}.
%
% The starred variant uses automatic scaling for the delimiters depending on the height of its contents.
% \end{function}
%
% \paragraph{Examples}
%
% The following examples showcase the use of those predefined delimited operators:
% \begin{enumerate}
% \item Different delimited operators without any scaling:
%
% \begin{verbatim}
% \[
%   \pcos{x} \times \bcos{y}
%   \times \Bcos{z} \times \vcos{a}
%   \times \Vcos{b}
% \]
% \end{verbatim}
% \[
%   \pcos{x} \times \bcos{y}
%   \times \Bcos{z} \times \vcos{a}
%   \times \Vcos{b}
% \]
%
% \item Delimited operator with automatic scaling:
%
% \begin{minipage}{.49\linewidth}
% \begin{verbatim}
% \[
%   \pcos*{\frac{x^{2}}{2}}
% \]
% \end{verbatim}
% \end{minipage}\hfill
% \begin{minipage}{.49\linewidth}
% \[
%     \pcos*{\frac{x^{2}}{2}}
%  \]
% \end{minipage}
%
% \item Delimited operator with manual scaling:
%
% \begin{minipage}{.49\linewidth}
% \begin{verbatim}
% \[
%   \pcos[\Big]{\frac{x^{2}}{2}}
% \]
% \end{verbatim}
% \end{minipage}\hfill
% \begin{minipage}{.49\linewidth}
% \[
%   \pcos[\Big]{\frac{x^{2}}{2}}
% \]
% \end{minipage}
%
% \item Delimited operator with subscript:
%
% \begin{minipage}{.49\linewidth}
% \begin{verbatim}
% \[
%   \plog_{10}{1+x}
% \]
% \end{verbatim}
% \end{minipage}\hfill
% \begin{minipage}{.49\linewidth}
% ^^X XXX: Using an underscore is broken here (because of l3doc???) therefore I
% ^^X use "\sb" instead.
% \[
%   \plog\sb{10}{1+x}
% \]
% \end{minipage}
%
% \item Delimited operator with superscript:
%
% \begin{minipage}{.49\linewidth}
% \begin{verbatim}
% \[
%   \pcos^{2}{x}
% \]
% \end{verbatim}
% \end{minipage}\hfill
% \begin{minipage}{.49\linewidth}
% \[
%   \pcos^{2}{x}
% \]
% \end{minipage}
%
% \item Delimited operator with both sub- and superscript and manual scaling:
%
% \begin{minipage}{.49\linewidth}
% \begin{verbatim}
% \[
%   \pcos[\Big]^{2}_{x}{\frac{x}{2}}
% \]
% \end{verbatim}
% \end{minipage}\hfill
% \begin{minipage}{.49\linewidth}
% ^^X XXX: Again use "\sb" instead of "_" because of l3doc
% \[
%   \pcos[\Big]^{2}\sb{x}{\frac{x}{2}}
% \]
% \end{minipage}
% \end{enumerate}
%
%
% \subsection{Declaring New Delimited Operators}
% \label{sec:delim-op-new}
%
% \begin{function}{\DeclareDelimitedOperator}
% \begin{syntax}
%   \cs{DeclareDelimitedOperator}\marg{new op}\marg{op}\marg{delim}
% \end{syntax}
% Creates a new delimited operator,
% the name of the new command will be \meta{new op}.
% The \meta{op} is the command of the operator to use,
% which is usually a command declared with
% \cs[module=amsmath,replace=false]{DeclareMathOperator}.
% \meta{delim} is the command to use as paired delimiter,
% it is expected to behave like a paired delimiter declared by
% \pkg{mathtools}~\autocite{mathtools}
% \cs[module=mathtools,replace=false]{DeclarePairedDelimiter}.
% \end{function}
%
% \paragraph{Example: Creating a New Delimited Operator}
% The following code creates a new operator and a paired delimiter
% and uses it afterwards to declare a paired operator.
% \begin{verbatim}
% \documentclass{scrartcl}
%
% \DeclareMathOperator{\glorb}{glorb}
% \DeclarePairedDelimiter{\inparen}{\lparen}{\rparen}
% \DeclarePairedOperator{\pglorb}{\glorb}{\inparen}
%
% \begin{document}
% \[
%   \pglorb{a}
% \]
% \end{document}
% \end{verbatim}
% The result then looks like this:
% \[
%   \pglorb{a}
% \]
%
%
% \section{Vector Calculus Operators}
% \label{sec:vector-calculus}
%
% \DescribeOption{no-vector}
% The commands in this section are only declared if the option |no-vector|
% has not been given to the package as a load time option.
%
% \DescribeOption{vcenter}
% The option "vcenter" controls if the symbols for the operators described below
% should be vertically centered along the math axis. Its default value is "true".
%
% This option only shows its effects if other options like "arrownabla",
% "arrowlaplace", or "boldnabla" are set to "true".
% Like in the example below:
% \begin{quote}
%   \begin{verbatim}
% \begin{gather*}
%   \grad f(x) \quad \grad[arrownabla,vcenter=false] f(x)
%   \quad \grad[arrownabla,vcenter=true] f(x)\\
%   \laplacian f(x) \quad \laplacian[arrowlaplace,vcenter=false] f(x)
%   \quad \laplacian[arrowlaplace,vcenter=true] f(x)
% \end{gather*}
%   \end{verbatim}
%   \begin{gather*}
%     \grad f(x) \quad \grad[arrownabla,vcenter=false] f(x)
%     \quad \grad[arrownabla,vcenter=true] f(x)\\
%     \laplacian f(x) \quad \laplacian[arrowlaplace,vcenter=false] f(x)
%     \quad \laplacian[arrowlaplace,vcenter=true] f(x)
%   \end{gather*}
% \end{quote}
%
% All of the commands described in this section take key-value options
% as optional argument, which are described in section~\ref{sec:options-vec-calc}
%
% \subsection{Gradient Operator Commands}
% \label{sec:grad-op-cmds}
%
% \subsubsection{Standalone Operator Command}
% \label{sec:grad-op-standalone}
%
% \begin{function}[updated=2024-07-08]{
%   \grad,
% }
%   \begin{syntax}
%     \cs{grad} \oarg{kv opts}
%     \cs{grad} \oarg{kv opts} \_\marg{subscript}
%   \end{syntax}
%   The \cs{grad} command produces a gradient operator
%   looking like this \enquote{\(\grad\)} by default.
%   The optional argument \meta{kv opts} accepts the key-value options
%   described in section~\ref{sec:options-vec-calc},
%   whitespace between the command name
%   and \oarg{kv~opts} is \emph{not allowed}.
%   An optional subscript using "_" may be given after the optional argument.
% \end{function}
%
% \paragraph{Examples of Use}
% \begin{enumerate}
%   \item Standalone gradient operator (with and without subscript):
%
%         \begin{minipage}{.49\linewidth}
%           \begin{verbatim}
% \[
%   \grad f(x), \quad \grad_{x} f(x)
% \]
%           \end{verbatim}
%         \end{minipage}\hfill
%         \begin{minipage}{.49\linewidth}
%           \[
%             \grad f(x), \quad \grad\sb{x} f(x)
%           \]
%         \end{minipage}
%
%   \item Bold version of the gradient operator:
%
%         \begin{minipage}{.49\linewidth}
%           \begin{verbatim}
% \[
%   \grad[boldnabla] f(x)
% \]
%           \end{verbatim}
%         \end{minipage}\hfill
%         \begin{minipage}{.49\linewidth}
%           \[
%             \grad[boldnabla] f(x)
%           \]
%         \end{minipage}
%
%   \item Gradient operator with an arrow:
%
%         \begin{minipage}{.49\linewidth}
%           \begin{verbatim}
% \[
%   \grad[arrownabla] f(x)
% \]
%           \end{verbatim}
%         \end{minipage}
%         \begin{minipage}{.49\linewidth}
%           \[
%             \grad[arrownabla] f(x)
%           \]
%         \end{minipage}
% \end{enumerate}
%
% \subsubsection{Operators with Delimiters}
% \label{sec:grad-op-delim}
%
% \begin{function}{
%   \pgrad,
%   \bgrad,
%   \Bgrad,
%   \vgrad,
%   \Vgrad
% }
% \begin{syntax}
%   \cs[no-index]{\meta{delim}grad} \oarg{size cmd} \marg{content}
%   \cs[no-index]{\meta{delim}grad} \oarg{kv opts} \marg{content}
%   \cs[no-index]{\meta{delim}grad} \oarg{size cmd} \_\marg{subscript} \marg{content}
%   \cs[no-index]{\meta{delim}grad} \oarg{kv opts} \_\marg{subscript} \marg{content}
%   \cs[no-index]{\meta{delim}grad}* \oarg{kv opts} \marg{content}
%   \cs[no-index]{\meta{delim}grad}* \oarg{kv opts} \_\marg{subscript} \marg{content}
% \end{syntax}
% The \cs[no-index]{\meta{delim}grad} family of commands produces
% gradient operator which is followed by \meta{contents} inside delimiters.
% The delimiter is determined by the first letter of the command \meta{delim},
% which may be "p" for parenthesis, "b" for brackets, "B" for braces,
% "v" for a single vertical line (\enquote{\(\vert\)}), or "V" for a double
% vertical line (\enquote{\(\Vert\)}).
%
% The commands accept either a \meta{size command} as optional argument,
% which gets passed to \pkg{mathtools}~\autocite{mathtools} paired delimiter
% or a list of \meta{key-value option}s,
% the \meta{kv opts} \emph{must} be given using the complete syntax,
% i.e.\ \meta{key}"="\meta{value},
% shorthands for options with an implicit default value (|arrownabla|),
% will not work here.
% The \meta{size~command} is usually one of \cs{big}, \cs{Big}, \cs{bigg} and
% \cs{Bigg}.
% Valid \meta{kv opts} are all options described in section~\ref{sec:options-vec-calc}
% and the key "scale" which accepts a \meta{size cmd}.
%
% \begin{quote}
%   \textbf{Note:}\\
%   Do not mix \meta{kv opts} and \meta{size cmd},
%   use either |\pgrad[\big]{f(x)}| or |\pgrad[arrownabla=true,scale=\big]{f(x)}|.
% \end{quote}
%
% \medskip\noindent
% An optional \meta{subscript} may be given between the optional argument,
% and \meta{contents}. One use case for this subscript is to write formulae
% using the so called Feynman-notation, where the gradient operator acts
% only on one variable.
%
% The starred version of the commands automatically scale the delimiters
% with its contents.
% \end{function}
%
% \paragraph{Examples:}
% \begin{enumerate}
%     \item Gradient operator with non-scaled delimiters:
%
%           \begin{minipage}{.49\linewidth}
%             \begin{verbatim}
% \[
%   \pgrad{1+\vec{x}}
% \]
%             \end{verbatim}
%           \end{minipage}\hfill
%           \begin{minipage}{.49\linewidth}
%             \[
%               \pgrad{1+\vec{x}}
%             \]
%           \end{minipage}
%
%     \item Bold version:
%
%           \begin{minipage}{.49\linewidth}
%             \begin{verbatim}
% \[
%   \pgrad[boldnabla=true]{1+\vec{x}}
% \]
%             \end{verbatim}
%           \end{minipage}\hfill
%           \begin{minipage}{.49\linewidth}
%             \[
%               \pgrad[boldnabla=true]{1+\vec{x}}
%             \]
%           \end{minipage}
%
%     \item Gradient operator with automatically scaled delimiters:
%
%           \begin{minipage}{.49\linewidth}
%             \begin{verbatim}
% \[
%   \pgrad*{\frac{1}{x}}
% \]
%             \end{verbatim}
%           \end{minipage}\hfill
%           \begin{minipage}{.49\linewidth}
%             \[
%               \pgrad*{\frac{1}{x}}
%             \]
%           \end{minipage}
%
%     \item Manually scaled version:
%
%           \begin{minipage}{.49\linewidth}
%             \begin{verbatim}
% \begin{gather*}
%   \pgrad[\Big]{\frac{1}{x}}\\[.5ex]
%   \pgrad[scale=\Big]{\frac{1}{x}}
% \end{gather*}
%             \end{verbatim}
%           \end{minipage}\hfill
%           \begin{minipage}{.49\linewidth}
%             \begin{gather*}
%               \pgrad[\Big]{\frac{1}{x}}\\[.5ex]
%               \pgrad[scale=\Big]{\frac{1}{x}}
%             \end{gather*}
%           \end{minipage}
%
%     \item Feynman-Notation:
%
%           \begin{minipage}{.49\linewidth}
%             \begin{verbatim}
% \[
%   \pgrad_{x}{x+y+z}
% \]
%             \end{verbatim}
%           \end{minipage}\hfill
%           \begin{minipage}{.49\linewidth}
%             \[
%               \pgrad\sb{x}{x+y+z}
%             \]
%           \end{minipage}
% \end{enumerate}
%
% \subsection{Divergence Operator Commands}
% \label{sec:div-op-cmds}
%
% \subsubsection{Standalone Operator Command}
% \label{sec:div-op-standalone}
%
% \begin{function}[updated=2024-07-08]{\divergence}
%   \begin{syntax}
%     \cs{divergence} \oarg{kv opts}
%     \cs{divergence} \oarg{kv opts} \_\marg{subscript}
%   \end{syntax}
%   The \cs{divergence} command produces the divergence operator
%   \enquote{\(\divergence\)},
%   its usage is analogous to the use of the \cs{grad} command,
%   which is described in section~\ref{sec:grad-op-standalone}.
% \end{function}
%
% \paragraph{Examples}
% \begin{enumerate}
%   \item Standalone divergence operator
%
%         \begin{minipage}{.49\linewidth}
%           \begin{verbatim}
% \[
%   \divergence f(x)
% \]
%           \end{verbatim}
%         \end{minipage}\hfill
%         \begin{minipage}{.49\linewidth}
%           \[\divergence f\]
%         \end{minipage}
%
%     \item Standalone divergence operator
%           with an arrow over the gradient operator
%
%           \begin{minipage}{.49\linewidth}
%             \begin{verbatim}
% \[
%   \divergence[arrownabla] f(x)
% \]
%             \end{verbatim}
%           \end{minipage}\hfill
%           \begin{minipage}{.49\linewidth}
%             \[\divergence[arrownabla] f(x)\]
%           \end{minipage}
%
%     \item Standalone divergence operator with subscript
%
%           \begin{minipage}{.49\linewidth}
%             \begin{verbatim}
% \[
%   \divergence_{x} f(x)
% \]
%             \end{verbatim}
%           \end{minipage}\hfill
%           ^^X We need to use \sb because of the subscript package
%           \begin{minipage}{.49\linewidth}
%             \[\divergence\sb{x} f(x)\]
%           \end{minipage}
% \end{enumerate}
% \subsubsection{Operators with Delimiters}
% \label{sec:div-op-delim}
%
% \begin{function}{
%   \pdiv,
%   \bdiv,
%   \Bdiv,
%   \vdiv,
%   \Vdiv
% }
% \begin{syntax}
%   \cs[no-index]{\meta{delim}div} \oarg{size cmd} \marg{content}
%   \cs[no-index]{\meta{delim}div} \oarg{kv opts} \marg{content}
%   \cs[no-index]{\meta{delim}div} \oarg{size cmd} \_\marg{subscript} \marg{content}
%   \cs[no-index]{\meta{delim}div} \oarg{kv opts} \_\marg{subscript} \marg{content}
%   \cs[no-index]{\meta{delim}div}* \oarg{kv opts} \marg{content}
%   \cs[no-index]{\meta{delim}div}* \oarg{kv opts} \_\marg{subscript} \marg{content}
% \end{syntax}
% The \cs[no-index]{\meta{delim}div} family of commands produces the divergence
% operator with its arguments placed inside delimiters.
% The usage of these commands is analogous to the \cs[no-index]{\meta{delim}grad}
% family of commands described in section~\ref{sec:grad-op-delim}.
% \end{function}
%
% \paragraph{Examples}
%
% \begin{enumerate}
%   \item Divergence operator with parenthesis and no scaling
%
%         \begin{minipage}{.49\linewidth}
%           \begin{verbatim}
% \[
%   \pdiv{1+x}
% \]
%           \end{verbatim}
%         \end{minipage}\hfill
%         \begin{minipage}{.49\linewidth}
%           \[\pdiv{1+x}\]
%         \end{minipage}
%     \item Bold version with manual scaling and subscript
%
%           \begin{verbatim}
% \[
%   \pdiv[boldnabla=true,scale=\Big]_{x}{1 + \frac{1}{x}}
% \]
%             \end{verbatim}
%             \[\pdiv[boldnabla=true,scale=\Big]\sb{x}{1+\frac{1}{x}}\]
%
%     \item Automatic scaling
%
%           \begin{minipage}{.49\linewidth}
%             \begin{verbatim}
% \[
%   \pdiv*{1 + \frac{1}{x}}
% \]
%             \end{verbatim}
%           \end{minipage}\hfill
%           \begin{minipage}{.49\linewidth}
%             \[\pdiv*{1 + \frac{1}{x}}\]
%           \end{minipage}
% \end{enumerate}
%
% \subsection{Curl Operator Commands}
% \label{sec:curl-op-cmds}
%
% \subsubsection{Standalone Operator Command}
% \label{sec:curl-op-standalone}
%
% \begin{function}[updated=2024-07-08]{\curl}
%   \begin{syntax}
%     \cs{curl} \oarg{kv opts}
%     \cs{curl} \oarg{kv opts} \_\marg{subscript}
%   \end{syntax}
%   The \cs{curl} command produces the curl operator \enquote{\(\curl\)},
%   its usage is analogous to the use of the \cs{grad} command
%   described in section~\ref{sec:grad-op-standalone}.
% \end{function}
%
% \paragraph{Examples}
% \begin{enumerate}
%  \item Standalone curl operator
%
%       \begin{minipage}{.49\linewidth}
%         \begin{verbatim}
% \[
%   \curl f(x)
% \]
%         \end{verbatim}
%       \end{minipage}\hfill
%       \begin{minipage}{.49\linewidth}
%         \[\curl f(x)\]
%       \end{minipage}
%
%     \item Standalone curl operator with an arrow over the gradient operator
%
%           \begin{minipage}{.49\linewidth}
%             \begin{verbatim}
% \[
%   \curl[arrownabla] f(x)
% \]
%             \end{verbatim}
%           \end{minipage}\hfill
%           \begin{minipage}{.49\linewidth}
%             \[\curl[arrownabla] f(x)\]
%           \end{minipage}
%
%     \item Standalone curl operator with subscript
%
%           \begin{minipage}{.49\linewidth}
%             \begin{verbatim}
% \[
%   \curl_{x} f(x,y)
% \]
%             \end{verbatim}
%           \end{minipage}\hfill
%           \begin{minipage}{.49\linewidth}
%             \[\curl\sb{x} f(x,y)\]
%           \end{minipage}
% \end{enumerate}
%
% \subsubsection{Operators with Delimiters}
% \label{sec:curl-op-delim}
%
% \begin{function}{
%   \pcurl,
%   \bcurl,
%   \Bcurl,
%   \vcurl,
%   \Vcurl
% }
% \begin{syntax}
%   \cs[no-index]{\meta{delim}curl} \oarg{size cmd} \marg{content}
%   \cs[no-index]{\meta{delim}curl} \oarg{kv opts} \marg{content}
%   \cs[no-index]{\meta{delim}curl} \oarg{size cmd} \_\marg{subscript} \marg{content}
%   \cs[no-index]{\meta{delim}curl} \oarg{kv opts} \_\marg{subscript} \marg{content}
%   \cs[no-index]{\meta{delim}curl}* \oarg{kv opts} \marg{content}
%   \cs[no-index]{\meta{delim}curl}* \oarg{kv opts} \_\marg{subscript} \marg{content}
% \end{syntax}
% The \cs[no-index]{\meta{delim}curl} family of commands produce the curl operator
% with its arguments placed inside delimiters.
% The usage of these commands is analogous to the \cs[no-index]{\meta{delim}grad}
% family of commmands described in section~\ref{sec:grad-op-delim}.
% \end{function}
%
% \paragraph{Examples}
% \begin{enumerate}
%   \item Curl operator with parenthesis without scaling
%
%           \begin{minipage}{.49\linewidth}
%             \begin{verbatim}
% \[
%   \pcurl{1+x}
% \]
%             \end{verbatim}
%           \end{minipage}\hfill
%           \begin{minipage}{.49\linewidth}
%             \[\pcurl{1+x}\]
%           \end{minipage}
%
%   \item Bold version with manual scaling and subscript
%
%         \begin{verbatim}
% \[
%   \pcurl[boldnabla=true,scale=\Big]_{x}{1 + \frac{1}{x}}
% \]
%         \end{verbatim}
%         \[
%           \pcurl[boldnabla=true,scale=\Big]\sb{x}{1 + \frac{1}{x}}
%         \]
%
%   \item Automatic scaling
%
%         \begin{minipage}{.49\linewidth}
%           \begin{verbatim}
% \[
%   \pcurl*{1 + \frac{1}{x}}
% \]
%           \end{verbatim}
%         \end{minipage}\hfill
%         \begin{minipage}{.49\linewidth}
%           \[\pcurl*{1 + \frac{1}{x}}\]
%         \end{minipage}
% \end{enumerate}
% \changes{v0.2.0}{\GetVersionDate{v0.2.0}}{%
% Fixed example to match description.
% }
%
%
% \subsection{Laplace Operator Commands}
% \label{sec:laplace-op-cmds}
%
% This section describes commands which can be used to typeset a Laplace
% operator.
%
%
% Like the commands described in sections~\ref{sec:grad-op-cmds},
% \ref{sec:div-op-cmds}, and~\ref{sec:curl-op-cmds}
% the commands in this section accept key-value options via an optional
% argument.
% There is some deviation from the options compared to the above commands:
% The "arrownabla" option is ignored,
% instead the "arrowlaplace" option produces an arrow over the operator.
% The "boldnabla" option on the other hand is not ignored.
% Finally the "delta-laplace" option replaces the symbol used
% for the operator from \(\laplacian\) to \(\laplacian[delta-laplace]\)
%
% \subsubsection{Standalone Operator Command}
% \label{sec:laplace-op-standalone}
%
%
% \begin{function}[updated=2024-07-08]{\laplacian}
%   \begin{syntax}
%     \cs{laplacian} \oarg{kv opts}
%     \cs{laplacian} \oarg{kv opts} \_\marg{subscript}
%   \end{syntax}
%   The \cs{laplacian} command produces a Laplace operator,
%   which looks by default like this: \(\laplacian\).
%
%   Its interface is analogous to the \cs{grad}, \cs{divergence},
%   and \cs{curl} commands described above,
%   with the difference in key-value options described at the start of this
%   subsection.
% \end{function}
%
%
% \subsubsection{Operators with Delimiters}
% \label{sec:laplace-op-delim}
%
% \begin{function}{
%   \plaplacian,
%   \blaplacian,
%   \Blaplacian,
%   \vlaplacian,
%   \Vlaplacian
% }
% \begin{syntax}
%   \cs[no-index]{\meta{delim}laplacian} \oarg{size cmd} \marg{content}
%   \cs[no-index]{\meta{delim}laplacian} \oarg{kv opts} \marg{content}
%   \cs[no-index]{\meta{delim}laplacian} \oarg{size cmd} \_\marg{subscript} \marg{content}
%   \cs[no-index]{\meta{delim}laplacian} \oarg{kv opts} \_\marg{subscript} \marg{content}
%   \cs[no-index]{\meta{delim}laplacian}* \oarg{kv opts} \marg{content}
%   \cs[no-index]{\meta{delim}laplacian}* \oarg{kv opts} \_\marg{subscript} \marg{content}
% \end{syntax}
% \end{function}
%
% \paragraph{Examples}
% \begin{enumerate}
%   \item Laplace operator delimited by parenthesis without scaling
%
%         \begin{minipage}{.49\linewidth}
%           \begin{verbatim}
% \[
%   \plaplacian{1+x}
% \]
%           \end{verbatim}
%         \end{minipage}\hfill
%         \begin{minipage}{.49\linewidth}
%           \[\plaplacian{1+x}\]
%         \end{minipage}
%
%   \item Version with arrow, manual scaling and subscript
%
%         \begin{verbatim}
% \[
%   \plaplacian[arrowlaplace=true,scale=\Big]_{x}{1 + \frac{1}{x}}
% \]
%         \end{verbatim}
% \[
%   \plaplacian[arrowlaplace=true,scale=\Big]\sb{x}{1 + \frac{1}{x}}
% \]
%
%   \item Version with automatic scaling
%
%         \begin{minipage}{.49\linewidth}
%           \begin{verbatim}
% \[
%   \plaplacian*{1 + \frac{1}{x}}
% \]
%           \end{verbatim}
%         \end{minipage}\hfill
%         \begin{minipage}{.49\linewidth}
%           \[\plaplacian*{1 + \frac{1}{x}}\]
%         \end{minipage}
%
%   \item Using a delta as symbol for the Laplacian
%
%         \begin{minipage}{.49\linewidth}
%           \begin{verbatim}
% \[
%   \plaplacian[delta-laplace=true]{1+x}
% \]
%           \end{verbatim}
%         \end{minipage}\hfill
%         \begin{minipage}{.49\linewidth}
%           \[\plaplacian[delta-laplace=true]{1+x}\]
%         \end{minipage}
% \end{enumerate}
%
%
% \subsection{^^X
%   Commands Producing a \foreignlanguage{french}{d'Alembert} operator
% }
% \label{sec:dalembert-op-cmds}
%
% \subsubsection{Standalone Operator Command}
% \label{sec:dalembert-op-standalone}
%
% \begin{function}[
%   added = 2024-07-04,
%   updated=2024-07-08,
% ]{\quabla}
%   \begin{syntax}
%     \cs{quabla} \oarg{kv~opts}
%     \cs{quabla} \oarg{kv~opts} \_\marg{subscript}
%   \end{syntax}
%   The \cs{quabla} command produces
%   the \foreignlanguage{french}{d'Alembert} operator \enquote{\(\quabla\)}.
%   This command accepts an optional subscript.
% \end{function}
%
% The command is called \cs{quabla} because thats shorter and easier to type than
% \cs[no-index]{dalembertian}.
% If you want to have a command called \cs[no-index]{dalembertian},
% put the following in your documents preamble.
% \begin{verbatim}
% \NewCommandCopy\quabla\dalembertian
% \end{verbatim}
%
% \paragraph{Example of Use:}
% \begin{quote}
%   \begin{minipage}{.49\linewidth}
%     \begin{verbatim}
% \[
%   \quabla f(x)
% \]
%     \end{verbatim}
%   \end{minipage}\hfill
%   \begin{minipage}{.49\linewidth}
%     \[\quabla f(x)\]
%   \end{minipage}
% \end{quote}
%
%
% \subsubsection{Operators with Delimiters}
% \label{sec:dalembert-op-delim}
%
% \begin{function}[added = 2024-07-04]{
%   \pquabla,
%   \bquabla,
%   \Bquabla,
%   \vquabla,
%   \Vquabla,
% }
%   \begin{syntax}
%     \cs[no-index]{\meta{delim}quabla} \oarg{size~cmd} \marg{contents}
%     \cs[no-index]{\meta{delim}quabla} \oarg{kv~opts} \marg{contents}
%     \cs[no-index]{\meta{delim}quabla} \oarg{size~cmd} \_\marg{subscript} \marg{contents}
%     \cs[no-index]{\meta{delim}quabla} \oarg{kv~opts} \_\marg{subscript} \marg{contents}
%     \cs[no-index]{\meta{delim}quabla}* \oarg{kv~opts} \marg{contents}
%     \cs[no-index]{\meta{delim}quabla}* \oarg{kv~opts} \_\marg{subscript} \marg{contents}
%   \end{syntax}
%   The \cs[no-index]{\meta{delim}quabla} family of commands produce a
%   \foreignlanguage{french}{d'Alembert} operator with <contents> placed inside
%   delimiters.
%   Their usage is analogous to the \cs[no-index]{\meta{delim}grad} family of
%   commands described in section~\ref{sec:grad-op-delim}.
% \end{function}
%
% \changes{v0.2.0}{\GetVersionDate{v0.2.0}}{%
%   Fixed syntax description of standalone operators not matching reality.
% }
%
%
% \section{Row- and Column Vectors}
% \label{sec:rc-vectors}
%
% \DescribeOption{no-crvector}
% The command in this section are only declared if the option |no-crvector|
% has not been given as a package option.
%
% ^^X \DescribeOption{delimiter}
% ^^X \DescribeOption{fill}
% ^^X \DescribeOption{align}
% ^^X The commands in this section all accept key-value options as optional argument.
% Valid keys are "delimiter", "fill", and "align",
% their usage is described in section~\ref{sec:options-matrix}.
%
% \begin{function}{
%   \cvector,
%   \rvector
% }
% \begin{syntax}
%   \cs{cvector} \oarg{kv opts} \marg{clist}
%   \cs{rvector} \oarg{kv opts} \marg{clist}
% \end{syntax}
% The commands \cs{cvector} and \cs{rvector} produce row and column vectors
% respectively. Both of them accept key-value options as optional argument
% \meta{kv opts}. Valid keys and values are described in section~\ref{sec:options-matrix}.
%
% The mandatory argument \meta{clist} is a comma-separated-list,
% whose elements are the entries of the column/row vector.
% If a comma has to appear inside an entry the entire entry has to be wrapped
% in braces.
%
% The delimiter of the row or column vectors depends on the current value of
% the option |delimiter|, by default empty. The option |fill| has no effect
% on the commands and is simply ignored.
%
% \begin{texnote}
%   If you were to define your own \env{matrix*}-like environment
%   called \env{mymatrix*},
%   which has an interface compatible to \pkg{mathtools}'s \env{matrix*}
%   family of environments,
%   you could make use of it by setting the value of |delimiter| to |my|.
% \end{texnote}
% \end{function}
%
% \medbreak\noindent
% \textbf{Examples:}
% \begin{enumerate}
%   \item Column \enquote{vector} without delimiters:
%
%         \begin{minipage}{.49\linewidth}
%           \begin{verbatim}
% \[
%   \cvector{a_{1},a_{2},a_{3}}
% \]
%           \end{verbatim}
%         \end{minipage}\hfill
%         \begin{minipage}{.49\linewidth}
%           \[
%             \cvector{a_{1},a_{2},a_{3}}
%           \]
%         \end{minipage}
%
%     \item A column vectors delimited with parenthesis and different alignment:
%
%          \begin{minipage}{.49\linewidth}
%             \begin{verbatim}
% \[
%   \cvector[delimiter=p,align=c]{-1,2,3}
% \]
%             \end{verbatim}
%          \end{minipage}\hfill
%          \begin{minipage}{.49\linewidth}
%            \[
%               \cvector[delimiter=p,align=c]{-1,2,3}
%            \]
%          \end{minipage}\\[1ex]
%          \begin{minipage}{.49\linewidth}
%             \begin{verbatim}
% \[
%   \cvector[delimiter=p,align=r]{-1,2,3}
% \]
%             \end{verbatim}
%          \end{minipage}\hfill
%          \begin{minipage}{.49\linewidth}
%            \[
%               \cvector[delimiter=p,align=r]{-1,2,3}
%            \]
%          \end{minipage}\\[1ex]
%          \begin{minipage}{.49\linewidth}
%             \begin{verbatim}
% \[
%   \cvector[delimiter=p,align=l]{-1,2,3}
% \]
%             \end{verbatim}
%          \end{minipage}\hfill
%          \begin{minipage}{.49\linewidth}
%            \[
%               \cvector[delimiter=p,align=l]{-1,2,3}
%            \]
%          \end{minipage}
% \end{enumerate}
%
% \paragraph{Row- and Column Vectors with Predefined Delimiters}
%
% As vectors are commonly delimited by parenthesis, brackets, braces, etc.\
% several shorthands producing delimited vectors are also available.
% \begin{function}{
%   \pcvector,
%   \bcvector,
%   \Bcvector,
%   \vcvector,
%   \Vcvector,
%   \prvector,
%   \brvector,
%   \Brvector,
%   \vrvector,
%   \Vrvector,
%   \prvector,
%   \brvector,
%   \Brvector,
%   \vrvector,
%   \Vrvector
% }
%   \begin{syntax}
%     \cs[no-index]{\meta{delim}cvector} \oarg{kv opts} \marg{clist}
%     \cs[no-index]{\meta{delim}rvector} \oarg{kv opts} \marg{clist}
%   \end{syntax}
%   The \cs[no-index]{\meta{delim}\meta{c or r}vector} family of commands,
%   accepts a list of key-value options as optional argument \meta{kv opts}.
%   Valid keys are described in section~\ref{sec:options-matrix}.
%
%   The mandatory argument \meta{clist} is a comma-separated-list of
%   the entries of the vector.
%   If a comma needs to appear inside an entry of the vector,
%   that entry has to be wrapped in braces.
%
%   \meta{delim} may have the value "p" for parenthesis,
%   "b" for brackets, "B" for braces, "v" for a single vertical line,
%   or "V" for a double vertical line.
% \end{function}
%
% \subparagraph{Example}
% \begin{quote}
% \begin{minipage}{.49\linewidth}
%   \begin{verbatim}
% \[
%   \pcvector{a_{1},a_{2},a_{3}}
% \]
%   \end{verbatim}
% \end{minipage}\hfill
% \begin{minipage}{.49\linewidth}
%   \[
%     \pcvector{a_{1},a_{2},a_{3}}
%   \]
% \end{minipage}
% \end{quote}
%
% \subsection{Small Versions for Inline Math}
% \label{sec:rc-vectors-small}
%
% As the \env{matrix} and \env{matrix*} family of environments is unsuitable,
% for inline math,
% \pkg{mathtools}~\autocite{mathtools} provides the \env{smallmatrix}
% and \env{smallmatrix*} family of environments.
% This package provides analogous commands for row and column vectors
% to be typeset in inline math mode.
%
% \begin{function}{
%   \smallcvector,
%   \smallrvector
% }
% \begin{syntax}
%   \cs{smallcvector} \oarg{kv opts} \marg{clist}
%   \cs{smallrvector} \oarg{kv opts} \marg{clist}
% \end{syntax}
%   \cs{smallcvector} and \cs{smallrvector} produce column and row vectors
%   suitable for inline math.
%   Both commands accept the same optional key-value arguments as
%   \cs{cvector} and \cs{rvector}.
% \end{function}
%
%
% \medbreak\noindent
% \textbf{Example:}
% \begin{quote}
%   \begin{verbatim}
% An inline version of \verb|\cvector| looks like this
% \(\smallcvector{1,0}\).
%   \end{verbatim}
% An inline version of \verb|\cvector| looks like this
% \(\smallcvector{1,0}\).
% \end{quote}
%
% \begin{function}{
%   \psmallcvector,
%   \bsmallcvector,
%   \Bsmallcvector,
%   \vsmallcvector,
%   \Vsmallcvector,
%   \psmallrvector,
%   \bsmallrvector,
%   \Bsmallrvector,
%   \vsmallrvector,
%   \Vsmallrvector,
% }
% \begin{syntax}
%   \cs{psmallcvector} \oarg{kv opts} \marg{clist}
%   \cs[no-index]{psmallrvector} \oarg{kv opts} \marg{clist}
%   \cs[no-index]{\meta{delim}small\meta{c or r}vector} \oarg{kv opts} \marg{clist}
% \end{syntax}
% The \cs[no-index]{\meta{delim}small\meta{c or r}vector} family of commands
% like \cs{psmallcvector} produce small inline math version of row and column
% vectors. Their interface is identical to the commands described above.
% \end{function}
%
% \medbreak\noindent
% \textbf{Example:}\nobreak
% \begin{quote}
%   \begin{verbatim}
% An inline version of \verb|\pcvector| looks like this
% \(\psmallcvector{1,0}\).
%   \end{verbatim}
% An inline version of \verb|\pcvector| looks like this
% \(\psmallcvector{1,0}\).
% \end{quote}
%
%
% \section{Shorthands for Simple Matrices}
% \label{sec:matrices}
%
% \DescribeOption{no-matrix}
% The commands in this section are only defined if the option "no-matrix"
% \emph{has not been given} to the package at load-time.
%
%
% \subsection{(Anti-)diagonal Matrices}
% \label{sec:diagonal-matrices}
%
% \begin{function}{
%   \diagmat,
%   \antidiagmat,
% }
% \begin{syntax}
%   \cs{diagmat} \oarg{kv opts} \marg{diagonal}
%   \cs{antidiagmat} \oarg{kv opts} \marg{diagonal}
% \end{syntax}
% \cs{diagmat} and \cs{antidiagmat} produce a diagonal or anti-diagonal matrix
% respectively.
% The optional argument \meta{kv~opts} accepts the key value options
% described in section~\ref{sec:options-matrix}.
%
% The key "fill" determines the contents of the matrix entries
% which are not part of the (anti-)diagonal,
% its default value is |{}| i.e.\ empty.
% The key "align" determines the alignment of the entries inside the
% matrix, valid values are usually "l", "c", and "r",
% the default is "c".
% The key "delimiter" determines the delimiter around the matrix,
% its default value is |{}| (none).
% Valid values for the delimiters are "p" for parenthesis,
% "b" for brackets, "B" for braces, "v" for a single vertical line
% (\enquote{\(\vert\)}), and "V" for a double vertical line (\enquote{\(\Vert\)}).
% See the \pkg{mathtools} manual~\autocite{mathtools} for more information.
%
% \begin{texnote}
%   The value of "delimiter" gets inserted inside the |\begin{#1matrix*}|
%   and |\end{#1matrix*}| commands.
%   Therefore it would be possible to define your own \env{matrix*}
%   like environment,
%   called for example \env{mymatrix*} and set "delimiter=my"
%   to make use of it.
% \end{texnote}
%
% The mandatory argument \meta{diagonal} has to be a comma separated list
% of the entries of the (anti-)diagonal.
% If an entry of the diagonal needs to contain a comma "," the entire entry
% has to be wrapped in braces.
% \end{function}
%
% \medskip\noindent
% \textbf{Examples:}
% \begin{enumerate}
%   \item A diagonal and an anti-diagonal matrix without delimiters:
%
%        \begin{minipage}{.49\linewidth}
%           \begin{verbatim}
% \begin{gather*}
%   \diagmat{1,2,3}\\[.5ex]
%   \antidiagmat{1,2,3}
% \end{gather*}
%           \end{verbatim}
%        \end{minipage}\hfill
%        \begin{minipage}{.49\linewidth}
%          \begin{gather*}
%             \diagmat{1,2,3}\\[.5ex]
%             \antidiagmat{1,2,3}
%          \end{gather*}
%        \end{minipage}
%     \item A diagonal matrix delimited by parenthesis filled with zeros:
%
%           \begin{minipage}{.49\linewidth}
%             \begin{verbatim}
% \[
%   \diagmat[delimiter=p,fill=0]{1,2,3}
% \]
%             \end{verbatim}
%           \end{minipage}\hfill
%           \begin{minipage}{.49\linewidth}
%             \[
%                \diagmat[delimiter=p,fill=0]{1,2,3}
%             \]
%           \end{minipage}
%
%     \item A diagonal matrix filled with a symbol:
%
%       \begin{minipage}{.49\linewidth}
%         \begin{verbatim}
% \[
%   \diagmat[fill=\square]{1,2,3}
% \]
%         \end{verbatim}
%       \end{minipage}\hfill
%       \begin{minipage}{.49\linewidth}
%         \[\diagmat[fill=\square]{1,2,3}\]
%       \end{minipage}
%
%   \item Diagonal matrices with different alignment:
%
%         \begin{verbatim}
% \[
%   \diagmat[delimiter=p,align=c,fill=0]{-1,-2,-3} \qquad
%   \diagmat[delimiter=p,align=l,fill=0]{-1,-2,-3} \qquad
%   \diagmat[delimiter=p,align=r,fill=0]{-1,-2,-3}
% \]
%         \end{verbatim}
%         \[
%           \diagmat[delimiter=p,align=c,fill=0]{-1,-2,-3} \qquad
%           \diagmat[delimiter=p,align=l,fill=0]{-1,-2,-3} \qquad
%           \diagmat[delimiter=p,align=r,fill=0]{-1,-2,-3}
%         \]
% \end{enumerate}
%
%
% \paragraph{Shorthands for (Anti-)diagonal Matrices with Delimiters}
%
% As matrices are more often than not written inside of delimiters,
% \pkg{moremath} provides several shorthands for producing those matrices
% without having to set the "delimiter" key explicitly.
% \begin{function}{
%   \pdiagmat,
%   \bdiagmat,
%   \Bdiagmat,
%   \vdiagmat,
%   \Vdiagmat
% }
% \begin{syntax}
%   \cs{pdiagmat} \oarg{kv opts} \marg{diagonal}
%   \cs{bdiagmat} \oarg{kv opts} \marg{diagonal}
%   \cs{Bdiagmat} \oarg{kv opts} \marg{diagonal}
%   \cs{vdiagmat} \oarg{kv opts} \marg{diagonal}
%   \cs{Vdiagmat} \oarg{kv opts} \marg{diagonal}
% \end{syntax}
% The \cs[no-index]{\meta{delim}diagmat} family of commands provides
% shorthands for producing a delimited (anti-)diagonal matrix
% without having to set the "delimiter" key explicitly every time.
% The pre-set "delimiter" may be overwritten by explicitly passing "delimiter"
% as a key-value option.
%
% The optional argument \meta{kv~opts} accepts the same key-value arguments
% as \cs{diagmat}.
% \end{function}
%
% \medskip\noindent
% \textbf{Example:}
% \begin{quote}
%   \begin{minipage}{.49\linewidth}
%     \begin{verbatim}
% \[
%   \pdiagmat{1,2,3}
% \]
%     \end{verbatim}
%   \end{minipage}\hfill
%   \begin{minipage}{.49\linewidth}
%     \[
%       \pdiagmat{1,2,3}
%     \]
%   \end{minipage}
% \end{quote}
%
% \begin{function}{
%   \pantidiagmat,
%   \bantidiagmat,
%   \Bantidiagmat,
%   \vantidiagmat,
%   \Vantidiagmat
% }
% \begin{syntax}
%   \cs{pantidiagmat} \oarg{kv opts} \marg{diagonal}
%   \cs{bantidiagmat} \oarg{kv opts} \marg{diagonal}
%   \cs{Bantidiagmat} \oarg{kv opts} \marg{diagonal}
%   \cs{vantidiagmat} \oarg{kv opts} \marg{diagonal}
%   \cs{Vantidiagmat} \oarg{kv opts} \marg{diagonal}
% \end{syntax}
% The \cs[no-index]{\meta{delim}antidiagmat} family of commands behaves
% like the \cs[no-index]{\meta{delim}diagmat} commands described above,
% except they produce anti-diagonal matrices.
% \end{function}
%
% \medskip\noindent
% \textbf{Example:}
% \begin{quote}
%   \begin{minipage}{.49\linewidth}
%     \begin{verbatim}
% \[
%   \pantidiagmat{1,2,3}
% \]
%     \end{verbatim}
%   \end{minipage}
%   \begin{minipage}{.49\linewidth}
%     \[
%       \pantidiagmat{1,2,3}
%     \]
%   \end{minipage}
% \end{quote}
%
%
% \subsubsection{Small Versions for Inline Math}
% \label{sec:matrices-small}
%
% \pkg{mathtools} defines special matrix environments for use in
% inline math mode,
% the \env{smallmatrix} and \env{smallmatrix*} family of environments.
% \pkg{moremath} provides for the case of typesetting an (anti-)diagonal
% matrix several commands which utilize these inline math versions.
%
% \begin{function}{
%   \smalldiagmat,
%   \smallantidiagmat,
% }
% \begin{syntax}
%   \cs{smalldiagmat} \oarg{kv opts} \marg{diagonal}
%   \cs{smallantidiagmat} \oarg{kv opts} \marg{diagonal}
% \end{syntax}
% The \cs{smalldiagmat} and \cs{smallantidiagmat} commands behave like
% their non-"small" counterpart described above,
% see their description for more information.
% \end{function}
%
% \medskip\noindent
% \textbf{Example:}
% \begin{quote}
%   \begin{verbatim}
% An inline version of a diagonal matrix looks like this
% \(\smalldiagmat{1,1}\).
%   \end{verbatim}
% An inline version of a diagonal matrix looks like this
% \(\smalldiagmat{1,1}\).
% \end{quote}
%
% \begin{function}{
%   \psmalldiagmat,
%   \bsmalldiagmat,
%   \Bsmalldiagmat,
%   \vsmalldiagmat,
%   \Vsmalldiagmat,
% }
% \begin{syntax}
%   \cs{psmalldiagmat} \oarg{kv opts} \marg{diagonal}
%   \cs{bsmalldiagmat} \oarg{kv opts} \marg{diagonal}
%   \cs{Bsmalldiagmat} \oarg{kv opts} \marg{diagonal}
%   \cs{vsmalldiagmat} \oarg{kv opts} \marg{diagonal}
%   \cs{Vsmalldiagmat} \oarg{kv opts} \marg{diagonal}
% \end{syntax}
% Like the \cs[no-index]{\meta{delim}diagmat} commands described above
% there are also shorthand commands for producing an inline math version
% of a diagonal matrix with pre-set delimiters.
% \end{function}
%
% \medbreak\noindent
% \textbf{Example:}
% \begin{quote}
%   \begin{verbatim}
% An inline math delimited diagonal matrix looks like this
% \(\psmalldiagmat{1,1}\).
%   \end{verbatim}
% An inline math delimited diagonal matrix looks like this
% \(\psmalldiagmat{1,1}\).
% \end{quote}
%
% \begin{function}{
%   \psmallantidiagmat,
%   \bsmallantidiagmat,
%   \Bsmallantidiagmat,
%   \vsmallantidiagmat,
%   \Vsmallantidiagmat,
% }
% \begin{syntax}
%   \cs{psmallantidiagmat} \oarg{kv opts} \marg{diagonal}
%   \cs{bsmallantidiagmat} \oarg{kv opts} \marg{diagonal}
%   \cs{Bsmallantidiagmat} \oarg{kv opts} \marg{diagonal}
%   \cs{vsmallantidiagmat} \oarg{kv opts} \marg{diagonal}
%   \cs{Vsmallantidiagmat} \oarg{kv opts} \marg{diagonal}
% \end{syntax}
% Like the \cs[no-index]{\meta{delim}antidiagmat} commands described above
% there are also shorthand commands for producing inline math versions
% of anti-diagonal matrices inside of delimiters.
% \end{function}
%
% \medbreak\noindent
% \textbf{Example:}
% \begin{quote}
%   \begin{verbatim}
% A delimited version of an inline math anti-diagonal matrix
% looks like this \(\psmallantidiagmat{1,1}\).
%   \end{verbatim}
% A delimited version of an inline math anti-diagonal matrix
% looks like this \(\psmallantidiagmat{1,1}\).
% \end{quote}
%
% \subsection{Identity Matrices}
% \label{sec:id-matrices}
%
% As identity matrices are always quadratic,
% one can further simplify the typesetting of identity matrices to a command,
% which constructs the identity matrix from the number of dimensions.
% The commands in this subsection perform this.
%
% \begin{function}[added = 2024-07-04]{
%   \idmat,
%   \smallidmat,
% }
%   \begin{syntax}
%     \cs{idmat} \oarg{kv~opts} \marg{dimension}
%     \cs{smallidmat} \oarg{kv~opts} \marg{dimension}
%   \end{syntax}
%   The \cs{idmat} command produces an identity matrix.
%   The optional argument <kv~opts> accepts any valid \emph{matrix} key-value
%   arguments,
%   which are described in section~\ref{sec:options-matrix}.
%   The mandatory argument <dimension> expects a \emph{positive} integer,
%   which specifies the dimensions of the identity matrix.
%
%   The \cs{smallidmat} command behaves like the \cs{idmat} command,
%   except it produces a matrix suitable for inline math mode.
%
%   \begin{texnote}
%     The maximum number of columns is determined by the \TeX{} counter
%     "MaxMatrixCols",
%     which has a default value of 10.
%     See \textcite{amsmath} for more information.
%   \end{texnote}
% \end{function}
%
%
% \begin{function}[added = 2024-07-04]{
%   \pidmat,
%   \bidmat,
%   \Bidmat,
%   \vidmat,
%   \Vidmat
% }
% \begin{syntax}
%   \cs{pidmat} \oarg{kv~opts} \marg{dimension}
%   \cs{bidmat} \oarg{kv~opts} \marg{dimension}
%   \cs{Bidmat} \oarg{kv~opts} \marg{dimension}
%   \cs{vidmat} \oarg{kv~opts} \marg{dimension}
%   \cs{Vidmat} \oarg{kv~opts} \marg{dimension}
% \end{syntax}
%   Like the \cs[no-index]{\meta{delim}diagmat} commands,
%   the \cs[no-index]{\meta{delim}idmat} commands produce an identity matrix
%   with a pre-set delimiter around the matrix,
%   i.e.\ they behave like \cs{idmat}|[delimiter = |<delim>|]|\marg{dimension}.
%   <delim> can be "p" for parenthesis,
%   "b" for brackets,
%   "B" for braces,
%   "v" for a single vertical line (\enquote{\(\vert\)}),
%   or "V" for a double vertical line (\enquote{\(\Vert\)}).
%
%   Furthermore these commands accept the same <kv~opts> as the \cs{idmat} command.
% \end{function}
%
% \begin{function}[added = 2024-07-04]{
%   \psmallidmat,
%   \bsmallidmat,
%   \Bsmallidmat,
%   \vsmallidmat,
%   \Vsmallidmat
% }
% \begin{syntax}
%   \cs{psmallidmat} \oarg{kv~opts} \marg{dimension}
%   \cs{bsmallidmat} \oarg{kv~opts} \marg{dimension}
%   \cs{Bsmallidmat} \oarg{kv~opts} \marg{dimension}
%   \cs{vsmallidmat} \oarg{kv~opts} \marg{dimension}
%   \cs{Vsmallidmat} \oarg{kv~opts} \marg{dimension}
% \end{syntax}
% These commands are provide inline math suitable versions of the
% \cs[no-index]{\meta{delim}idmat} commands described above.
% See their description for more information.
%
% The commands accept any of the key-value options described in section~\ref{sec:options-matrix}
% as optional argument <kv~opts>.
% \end{function}
%
% \paragraph{Examples of Use}
% \begin{enumerate}
%     \item Identity matrix without delimiters:
%
%           \begin{minipage}{.49\linewidth}
%             \begin{verbatim}
% \[
%   \idmat{3}
% \]
%             \end{verbatim}
%           \end{minipage}\hfill
%           \begin{minipage}{.49\linewidth}
%             \[\idmat{3}\]
%           \end{minipage}
%
%     \item Identity matrices with different delimiters:
%
%          \begin{verbatim}
% \[
%   \pidmat{3}\text{,}\quad \bidmat{3}\text{,}\quad
%   \Bidmat{3}\text{,}\quad \vidmat{3}\text{,}\quad
%   \Vidmat{3}
% \]
%          \end{verbatim}
%          \[
%            \pidmat{3}\text{,}\quad
%            \bidmat{3}\text{,}\quad
%            \Bidmat{3}\text{,}\quad
%            \vidmat{3}\text{,}\quad
%            \Vidmat{3}
%          \]
%
%   \item Inline math versions of identity matrices with delimiters:
%
%         \begin{verbatim}
% Inline math mode matrices look like this:
% \(\psmallidmat{2}\), \(\bsmallidmat{2}\), \(\Bsmallidmat{2}\),
% \(\vsmallidmat{2}\), \(\Vsmallidmat{2}\).
%         \end{verbatim}
%         Inline math mode matrices look like this:
%         \(\psmallidmat{2}\), \(\bsmallidmat{2}\), \(\Bsmallidmat{2}\),
%         \(\vsmallidmat{2}\), \(\Vsmallidmat{2}\).
%
%   \item Identity matrix using the "fill" option for the non-diagonal elements
%
%         \begin{minipage}{.49\linewidth}
%           \begin{verbatim}
% \[
%   \pidmat[fill=\star]{3}
% \]
%           \end{verbatim}
%         \end{minipage}\hfill
%         \begin{minipage}{.49\linewidth}
%           \[\pidmat[fill=\star]{3}\]
%         \end{minipage}
% \end{enumerate}
%
%
% \section{Shorthands for Absolute Value and Norm}
% \label{sec:shorthands}
%
% \DescribeOption{no-abs-shorthands}
% The commands in this section are only declared if the option "no-abs-shorthands"
% has not been given to the package as a load time option.
%
% \begin{function}{
%   \abs,
%   \norm
% }
%   \begin{syntax}
%     \cs{abs} \oarg{size cmd} \marg{content}
%     \cs{abs}* \marg{content}
%     \cs{norm} \oarg{size cmd} \marg{content}
%     \cs{norm}* \marg{content}
%   \end{syntax}
%   The commands \cs{abs} and \cs{norm}, produce \(\abs{\cdot}\) and
%   \(\norm{\cdot}\) respectively.
%   These commands are simply paired delimiters defined using \pkg{mathtools}'
%   \cs[module=mathtools,replace=false]{DeclarePairedDelimiter} command,
%   instruction on their usage can therefore be found in~\textcite{mathtools}.
% \end{function}
%
%
% \section{Vertically Centering Math Along the Math Axis}
% \label{sec:vcenter}
%
% Sometimes it is useful to explicitly center a math symbol along the math axis.
% A prominent example is the case of |\mathop{\boldsymbol\nabla}\nolimits|
% vs.\ |\mathop{\nabla}\nolimits|,
% as displayed below.
% \[
%   \mathop{\boldsymbol\nabla}\nolimits f(x)
%   \quad \mathop{\nabla}\nolimits f(x)
% \]
% Here the bold nabla is slightly higher up than the non-bold version.
%
% \begin{function}[added=2024-07-08]{\VCenterMath}
%   \begin{syntax}
%     \cs{VCenterMath} \marg{contents}
%   \end{syntax}
% \begin{danger}
%   The command \cs{VCenterMath} centers <contents> along the current math axis,
%   while adhering to the current math style.
%   This command is not in a \TeX{}-sense
%   \tn[no-index]{long}, i.e.\ it does not take \tn[no-index]{par} tokens.
%
%   This command is somewhat dangerous as it utilizes the \tn{vcenter} primitive,
%   which leaves math mode and requires to reenter it.
%   \emph{Use at your own responsibility.}
% \end{danger}
% \begin{texnote}
%   This command is a wrapper around \tn{vcenter}, \cs{hbox:n}
%   and \tn{mathpalette}, it reenters math mode inside the hbox.
% \end{texnote}
% \end{function}
%
% With \cs{VCenterMath} it is possible to fix the above example to:
% \begin{quote}
%   \begin{verbatim}
% \[
%   \mathop{\VCenterMath{\boldsymbol\nabla}}\nolimits f(x)
%   \quad \mathop{\nabla}\nolimits f(x)
% \]
%   \end{verbatim}
% \[
%   \mathop{\VCenterMath{\boldsymbol\nabla}}\nolimits f(x)
%   \quad \mathop{\nabla}\nolimits f(x)
% \]
% \end{quote}
%
%
% \part{Documentation for Class and Package Authors}
% \label{sec:package-auth-doc}
%
% \section{Setting of Options}
% \label{sec:authors-options}
%
% \begin{function}[updated = 2024-07-15]{
%   \moremath_setup:n
% }
%   \begin{syntax}
%     \cs{moremath_setup:n} \marg{kv opts}
%   \end{syntax}
%   This functions sets the key-value options \meta{kv~opts} in the "moremath" namespace.
% \end{function}
%
% \section{Delimited Operators}
% \label{sec:authors-delim-op}
%
% \begin{function}[updated = 2024-07-15]{
%   \moremath_delim_op_noscale:NNnnn,
%   \moremath_delim_op_autoscale:NNnnn
% }
%   \begin{syntax}
%     \cs{moremath_delim_op_noscale:NNnnn} \meta{math~op} \meta{delim}
%     \hspace*{2ex} \marg{sup~tl} \marg{sub~tl} \marg{tl}
%     \cs{moremath_delim_op_autoscale:NNnnn} \meta{math op} \meta{delim}
%     \hspace*{2ex} \marg{sup~tl} \marg{sub~tl} \marg{tl}
%   \end{syntax}
%   The two functions \cs{moremath_delim_op_noscale:NNnnn} and \cs{moremath_delim_op_autoscale:NNnnn}
%   provide a version of a delimited operator with no scaling or automatic scaling respectively.
%   The token list arguments \meta{sup~tl} and \meta{sub~tl} take the super- and subscript
%   of the operator respectively. Both of those token lists may be empty.
%
%   The first argument to the function is a control sequence \meta{math~op},
%   which should be an operator declared with
%   \cs[module=amsmath,replace=false]{DeclareMathOperator}.
%   The second argument \meta{delim} has to be a control sequence of a paired delimiter,
%   as defined by \pkg{mathtools}'s
%   \cs[module=mathtools,replace=false]{DeclarePairedDelimiter}.
%
%   The final argument \meta{tl} is a list of arbitrary token to put inside the delimiters.
% \end{function}
%
% \begin{function}[updated=2024-07-15]{
%   \moremath_delim_op_manuscale:NNnnnn,
%   \moremath_delim_op_manuscale:NNVnnn
% }
%   \begin{syntax}
%     \cs{moremath_delim_op_manuscale:NNnnnn} \meta{math op} \meta{delim} \marg{scale cmd}
%      \hspace*{2ex}  \marg{sup} \marg{sub} \marg{tl}
%   \end{syntax}
%   The function \cs{moremath_delim_op_manuscale:NNnnnn} typesets a math operator with
%   manual scaling of its delimiters.
%   Its arguments are identical to \cs{moremath_delim_op_noscale:NNnnn}
%   except for the argument \meta{scale~cmd} which has to be a token list containing
%   a \meta{size~cmd} that can be understood by \pkg{mathtools},
%   usually \cs{big}, \cs{Big}, \cs{bigg} and \cs{Bigg}.
% \end{function}
%
% \begin{function}[updated=2024-07-15]{
%   \moremath_new_delim_op_command:NNN,
%   \moremath_new_delim_op_command:cNN,
% }
% \begin{syntax}
%   \cs{moremath_new_delim_op_command:NNN} \meta{new~csname} \meta{operator} \meta{delim}
% \end{syntax}
% The function \cs{moremath_new_delim_op_command:NNN} defines a new document level command
% called \meta{new~csname}, utilizing the operator \meta{operator} and the delimiter \meta{delim}.
%
% \meta{operator} has to be a control sequence referring to an operator,
% as declared by \cs{DeclareMathOperator}.
%
% \meta{delim} has to be a paired delimiter as defined by \pkg{mathtools}'
% \cs{DeclarePairedDelimiter}.
% \end{function}
%
% \section{Vector Calculus Typesetting}
% \label{sec:authors-vec-calc}
%
% \subsection{Functions Producing Standalone Operators with Subscripts}
% \label{sec:authors-vec-calc-operators}
%
% \begin{function}[updated=2024-07-15]{
%   \moremath_gradient_operator:n,
%   \moremath_divergence_operator:n,
%   \moremath_curl_operator:n,
%   \moremath_laplace_operator:n,
%   \moremath_dalembert_operator:n
% }
%   \begin{syntax}
%     \cs{moremath_gradient_operator:n} \marg{subscript}
%     \cs{moremath_divergence_operator:n} \marg{subscript}
%     \cs{moremath_curl_operator:n} \marg{subscript}
%     \cs{moremath_laplace_operator:n} \marg{subscript}
%     \cs{moremath_dalembert_operator:n} \marg{subscript}
%   \end{syntax}
%   Each of these functions return a vector calculus operator,
%   with an (optional) subscript.
%   The argument \meta{subscript}, a token list, may be empty,
%   in which case no subscript is produced.
%
%   The actual operator produced depends on the current settings
%   of the key-value options described in section~\ref{sec:pack-opts}
%   inside the current group.
% \end{function}
%
% \subsection{Functions Producing Operators with Delimiters}
% \label{sec:authors-vec-calc-delim-ops}
%
% \begin{function}[updated=2024-07-15]{
%   \moremath_delim_nabla_op_noscale:NNnn,
%   \moremath_delim_nabla_op_autoscale:NNnn,
% }
%   \begin{syntax}
%     \cs{moremath_delim_nabla_op_noscale:NNnn} \meta{operator} \meta{delim}
%     \hspace*{2ex} \marg{sub} \marg{contents}
%
%     \cs{moremath_delim_nabla_op_autoscale:NNnn} \meta{operator} \meta{delim}
%     \hspace*{2ex} \marg{sub} \marg{contents}
%   \end{syntax}
%   The usage of these functions is similar to the functions
%   \cs{moremath_delim_op_noscale:NNnnn} and \cs{moremath_delim_op_autoscale:NNnnn}
%   except for the missing superscript.
%
%   The first argument \meta{operator} has to be a function which accepts one argument
%   (the subscript).
%   This is usually one of the \cs[no-index]{moremath_\meta{op}_operator:n} commands.
%
%   The argument \meta{sub}, which may be empty,
%   is a token list to pass as the subscript to
%   \cs[no-index]{moremath_\meta{op}_operator:n}.
%
%   The last argument \meta{contents} is a token list to put inside the delimiters.
%
% \end{function}
%
% \begin{function}[updated=2024-07-15]{
%   \moremath_delim_nabla_op_manuscale:NNnnn,
%   \moremath_delim_nabla_op_manuscale:NNVnn,
% }
%   \begin{syntax}
%     \cs{moremath_delim_nabla_op_manuscale:NNnnn} \meta{operator} \meta{delim}
%     \hspace*{2ex} \marg{scale~cmd} \marg{sub} \marg{contents}
%   \end{syntax}
%   The usage of function is similar to the above functions
%   except for the additional argument \meta{scale~cmd} which is a token list
%   with a \meta{size~cmd} that should be understood by \pkg{mathtools}
%   (e.g.\ \cs{big}, \cs{Big}, etc.).
% \end{function}
%
% \section{Formatting Matrices and Vectors}
% \label{sec:authors-matrix}
%
% \begin{function}[updated=2024-07-15]{
%   \moremath_column_vector:nn,
%   \moremath_row_vector:nn,
%   \moremath_column_smallvector:nn,
%   \moremath_row_smallvector:nn
% }
% \begin{syntax}
%   \cs{moremath_column_vector:nn} \marg{delim spec tl} \marg{clist}
%   \cs{moremath_row_vector:nn} \marg{delim spec tl} \marg{clist}
%   \cs{moremath_column_smallvector:nn} \marg{delim spec tl} \marg{clist}
%   \cs{moremath_row_smallvector:nn} \marg{delim spec tl} \marg{clist}
% \end{syntax}
% These functions produce a column or row vector.
% The "small"\meta{c~or~r}"vector" versions are intended for inline math mode.
%
% The first argument \meta{delim spec tl} should be the specification of
% a delimiter, as used in the naming of the \env{matrix*} environments
% by \pkg{mathtools}.
%
% The second argument \meta{clist}, is a comma-separated list of entries
% of the vector.
%
% Of this commands the "c" versions produce column vectors, the "r" versions
% produce row vectors.
%
% The alignment of the entries inside the vector depends on the current
% value of the key "align".
% \end{function}
%
% \begin{function}[
%   updated = 2024-07-04,
%   updated=2024-07-15,
% ]{
%   \moremath_diagonal_matrix:nn,
%   \moremath_diagonal_matrix:nV,
%   \moremath_diagonal_matrix:Vn,
%   \moremath_diagonal_matrix:VV,
%   \moremath_antidiagonal_matrix:nn,
%   \moremath_antidiagonal_matrix:nV,
%   \moremath_antidiagonal_matrix:Vn,
%   \moremath_antidiagonal_matrix:VV,
%   \moremath_diagonal_smallmatrix:nn,
%   \moremath_diagonal_smallmatrix:nV,
%   \moremath_diagonal_smallmatrix:Vn,
%   \moremath_diagonal_smallmatrix:VV,
%   \moremath_antidiagonal_smallmatrix:nn,
%   \moremath_antidiagonal_smallmatrix:nV,
%   \moremath_antidiagonal_smallmatrix:Vn,
%   \moremath_antidiagonal_smallmatrix:VV,
% }
% \begin{syntax}
%   \cs{moremath_diagonal_matrix:nn} \marg{delim tl} \marg{clist}
%   \cs{moremath_antidiagonal_matrix:nn} \marg{delim tl} \marg{clist}
%   \cs{moremath_diagonal_smallmatrix:nn} \marg{delim tl} \marg{clist}
%   \cs{moremath_antidiagonal_smallmatrix:nn} \marg{delim tl} \marg{clist}
% \end{syntax}
%  These functions produce (anti-)diagonal matrices.
%  The argument \meta{delim~tl} is a token list,
%  intended to specify the delimiter of the matrix or no delimiter if the
%  <tl> is empty no delimiters are produced.
%
%  The argument <clist> is a list of comma separated values which specify the
%  entries of the diagonal.
%  Other properties of the matrix to produce like alignment ("align")
%  and the tokens to insert for non-diagonal entries ("fill")
%  depend on the current setting of the keys described in section~\ref{sec:options-matrix}.
%
%  The "smallmatrix" versions are intended for use in inline math mode.
% \end{function}
%
% \begin{function}[
%   added = 2024-07-04,
%   updated=2024-07-15,
% ]{
%   \moremath_id_matrix:n,
%   \moremath_id_matrix:V,
%   \moremath_id_smallmatrix:n,
%   \moremath_id_smallmatrix:V,
% }
%   \begin{syntax}
%     \cs{moremath_id_matrix:n} \marg{dimensions}
%     \cs{moremath_id_smallmatrix:n} \marg{dimensions}
%   \end{syntax}
%   These functions produce a identity matrix with <dimensions> rows and columns.
%   <dimensions> is expected to be a \emph{positive} integer.
%   The "smallmatrix" version utilizes \pkg{mathtools}' \env{smallmatrix*} environment,
%   which makes it suitable for inline math mode.
%
%   The delimiter around the matrix is determined by the current value of the key
%   "matrix / delimiter" inside the current group.
%
%   As these functions utilize \cs{moremath_diagonal_matrix:VV} or
%   \cs{moremath_diagonal_smallmatrix:VV} and make use of temporary variables
%   it is advised to put them into their own group,
%   i.e.\ surround them by \cs{group_begin:} and \cs{group_end:}.
% \end{function}
%
% \section{Vertically Centering Math Along the Math Axis}
% \label{sec:autors-vcenter}
%
% As already said in section~\ref{sec:vcenter} sometimes it is needed to
% explicitly center some math code.
%
% \begin{function}[
%   added=2024-07-08,
%   updated=2024-07-15,
% ]{
%   \moremath_vcenter:n,
% }
%   \begin{syntax}
%     \cs{moremath_vcenter:n} \marg{contents}
%   \end{syntax}
%   The function \cs{moremath_vcenter:n} places <contents> inside a "hbox",
%   which is centered along the math axis by \tn{vcenter}.
%   The contents inside the "hbox" are set in math mode.
%   The function also applies the \enquote{outer} math style
%   to the contents of the "hbox".
%
%   This function is not \tn[no-index]{long},
%   as it sets its contents inside a "hbox".
%
%   \begin{texnote}
%     This function is a wrapper around \tn{vcenter}, \cs{hbox:n},
%     and \tn{mathpalette}.
%     Inside the "hbox" math mode is reentered.
%   \end{texnote}
% \end{function}
%
% \end{documentation}
%
%
%
%
%
%
%
%
%
%
% \begin{implementation}
% \part{Implementation}
% \label{pt:implementation}
% Start the \pkg{DocStrip} guards.
%    \begin{macrocode}
%<*package>
%    \end{macrocode}
%
% Set the internal prefix for this package so that \pkg{DocStrip} knows what to
% do.
%    \begin{macrocode}
%<@@=moremath>
%    \end{macrocode}
%
% \section{Initial Setup}
%
% First set the required version of \LaTeX{},
% we need at least
%    \begin{macrocode}
\NeedsTeXFormat{LaTeX2e}[2022-11-01]
%    \end{macrocode}
% for key-value option handling, \pkg{xparse}-like document commands
% (especially for using the "=" option specification)
% and hooks.
%
% Then identify this package as \pkg{moremath}.
%    \begin{macrocode}
\ProvidesExplPackage{moremath}
  {2024-07-15}{v0.4.0}{More Math Macros}
%    \end{macrocode}
%
% ^^X Debugging switch
% \iffalse
% \debug_on:n {all}
% \fi
% \section{Key-value interfaces}
% \label{sec:key-val}
% To make use of key-value interfaces we need to define a few keys.
%
% \subsection{Package load time options}
% \label{sec:load-time-options}
%
% To allow for the conditional definition of macros at load time
% we define a few keys.
%
% But before doing so we define a few messages to write the package options to
% the log file.
% One message to issue if the \texttt{bm} option has been given:
%    \begin{macrocode}
\msg_new:nnn { moremath } {load / bm}
{
  Option~'bm'~given.\\
  Loading~the~bm~package~\msg_line_context:.
}
%    \end{macrocode}
% And a more generic message to issue for the other options,
% which all disable certain parts of the library.
%    \begin{macrocode}
\msg_new:nnn {moremath} { load / disabling }
{
  Option~'#1'~given.\\
  Disabling~#2~\msg_line_context:.
}
%    \end{macrocode}
%
% \begin{variable}{
%   \l_@@_predef_vector_op_bool,
%   \l_@@_predef_operators_bool,
%   \l_@@_predef_crvector_bool,
%   \l_@@_predef_matrix_bool,
%   \l_@@_predef_abs_bool,
% }
%   To store the values of several switch type key-value options
%   we declare several boolean variables.
%    \begin{macrocode}
\bool_new:N \l_@@_predef_vector_op_bool
\bool_new:N \l_@@_predef_operators_bool
\bool_new:N \l_@@_predef_crvector_bool
\bool_new:N \l_@@_predef_matrix_bool
\bool_new:N \l_@@_predef_abs_bool
%    \end{macrocode}
%
% The variables \cs{l_@@_predef_vector_op_bool}, \cs{l_@@_predef_operators_bool}
% \cs{l_@@_predef_abs_bool}, \cs{l_@@_predef_matrix_bool}
% and \cs{l_@@_predef_crvector_bool}
% control if the predefined macros
% for vector calculus, delimited operators, the matrix shorthands,
% the row and column vectors,
% and the shorthands for absolute value
% and norm shall be defined.
%
% \changes{v0.2.0}{\GetVersionDate{v0.2.0}}{
%   Explicitly declared boolean variables used to store key-value options.
% }
% \end{variable}
%
% \begin{optionenv}{
%   bm,
%   no-vector,
%   no-operators,
%   no-abs-shorthands,
%   no-matrix,
%   no-crvector,
%   nopredef,
% }
% Now we define package load time keys:
%    \begin{macrocode}
\keys_define:nn { moremath / load }
{
%    \end{macrocode}
% We provide an option to conditionally load the \pkg{bm}-package~\autocite{bm}
% to provide better looking bold symbols.
%    \begin{macrocode}
  bm .code:n = {
    \msg_info:nn {moremath} {load / bm}
    \RequirePackage{bm}
  },
  bm .value_forbidden:n = true,
  bm. usage:n = load,
%    \end{macrocode}
% Then we define options for en-/disabling predefined macros of this package
% to avoid name clashes.
%    \begin{macrocode}
  no-vector .bool_set_inverse:N = \l_@@_predef_vector_op_bool,
  no-vector .default:n = true,
  no-vector .initial:n = false,
  no-vector .usage:n = load,
  no-operators .bool_set_inverse:N = \l_@@_predef_operators_bool,
  no-operators .default:n = true,
  no-operators .initial:n = false,
  no-operators .usage:n = load,
  no-abs-shorthands .bool_set_inverse:N = \l_@@_predef_abs_bool,
  no-abs-shorthands .default:n = true,
  no-abs-shorthands .initial:n = false,
  no-abs-shorthands .usage:n = load,
  no-matrix .bool_set_inverse:N = \l_@@_predef_matrix_bool,
  no-matrix .initial:n = false,
  no-matrix .default:n = true,
  no-matrix .usage:n =load,
  no-crvector .bool_set_inverse:N = \l_@@_predef_crvector_bool,
  no-crvector .default:n = true,
  no-crvector .initial:n = false,
  no-crvector .usage:n = load,
  nopredef .multichoice:,
  nopredef / operators .meta:nn = { moremath / load }
  {
    no-operators = true
  },
  nopredef / vector .meta:nn = { moremath / load }
  {
    no-vector = true
  },
  nopredef / abs .meta:nn = { moremath / load }
  {
    no-abs-shorthands = true,
  },
  noperdef / matrix .meta:nn = { moremath / load }
  {
    no-matrix = true,
  },
  nopredef / crvector .meta:nn = {moremath / load}
  {
    no-crvector = true,
  },
  nopredef / all .meta:nn = { moremath / load }
  {
    no-operators = true,
    no-vector = true,
    no-abs-shorthands = true
  },
  nopredef .usage:n = load,
%    \end{macrocode}
% Unknown package options get passed to \pkg{mathtools}.
%    \begin{macrocode}
  unknown .code:n = {\PassOptionsToPackage{\CurrentOption}{mathtools}},
  unknown .usage:n = load,
}
%    \end{macrocode}
% \changes{v0.1.0}{\GetVersionDate{v0.1.0}}{
%   Added package load-time options.
% }
% \end{optionenv}
%
%
% \subsection{Keys controlling appearance}
% \label{sec:keys-appearance}
%
% \begin{variable}{
%   \l_@@_nabla_tl,
%   \l_@@_nabla_arrow_bool,
%   \l_@@_nabla_arrow_bold_bool,
%   \l_@@_grad_op_tl,
%   \l_@@_laplacian_symb_tl,
%   \l_@@_laplacian_delta_bool,
%   \l_@@_laplacian_arrow_bool,
%   \l_@@_laplacian_tl,
%   \l_@@_dalembert_symb_tl,
% }
%   We declare several variables to store the values of the keys affecting
%   appearance.
%    \begin{macrocode}
\tl_new:N \l_@@_nabla_tl
\bool_new:N \l_@@_nabla_arrow_bool
\bool_new:N \l_@@_nabla_arrow_bold_bool
\tl_new:N \l_@@_grad_op_tl
%    \end{macrocode}
%
% The symbol to use as \enquote{nabla} is stored in \cs{l_@@_nabla_tl}.
% The variables \cs{l_@@_nabla_arrow_bool} and \cs{l_@@_nabla_bold_bool}
% determine if the nabla-symbol shall have an arrow over itself and/or be bold
% respectively.
%
% The variable \cs{l_@@_grad_op_tl} contains a user provided token list to
% overwrite the built-in gradient operator of the package.
%
%    \begin{macrocode}
\tl_new:N \l_@@_laplacian_symb_tl
\bool_new:N \l_@@_laplacian_delta_bool
\bool_new:N \l_@@_laplacian_arrow_bool
\tl_new:N \l_@@_laplacian_tl
%    \end{macrocode}
% The token list variable \cs{l_@@_laplacian_symb_tl} stores the tokens to be
% used for the Laplace operator.
% The boolean variable \cs{l_@@_laplacian_delta_bool}
% determines if a delta should be used instead
% of \(\laplacian\) for the laplacian symbol.
% If the boolean variable \cs{l_@@_laplacian_arrow_bool} a small arrow will be
% placed over the Laplace operator symbol.
% If the user wants to overwrite the symbol used for the Laplacian,
% the user provided list of tokens is stored in the variable
% \cs{l_@@_laplacian_tl}.
%
% Finally we define a variable to hold the symbol to use for the
% \foreignlanguage{french}{d'Alembert} operator.
%    \begin{macrocode}
\tl_new:N \l_@@_dalembert_symb_tl
%    \end{macrocode}
% \changes{v0.2.0}{\GetVersionDate{v0.2.0}}{%
% Explicitly declared variables before using them to set keys.
% }
% \changes{v0.2.0}{\GetVersionDate{v0.2.0}}{%
%   Added: New variable \cs{l_@@_dalembert_symb_tl}
% }
% \end{variable}
%
% \begin{variable}{
%   \l_@@_vcenter_bool,
% }
%   Additionally we declare a variable to decide if certain math symbols
%   shall be centered explicitly along the math axis.
%    \begin{macrocode}
\bool_new:N \l_@@_vcenter_bool
%    \end{macrocode}
%   If \cs{l_@@_vcenter_bool} is true, the symbols of certain math operators,
%   should be centered explicitly.
% \end{variable}
%
% \begin{optionenv}{
%   nabla,
%   arrownabla,
%   boldnabla,
%   grad-op,
%   laplacian-symb,
%   delta-laplace,
%   arrowlaplace,
%   laplacian,
%   dalembert-symb,
% }
% First define keys for the vector calculus macros.
%    \begin{macrocode}
\keys_define:nn { moremath }
{
%    \end{macrocode}
% First we define keys for use with the gradient and gradient based operators.
%    \begin{macrocode}
  % Symbol to use for the nabla
   nabla .tl_set:N = \l_@@_nabla_tl,
   nabla .initial:n = {\nabla},
   nabla .value_required:n = true,
   % shall the nabla have an arrow over it
   arrownabla .bool_set:N = \l_@@_nabla_arrow_bool,
   arrownabla .default:n = {true},
   arrownabla .initial:n = {false},
   % shall the nabla be bold
   boldnabla .bool_set:N = \l_@@_nabla_bold_bool,
   boldnabla .default:n = {true},
   boldnabla .initial:n = {false},
%    \end{macrocode}
% We also provide an override for the gradient operator
%    \begin{macrocode}
   % Symbol to use for the gradient operator
   grad-op .tl_set:N = \l_@@_grad_op_tl,
   grad-op .value_required:n = true,
%    \end{macrocode}
%
% Then we define keys for the laplacian.
%    \begin{macrocode}
  % Symbol to use for the laplacian
  laplacian-symb .tl_set:N = \l_@@_laplacian_symb_tl,
  laplacian-symb .initial:n = {\l_@@_nabla_tl},
  % shall the Laplace operator be a delta
  delta-laplace .bool_set:N = \l_@@_laplacian_delta_bool,
  delta-laplace .initial:n = {false},
  % shall the laplace operator have an arrow over itself
  arrowlaplace .bool_set:N = \l_@@_laplacian_arrow_bool,
  arrowlaplace .default:n = {true},
  arrowlaplace .initial:n = {false},
  % overwrite the laplace operator
  laplacian .tl_set:N = \l_@@_laplacian_tl,
  laplacian .value_required:n = true,
%    \end{macrocode}
% And keys for the \foreignlanguage{french}{d'Alembert} operator.
%    \begin{macrocode}
  dalembert-symb .tl_set:N = \l_@@_dalembert_symb_tl,
  dalembert-symb .initial:n = {\square},
%    \end{macrocode}
% \begin{optionenv}{
%   vcenter,
% }
%   The "vcenter" option will control the manual centering of certain math
%   operators.
%    \begin{macrocode}
  vcenter .bool_set:N = \l_@@_vcenter_bool,
  vcenter .initial:n = true,
  vcenter .value_required:n = true,
}% \keys_define:nn
%    \end{macrocode}
%   \changes{v0.3.0}{\GetVersionDate{v0.3.0}}{
%     Added new option.
%   }
% \end{optionenv} ^^X vcenter
% \changes{v0.1.0}{\GetVersionDate{v0.1.0}}{%
%   Added options controlling appearance.
%  }
% \changes{v0.2.0}{\GetVersionDate{v0.2.0}}{%
%   Added: New option "dalembert-symb".
% }
% \end{optionenv}
%
% \begin{optionenv}{
%   delimiter,
%   fill,
%   align
% }
% Then we define some keys for the matrix based environments:
%    \begin{macrocode}
\keys_define:nn { moremath / matrix }
{
%    \end{macrocode}
% \begin{variable}{
% \l_@@_matrix_delim_tl,
% \l_@@_matrix_fill_tl,
% \l_@@_matrix_align_tl,
% }
% Every one of the following keys stores its value inside a token list variable.
%    \begin{macrocode}
  delimiter .tl_set:N = \l_@@_matrix_delim_tl,
  delimiter .initial:n = {},
  fill .tl_set:N = \l_@@_matrix_fill_tl,
  fill .initial:n = {},
  align .tl_set:N = \l_@@_matrix_align_tl,
  align .initial:n = {c},
  align .value_required:n = true,
}
%    \end{macrocode}
% The keys \verb|delimiter| and \verb|fill| set the variables
% \cs{l_@@_matrix_delim_tl} and \cs{l_@@_matrix_fill_tl} respectively.
% \cs{l_@@_matrix_delim_tl} determines the delimiter in use to surround
% matrices and \cs{l_@@_matrix_fill_tl} determines the fill values of the
% \cs{diag}, \cs{smalldiag}, \cs[no-index]{Xdiag} and \cs[no-index]{Xsmalldiag}
% commands, which are used for non-diagonal matrix entries.
% The variable \cs{l_@@_matrix_align_tl} contains the alignment specifier
% for use with the \env{matrix*} environment.
%
% \changes{v0.1.0}{\GetVersionDate{v0.1.0}}{%
%   Added internal variables.
% }
% \end{variable}
% \changes{v0.1.0}{\GetVersionDate{v0.1.0}}{%
%   Added options for the matrix-based environments of \pkg{mathtools}.
% }
% \end{optionenv}
%
% \subsection{Functions for Setting Options}
% \label{sec:option-setters}
%
%
% Now we define a function for setting the options within the document
% \begin{macro}{\moremath_setup:n}
%    \begin{macrocode}
\cs_new_protected:Nn \moremath_setup:n
{
  \keys_set:nn {moremath} {#1}
}
%    \end{macrocode}
%   \changes{v0.1.0}{\GetVersionDate{v0.1.0}}{%
%     Added function for setting options.
%   }
% \end{macro}
%
% Additionally we provide the user with a version of this command.
% \begin{macro}{\moremathsetup}
%    \begin{macrocode}
\NewDocumentCommand \moremathsetup {m}
{
  \moremath_setup:n {#1}
}
%    \end{macrocode}
%   \changes{v0.1.0}{\GetVersionDate{v0.1.0}}{%
%     Added document command for setting package options.
%   }
% \end{macro}
% We also need a function for setting the package load time options.
% This function should set all given values for all key families
% and pass unknown options to mathtools.
% \begin{macro}{\@@_load_time_setup:}
%    \begin{macrocode}
\cs_new_protected:Nn \@@_load_time_setup:
{
  \ProcessKeyOptions[ moremath / load ]
}
%    \end{macrocode}
% \end{macro}
%
% ^^X SUBSECTION: Package Initialization
% \section{Package Initialization}
% \label{sec:init}
%
% We now \enquote{initialize} the package by processing the package options,
% all unknown options are passed to \pkg{mathtools}~\autocite{mathtools},
% which is loaded afterwards.
% Because certain \pkg{mathtools}-features are needed by this package,
% we need to require a version of at least 2004/06/05.
% As explained in section~\ref{sec:load-time-options} this may also load \pkg{bm}
% if the \verb|bm| package option has been given.
%    \begin{macrocode}
\@@_load_time_setup:
%    \end{macrocode}
% If the "no-vector" option has not been given during load time,
% we also need the \pkg{amssymb} package~\autocite{amsfonts} for the
% \cs[module=amssymb,replace=false]{square} command.
% We first define a message to inform the user about this.
%    \begin{macrocode}
\msg_new:nnn { moremath } { load / loading-amssymb }
{
  Vector~calculus~commands~enabled.\\
  Loading~amssymb~package~\msg_line_context:.
}
%    \end{macrocode}
% Then we conditionally load the package.
%    \begin{macrocode}
\bool_if:NT \l_@@_predef_vector_op_bool
{
  \msg_info:nn { moremath } { load / loading-amssymb }
  \RequirePackage{amssymb}
}
%    \end{macrocode}
% \changes{v0.2.0}{\GetVersionDate{v0.2.0}}{
%   New: Load \pkg{amssymb} if the "no-vector" option has \emph{not} been given.
% }
%
% Finally we load our most important dependency \pkg{mathtools}.
%    \begin{macrocode}
\RequirePackage{mathtools}[2004/06/05]
%    \end{macrocode}
%
% \section{Abstractions for \TeX{}-primitives and \LaTeX2e{}-commands}
% \label{sec:impl-abstractions}
%
% \begin{NOTE}{MI}
% TODO: Provide and use abstractions for the following commands and
% primitives:
%
% - [ ] \mathpalette
% - [ ] \vcenter
% - [ ] \mathsurround
% - [ ] \begin{matrix*}, \end{matrix*}
% - [ ] \mathop{ }
% \end{NOTE}
%
% As this package makes use of certain \TeX{}-primitives and \LaTeX2e{}-commands,
% which are either part of the \LaTeX{}-kernel or of some package.
% It is sensible to provide a (package internal) abstraction for those commands,
% so that they can be switched out once a \LaTeX3{}-command has been defined for them.
%
% \subsection{\TeX{}-primitives}
% \label{sec:impl-tex-abstractions}
%
% \begin{macro}[added=0000-00-00]{
%   \@@_tex_vcenter:n,
% }
%   \cs{@@_tex_vcenter:n} is an abstraction for \cs{tex_vcenter:D}
% i.e.\ \tn{vcenter}.
% Its single argument is passed to \cs{tex_vcenter:D}.
%    \begin{macrocode}
\cs_new_protected:Nn \@@_tex_vcenter:n
{
  \tex_vcenter:D {#1}
}
%    \end{macrocode}
% \end{macro}
%
% \begin{macro}[added=0000-00-00]{
%   \@@_tex_mathpalette:Nn,
% }
%   The function \cs{@@_tex_mathpalette:Nn} is an abstraction for the \tn{mathpalette},
%   primitive.
%   The function takes to arguments
%   \begin{arguments}
%     \item The \meta{csname} of the macro to pass to \tn{mathpalette}
%
%     This macro \textbf{must} accept \textbf{two} arguments:
%     The first is a single token containing the current math-style,
%     the second is the contents to \enquote{typeset}.
%     See \url{https://tex.stackexchange.com/questions/34393/the-mysteries-of-mathpalette/34412#34412}
%     for further information.
%
%     \item The \meta{tokens} to forward to the macro passed as first argument.
%   \end{arguments}
%    \begin{macrocode}
\cs_new_protected:Nn \@@_tex_mathpalette:Nn
{
  \mathpalette #1 {#2}
}
%    \end{macrocode}
%   \changes{vUNRELEASED}{\GetVersionDate{vUNRELEASED}}{
%   Added new function.
% }
% \end{macro}
%
% \begin{macro}[added=0000-00-00]{
%   \@@_tex_mathsurround:n
% }
%   The function \cs{@@_tex_mathsurround:n} provides (hopefully) a correct abstraction
%   for the \tn{mathsurround} primitive.
%   It takes one argument
%   \begin{arguments}
%     \item A \meta{length~expr} for the whitespace surrounding a math switch.
%   \end{arguments}
%    \begin{macrocode}
\cs_new_protected_nopar:Nn \@@_tex_mathsurround:n
{
  \tex_mathsurround:D = #1
}
%    \end{macrocode}
%   \changes{vUNRELEASED}{\GetVersionDate{vUNRELEASED}}{
%   Added new function.
% }
% \end{macro}
%
% \begin{macro}[added=0000-00-00]{
%   \@@_tex_mathop:n
% }
% The function \cs{@@_tex_mathop:n} provides an abstraction for the \tn{mathop}
% \TeX{}-primitive.
% It accepts a single argument:
% \begin{arguments}
%   \item \meta{Tokens} to treat as a math operator.
% \end{arguments}
%    \begin{macrocode}
\cs_new_protected_nopar:Nn \@@_tex_mathop:n
{
  \tex_mathop:D {#1}
}
%    \end{macrocode}
%   \changes{vUNRELEASED}{\GetVersionDate{vUNRELEASED}}{
%   Added new function.
% }
% \end{macro}
%
%
% \section{Centering Math Symbols Along the Math-Axis}
% \label{sec:impl-vcenter}
%
% Certain math constructs such cause \TeX{}
% to not center the operator along the math axis,
% like case of |\mathop{\nabla}\nolimits| vs.\
% |\mathop{\boldsymbol\nabla}\nolimits|,
% as can be seen below.
% \[
%   \mathop{\nabla}\nolimits f(x)
%   \quad \text{vs.} \quad
%   \mathop{\boldsymbol\nabla}\nolimits f(x)
% \]
% As can be seen the bold nabla symbol is slightly higher up than the non-bold
% version.
% Because of this it is sometimes useful to manually center some math symbols,
% along the math axis.\footnote{%
%   The code in this section is inspired by
%   \url{http://www.tug.org/TUGboat/Articles/tb22-4/tb72perlS.pdf}
% }
%
% \begin{macro}[added = 2024-07-08]{
%   \moremath_vcenter:n
% }
%   The function \cs{moremath_vcenter:n} is a wrapper around the \tn{vcenter}
%   \TeX{} primitive.
%   It takes a single argument.
%   \begin{arguments}
%     \item A \meta{tl} containing math mode material to center along the math axis.
%
%          This argument is typeset in math mode.
%   \end{arguments}
%   The function uses the \tn{mathpalette} primitive to switch to the right
%   math style.
%   The function is not in a \TeX{}-sense long,
%   i.e.\ it does not take \tn{par} tokens.
%
%   As this function might be useful not only for internal use by \pkg{moremath},
%   we declare it as a public function here.
%    \begin{macrocode}
\cs_new_protected_nopar:Nn \moremath_vcenter:n
{
  \mathpalette \@@_vcenter:Nn {#1}
}
%    \end{macrocode}
%   \changes{v0.3.0}{\GetVersionDate{v0.3.0}}{
%     Added new function.
%   }
% \end{macro}
%
% \begin{macro}[added=2024-07-08]{
%   \@@_vcenter:Nn,
% }
%   As \tn{mathpalette} needs as its first argument a macro which takes two
%   arguments
%   (the first is a math style switch and the second the contents).\footnote{%
%     See \TeX{}SE answer \url{https://tex.stackexchange.com/a/34412}
%   }
%   We define an internal helper function for \cs{moremath_vcenter:n} for
%   \cs{mathpalette} to call.
%
%   The function \cs{@@_vcenter:Nn} takes two arguments:
%   \begin{arguments}
%     \item The math style macro, which is passed to this function by
%           \tn{mathpalette}.
%
%     \item The \meta{tl} to typeset inside the \tn{vcenter}.
%   \end{arguments}
%
%   Because of the properties of \tn{vcenter}, it switches to vertical mode,
%   we need to put the contents to typeset inside a horizontal box.
%   Because of this we also need to reenter math mode, and because of this
%   we need to remove the spacing inserted by entering math mode
%   by setting \tn{mathsurround} to zero.
%    \begin{macrocode}
\cs_new_protected_nopar:Nn \@@_vcenter:Nn
{
  \vcenter
  {
    \hbox:n
    {
      $
      \mathsurround=0pt
      #1 {#2}
      $
    }
  }
}
%    \end{macrocode}
% \end{macro}
%
%
% \subsection*{Document Level Command}
%
% Although unlikely there might arise the need for a document author to center
% some math along the math axis.
% For this purpose we are going to define a new document level command.
%
% But first we are going to declare some messages to use by the command.
% The first message informs the user that the command is not available,
% because its \meta{csname} is already taken.
%    \begin{macrocode}
\msg_new:nnn { moremath } { csname-already-defined }
{
  Control~sequence~' #1 '~is~already~ defined.\\
  Skipping~definition~\msg_line_context:
}
%    \end{macrocode}
% The second message should be issued if a command that only works in math mode
% was given outside of it.
%    \begin{macrocode}
\msg_new:nnnn { moremath } { vcenter / only-in-math-mode }
{
  Command~#1~used~outside~math~mode~\msg_line_context:.
}
{
  The~command~#1~may~only~be~used~inside~math~mode.
}
%    \end{macrocode}
% \begin{macro}{
%   \VCenterMath,
% }
% Now to the document level command for centering math along the math axis.
%    \begin{macrocode}
\cs_if_free:NTF \VCenterMath
{
  \NewDocumentCommand \VCenterMath { m }
  {
    \mode_if_math:TF
    {
      \moremath_vcenter:n {#1}
    }{% \mode_if_math:TF FALSE BRANCH
      \msg_error:nnn { moremath } { vcenter / only-in-math-mode } {\VCenterMath}
    }
  }
}{% \cs_if_free:NTF \VCenterMath FALSE BRANCH
  \msg_warning:nnn { moremath } { csname-already-defined } {\VCenterMath}
}
%    \end{macrocode}
%   \changes{v0.3.0}{\GetVersionDate{v0.3.0}}{
%     Added new document level command.
%   }
% \end{macro}
%
%
% ^^X SECTION: Paired Delimiters
% \section{Declaring Paired Delimiters for Internal Use}
% \label{sec:impl-paired-delim}
%
% \begin{macro}{
%   \@@_inparent:w,
%   \@@_inbrack:w,
%   \@@_inbrace:w,
%   \@@_in_vert:w,
%   \@@_in_Vert:w
% }
% As we are going to use \pkg{mathtools}' \emph{paired delimiters}
% at several places throughout this package,
% we define \emph{paired delimiters} for internal use, as functions with weird
% syntax.
%    \begin{macrocode}
\DeclarePairedDelimiter{\@@_inparent:w}{\lparen}{\rparen}
\DeclarePairedDelimiter{\@@_inbrack:w}{\lbrack}{\rbrack}
\DeclarePairedDelimiter{\@@_inbrace:w}{\lbrace}{\rbrace}
\DeclarePairedDelimiter{\@@_in_vert:w}{\lvert}{\rvert}
\DeclarePairedDelimiter{\@@_in_Vert:w}{\lVert}{\rVert}
%    \end{macrocode}
% \end{macro}
%
% ^^X SECTION: Delimited Operators
% \section{Delimited Operators}
% \label{sec:impl-delim-op}
%
% We need three different functions for providing the delimited operators.
% But as we share a lot of code between those, we define an additional
% helper function beforehand.
% \begin{macro}{
%   \@@_operator:Nnn
% }
% The function \cs{@@_operator:Nnn} takes care of the operator part
% of the new delimiter.
% It allows the operator to have super- and subscripts.
% It takes three arugments.
% \begin{arguments}
%   \item The \meta{csname} of the operator to use.
%   \item A \meta{token list}, which is used as the superscript operator.
%
%         This argument may be empty
%   \item A \meta{tl}, which is used as the subscript operator.
%
%         The \meta{tl} may be empty.
% \end{arguments}
%    \begin{macrocode}
\cs_new_protected:Nn \@@_operator:Nnn
{
  #1
  % add superscript if present
  \tl_if_empty:nF {#2} {^{#2}}
  % add subscript if present
  \tl_if_empty:nF {#3} { \c_math_subscript_token {#3} }
}
%    \end{macrocode}
% \end{macro}
%
% We now define three version of the delimited operators.
% \begin{macro}{
%   \moremath_delim_op_noscale:NNnnn,
%   \moremath_delim_op_autoscale:NNnnn,
% }
%   \cs{moremath_delim_op_noscale:NNnnn} is provides a delimited operator without
%   any scaling of the delimiters
%   and \cs{moremath_delim_op_autoscale:NNnnn} provides a version with
%   automatic scaling of the delimiters.
%   Both of them take five arguments:
%   \begin{arguments}
%     \item The \meta{csname} of the operator to use.
%
%           Any operator declared with \pkg{amsmath}'s \cs{operatorname}
%           and/or\linebreak[3]
%           \cs[module=amsmath,replace=false]{DeclareMathOperator}
%           is valid for this.
%
%     \item The \meta{csname} of a paired delimiter declared by
%     \pkg{mathtools}'~\autocite{mathtools}
%     \cs[module=mathtools,replace=false]{DeclarePairedDelimiter}.
%
%     \item A \meta{tl} to use as the superscript of the operator.
%
%     \item A \meta{tl} to use as the subscript of the operator.
%
%     \item A \meta{tl} to insert inside the delimiters.
%   \end{arguments}
%    \begin{macrocode}
\cs_new_protected_nopar:Nn \moremath_delim_op_noscale:NNnnn
{
  \@@_operator:Nnn #1 {#3} {#4}
  % #2 is the paired delimiter
  #2 {#5}
}

\cs_new_protected_nopar:Nn \moremath_delim_op_autoscale:NNnnn
{
  \@@_operator:Nnn #1 {#3} {#4}
  % #2 is the paired delimiter
  #2 * {#5}
}
%    \end{macrocode}
%   \changes{v0.1.0}{\GetVersionDate{v0.1.0}}{%
%     Added "noscale" and "autoscale" versions.
%   }
% \end{macro}
% \begin{macro}{
%   \moremath_delim_op_manuscale:NNnnnn,
%   \moremath_delim_op_manuscale:NNVnnn
% }
%   \cs{moremath_delim_op_manuscale:NNnnnn} provides a delimited operator with
%   manual scaling.
%   This version takes six arguments:
%   \begin{arguments}
%     \item The \meta{csname} of the operator to use.
%
%     \item The \meta{csname} of the paired delimiter
%     declared by \pkg{mathtools}'~\autocite{mathtools}\linebreak[3]
%     \cs[module=mathtools,replace=false]{DeclarePairedDelimiter}.
%
%     \item A \meta{tl} containing the scaling macro i.e.\ \cs{big}, \cs{Big},
%       \cs{Bigg},\dots
%
%     \item A \meta{tl} containing the superscript of the operator.
%
%      The \meta{tl} may be empty
%
%     \item A \meta{tl} containing the subscript of the operator
%
%       The \meta{tl} may be empty.
%
%     \item A \meta{tl} to insert inside the delimiters
%   \end{arguments}
%    \begin{macrocode}
\cs_new_protected_nopar:Nn \moremath_delim_op_manuscale:NNnnnn
{
  \@@_operator:Nnn #1 {#4} {#5}
  % #2 is the paired delimiter
  #2 [ #3 ] {#6}
}
%    \end{macrocode}
% We also provide a variant for the scaling macro.
%    \begin{macrocode}
\cs_generate_variant:Nn \moremath_delim_op_manuscale:NNnnnn {NNVnnn}
%    \end{macrocode}
%   \changes{v0.1.0}{\GetVersionDate{v0.1.0}}{%
%     Function added.
%    }
% \end{macro}
%
% For the creation of document level commands we also create a function,
% so that the user is also able to declare new delimited operators.
% But before we do so we create a message to inform the user if a
% \meta{csname} was already taken.
%    \begin{macrocode}
\msg_new:nnn { moremath } { delimop / already-defined-skip }
{
  Control~sequence~'#1'~is~already~defined.\\
  Skipping~definition~of~delimited~operator~'#1'~\msg_line_context:.
}
%    \end{macrocode}
%
% We also create a message to inform the user about conflicting options
% given to the command.
%    \begin{macrocode}
\msg_new:nnn { moremath } { delimop / auto-manu-scale-conflict }
{
  Both~star~and~scale~cmd~given~to~'#1'.\\
  Automatic~scaling~will~be~preferred,~the~size~command~will~be~
  ignored~\msg_line_context:.
}
%    \end{macrocode}
% \begin{macro}{
%   \moremath_new_delim_op_command:NNN,
%   \moremath_new_delim_op_command:cNN,
%}
% \cs{moremath_new_delim_op_command:NNN} takes three arguments:
% \begin{arguments}
%   \item The \meta{csname} to be defined.
%
%   \item The \meta{csname} of the operator to use,
%   which should be an operator declared with
%   \cs[module=amsmath,replace=false]{DeclareMathOperator}.
%
%   \item The \meta{csname} of the delimiter to use,
%   which should have been declared with
%   \cs[module=mathtools,replace=false]{DeclarePairedDelimiter}.
% \end{arguments}
%    \begin{macrocode}
\cs_new_protected:Nn \moremath_new_delim_op_command:NNN
{
  \cs_if_free:NTF #1
  {
    \exp_args:NNe \NewDocumentCommand #1
    { s o E{ ^ \char_generate:nn {`_} {8} }{{}{}} m }
    {
      \tl_if_novalue:nTF {##2}
      {
        % second argument empty
        \bool_if:nTF {##1}
        {
          % star given
          \moremath_delim_op_autoscale:NNnnn #2 #3 {##3} {##4} {##5}
        }{
          % star not given
          \moremath_delim_op_noscale:NNnnn #2 #3 {##3} {##4} {##5}
        }
      }{
        % second argument present
        % star given?
        \bool_if:nTF {##1}
        {
          % Warn if both star and scaling factor are present
          \msg_warning:nnn { moremath } { delimop / auto-manu-scale-conflict }
            {#1}
          \moremath_delim_op_autoscale:NNnnn #2 #3 {##3} {##4} {##5}
        }{ % FALSE BRANCH
        \moremath_delim_op_manuscale:NNnnnn #2 #3 {##2} {##3} {##4} {##5}
        }
      }
    }
  }{ % \cs_if_free:nTF #1 FALSE BRANCH
    \msg_warning:nnn { moremath } { delimop / already-defined-skip }
      {#1}
  }
}

\cs_generate_variant:Nn \moremath_new_delim_op_command:NNN {cNN}
%    \end{macrocode}
%   \changes{v0.1.0}{\GetVersionDate{v0.1.0}}{%
%     Added function and variant.
%   }
% \end{macro}
%
% \subsection{Document Level Commands}
% \label{sec:impl-delim-op-doc-cmds}
%
% \begin{macro}{\DeclareDelimitedOperator}
%     Finally provide the user with a command to declare an additional
%     delimited operator.
%    \begin{macrocode}
\NewDocumentCommand\DeclareDelimitedOperator { m m m }
{
  \msg_redirect_name:nnn { moremath } { delimop / already-defined-skip } { error }
  \moremath_new_delim_op_command:NNN #1 #2 #3
  \msg_redirect_name:nnn { moremath } { delimop / already-defined-skip } {}
}
%    \end{macrocode}
%    \changes{v0.1.0}{\GetVersionDate{v0.1.0}}{Added document level command.}
% \end{macro}
%
% As \pkg{amsmath}~\autocite{amsmath} pre-defines the following operators
% it is only sensible to also define delimited versions of them:\\
% \input{\jobname-ams-op-table.tex}
%
% \begin{macro}{\@@_new_delim_op_cmds:nN}
%   \cs{@@_new_delim_op_cmds:nN} creates document level macros of the form
%   \cs[no-index]{\meta{prefix}\meta{op name}}.
%   It declares fife versions \cs[no-index]{p\meta{op name}},
%   \cs[no-index]{b\meta{op name}}, \cs[no-index]{B\meta{op name}},
%   \cs[no-index]{v\meta{op name}} and \cs[no-index]{V\meta{op name}}.
%
%   The function takes two arguments:
%   \begin{arguments}
%     \item A token list which contains \meta{op name}
%
%     \item The \meta{csname} of the operator to use.
%   \end{arguments}
%    \begin{macrocode}
\cs_new_protected:Nn \@@_new_delim_op_cmds:nN
{
  \moremath_new_delim_op_command:cNN {p #1} #2 \@@_inparent:w
  \moremath_new_delim_op_command:cNN {b #1} #2 \@@_inbrack:w
  \moremath_new_delim_op_command:cNN {B #1} #2 \@@_inbrace:w
  \moremath_new_delim_op_command:cNN {v #1} #2 \@@_in_vert:w
  \moremath_new_delim_op_command:cNN {V #1} #2 \@@_in_Vert:w
}
%    \end{macrocode}
% \end{macro}
%
%  The decision if the following macros are defined depends on a package load
%  time option.
%    \begin{macrocode}
\bool_if:NTF \l_@@_predef_operators_bool
{
%    \end{macrocode}
% We define the commands for the operators already declared by amsmath.
% \begin{macro}[documented-as=\parccos]{
%   \parccos, \barccos, \Barccos, \varccos, \Varccos,
% }
% For \cs[module=amsmath,replace=false]{arccos},
%    \begin{macrocode}
\@@_new_delim_op_cmds:nN {arccos} \arccos
%    \end{macrocode}
%   \changes{v0.1.0}{\GetVersionDate{v0.1.0}}{%
%   Added delimited document commands for all \pkg{amsmath}-defined operators.
%   }
% \end{macro}
% \begin{macro}[documented-as=\parccos]{
%   \parcsin, \barcsin, \Barcsin, \varcsin, \Varcsin,
% }
% \cs[module=amsmath,replace=false]{arcsin},
%    \begin{macrocode}
\@@_new_delim_op_cmds:nN {arcsin} \arcsin
%    \end{macrocode}
% \end{macro}
% \begin{macro}[documented-as=\parccos]{
%   \parctan, \barctan, \Barctan, \varctan, \Varctan,
% }
% \cs[module=amsmath,replace=false]{arctan},
%    \begin{macrocode}
\@@_new_delim_op_cmds:nN {arctan} \arctan
%    \end{macrocode}
% \end{macro}
% \begin{macro}[documented-as=\parccos]{
%   \Parg, \barg, \Barg, \varg, \Varg,
% }
% \cs[module=amsmath,replace=false]{arg}
%    \begin{macrocode}
\@@_new_delim_op_cmds:nN {arg} \arg
%    \end{macrocode}
% \end{macro}
% \begin{macro}[documented-as=\parccos]{
%   \pcos, \bcos, \Bcos, \vcos, \Vcos,
% }
% \cs[module=amsmath,replace=false]{cos}
%    \begin{macrocode}
\@@_new_delim_op_cmds:nN {cos} \cos
%    \end{macrocode}
% \end{macro}
% \begin{macro}[documented-as=\parccos]{
%   \pcosh, \bcosh, \Bcosh, \vcosh, \Vcosh,
% }
% \cs[module=amsmath,replace=false]{cosh},
%    \begin{macrocode}
\@@_new_delim_op_cmds:nN {cosh} \cosh
%    \end{macrocode}
% \end{macro}
% \begin{macro}[documented-as=\parccos]{
%   \pcot, \bcot, \Bcot, \vcot, \Vcot,
% }
% \cs[module=amsmath,replace=false]{cot},
%    \begin{macrocode}
\@@_new_delim_op_cmds:nN {cot} \cot
%    \end{macrocode}
% \end{macro}
% \begin{macro}[documented-as=\parccos]{
%   \pcoth, \bcoth, \Bcoth, \vcoth, \Vcoth,
% }
% \cs[module=amsmath,replace=false]{coth},
%    \begin{macrocode}
\@@_new_delim_op_cmds:nN {coth} \coth
%    \end{macrocode}
% \end{macro}
% \begin{macro}[documented-as=\parccos]{
%   \pcsc, \bcsc, \Bcsc, \vcsc, \Vcsc,
% }
% \cs[module=amsmath,replace=false]{csc},
%    \begin{macrocode}
\@@_new_delim_op_cmds:nN {csc} \csc
%    \end{macrocode}
% \end{macro}
% \begin{macro}[documented-as=\parccos]{
%   \pdeg, \bdeg, \Bdeg, \vdeg, \Vdeg,
% }
% \cs[module=amsmath,replace=false]{deg},
%    \begin{macrocode}
\@@_new_delim_op_cmds:nN {deg} \deg
%    \end{macrocode}
% \end{macro}
% \begin{macro}[documented-as=\parccos]{
%   \pdet, \bdet, \Bdet, \vdet, \Vdet,
% }
% \cs[module=amsmath,replace=false]{det},
%    \begin{macrocode}
\@@_new_delim_op_cmds:nN {det} \det
%    \end{macrocode}
% \end{macro}
% \begin{macro}[documented-as=\parccos]{
%   \pdim, \bdim, \Bdim, \vdim, \Vdim,
% }
% \cs[module=amsmath,replace=false]{dim},
%    \begin{macrocode}
\@@_new_delim_op_cmds:nN {dim} \dim
%    \end{macrocode}
% \end{macro}
% \begin{macro}[documented-as=\parccos]{
%   \pexp, \bexp, \Bexp, \vexp, \Vexp,
% }
% \cs[module=amsmath,replace=false]{exp},
%    \begin{macrocode}
\@@_new_delim_op_cmds:nN {exp} \exp
%    \end{macrocode}
% \end{macro}
% \begin{macro}[documented-as=\parccos]{
%   \pgcd, \bgcd, \Bgcd, \vgcd, \Vgcd,
% }
% \cs[module=amsmath,replace=false]{gcd},
%    \begin{macrocode}
\@@_new_delim_op_cmds:nN {gcd} \gcd
%    \end{macrocode}
% \end{macro}
% \begin{macro}[documented-as=\parccos]{
%   \phom, \bhom, \Bhom, \vhom, \Vhom,
% }
% \cs[module=amsmath,replace=false]{hom},
%    \begin{macrocode}
\@@_new_delim_op_cmds:nN {hom} \hom
%    \end{macrocode}
% \end{macro}
% \begin{macro}[documented-as=\parccos]{
%   \pinf, \binf, \Binf, \vinf, \Vinf,
% }
% \cs[module=amsmath,replace=false]{inf},
%    \begin{macrocode}
\@@_new_delim_op_cmds:nN {inf} \inf
%    \end{macrocode}
% \end{macro}
% \begin{macro}[documented-as=\parccos]{
%   \pinjlim, \binjlim, \Binjlim, \vinjlim, \Vinjlim,
% }
% \cs[module=amsmath,replace=false]{injlim},
%    \begin{macrocode}
\@@_new_delim_op_cmds:nN {injlim} \injlim
%    \end{macrocode}
% \end{macro}
% \begin{macro}[documented-as=\parccos]{
%   \pker, \bker, \Bker, \vker, \Vker,
% }
% \cs[module=amsmath,replace=false]{ker},
%    \begin{macrocode}
\@@_new_delim_op_cmds:nN {ker} \ker
%    \end{macrocode}
% \end{macro}
% \begin{macro}[documented-as=\parccos]{
%   \plg, \blg, \Blg, \vlg, \Vlg,
% }
% \cs[module=amsmath,replace=false]{lg},
%    \begin{macrocode}
\@@_new_delim_op_cmds:nN {lg} \lg
%    \end{macrocode}
% \end{macro}
% \begin{macro}[documented-as=\parccos]{
%   \plim, \blim, \Blim, \vlim, \Vlim,
% }
% \cs[module=amsmath,replace=false]{lim},
%    \begin{macrocode}
\@@_new_delim_op_cmds:nN {lim} \lim
%    \end{macrocode}
% \end{macro}
% \begin{macro}[documented-as=\parccos]{
%   \pliminf, \bliminf, \Bliminf, \vliminf, \Vliminf,
% }
% \cs[module=amsmath,replace=false]{liminf},
%    \begin{macrocode}
\@@_new_delim_op_cmds:nN {liminf} \liminf
%    \end{macrocode}
% \end{macro}
% \begin{macro}[documented-as=\parccos]{
%   \plimsup, \blimsup, \Blimsup, \vlimsup, \Vlimsup,
% }
% \cs[module=amsmath,replace=false]{limsup},
%    \begin{macrocode}
\@@_new_delim_op_cmds:nN {limsup} \limsup
%    \end{macrocode}
% \end{macro}
% \begin{macro}[documented-as=\parccos]{
%   \pln, \bln, \Bln, \vln, \Vln,
% }
% \cs[module=amsmath,replace=false]{ln}
%    \begin{macrocode}
\@@_new_delim_op_cmds:nN {ln} \ln
%    \end{macrocode}
% \end{macro}
% \begin{macro}[documented-as=\parccos]{
%   \plog, \blog, \Blog, \vlog, \Vlog,
% }
% \cs[module=amsmath,replace=false]{log},
%    \begin{macrocode}
\@@_new_delim_op_cmds:nN {log} \log
%    \end{macrocode}
% \end{macro}
% \begin{macro}[documented-as=\parccos]{
%   \pmax, \bmax, \Bmax, \vmax, \Vmax,
% }
% \cs[module=amsmath,replace=false]{max},
%    \begin{macrocode}
\@@_new_delim_op_cmds:nN {max} \max
%    \end{macrocode}
% \end{macro}
% \begin{macro}[documented-as=\parccos]{
%   \pmin, \bmin, \Bmin, \vmin, \Vmin,
% }
% \cs[module=amsmath,replace=false]{min},
%    \begin{macrocode}
\@@_new_delim_op_cmds:nN {min} \min
%    \end{macrocode}
% \end{macro}
% \begin{macro}[documented-as=\parccos]{
%   \pPr, \bPr, \BPr, \vPr, \VPr,
% }
% \cs[module=amsmath,replace=false]{Pr},
%    \begin{macrocode}
\@@_new_delim_op_cmds:nN {Pr} \Pr
%    \end{macrocode}
% \end{macro}
% \begin{macro}[documented-as=\parccos]{
%   \pprojlim, \bprojlim, \Bprojlim, \vprojlim, \Vprojlim,
% }
% \cs[module=amsmath,replace=false]{projlim},
%    \begin{macrocode}
\@@_new_delim_op_cmds:nN {projlim} \projlim
%    \end{macrocode}
% \end{macro}
% \begin{macro}[documented-as=\parccos]{
%   \psec, \bsec, \Bsec, \vsec, \Vsec,
% }
% \cs[module=amsmath,replace=false]{sec},
%    \begin{macrocode}
\@@_new_delim_op_cmds:nN {sec} \sec
%    \end{macrocode}
% \end{macro}
% \begin{macro}[documented-as=\parccos]{
%   \psin, \bsin, \Bsin, \vsin, \Vsin,
% }
% \cs[module=amsmath,replace=false]{sin},
%    \begin{macrocode}
\@@_new_delim_op_cmds:nN {sin} \sin
%    \end{macrocode}
% \end{macro}
% \begin{macro}[documented-as=\parccos]{
%   \psinh, \bsinh, \Bsinh, \vsinh, \Vsinh,
% }
% \cs[module=amsmath,replace=false]{sinh},
%    \begin{macrocode}
\@@_new_delim_op_cmds:nN {sinh} \sinh
%    \end{macrocode}
% \end{macro}
% \begin{macro}[documented-as=\parccos]{
%   \psup, \bsup, \Bsup, \vsup, \Vsup,
% }
% \cs[module=amsmath,replace=false]{sup},
%    \begin{macrocode}
\@@_new_delim_op_cmds:nN {sup} \sup
%    \end{macrocode}
% \end{macro}
% \begin{macro}[documented-as=\parccos]{
%   \ptan, \btan, \Btan, \vtan, \Vtan,
% }
% \cs[module=amsmath,replace=false]{tan},
%    \begin{macrocode}
\@@_new_delim_op_cmds:nN {tan} \tan
%    \end{macrocode}
% \end{macro}
% \begin{macro}[documented-as=\parccos]{
%   \ptanh, \btanh, \Btanh, \vtanh, \Vtanh
% }
% and \cs[module=amsmath,replace=false]{tanh}.
%    \begin{macrocode}
\@@_new_delim_op_cmds:nN {tanh} \tanh
%    \end{macrocode}
% \end{macro}
% \begin{macro}[documented-as=\parccos]{
%   \pvarinjlim, \bvarinjlim, \Bvarinjlim, \vvarinjlim, \Vvarinjlim,
%   \pvarprojlim, \bvarprojlim, \Bvarprojlim, \vvarprojlim, \Vvarprojlim,
%   \pvarliminf, \bvarliminf, \Bvarliminf, \vvarliminf, \Vvarliminf,
%   \pvarlimsup, \bvarlimsup, \Bvarlimsup, \vvarlimsup, \Vvarlimsup,
% }
%   We also provide delimited versions of \cs[module=amsmath,replace=false]{varinjlim},
%   \cs[module=amsmath,replace=false]{varprojlim},
%   \cs[module=amsmath,replace=false]{varliminf},
%   and \cs[module=amsmath,replace=false]{varlimsup}.
%    \begin{macrocode}
\@@_new_delim_op_cmds:nN {varinjlim} \varinjlim
\@@_new_delim_op_cmds:nN {varprojlim} \varprojlim
\@@_new_delim_op_cmds:nN {varliminf} \varliminf
\@@_new_delim_op_cmds:nN {varlimsup} \varlimsup
%    \end{macrocode}
% \end{macro}
%    \begin{macrocode}
}{
  \msg_info:nnnn {moremath} {load /disabling} {no-operators}
  {
    predefined~delimited~operator~macros
  }
} % End of the conditional
%    \end{macrocode}
%
% ^^X SECTION: Vector Calculus Macros
% \section{Vector Calculus Macros}
% \label{sec:vector-calc}
%
% For providing macros which help with vector differentials,
% we first need some setup functions.
%
% \subsection{Macros Providing Symbols of Operators}
% \label{sec:op-symbols}
%
% \begin{macro}{
%   \@@_maybe_vcenter:n,
% }
%   Sometimes a symbol should be centered explicitly,
%   as this will depend on the current setting of "vcenter",
%   i.e.\ the current value of \cs{l_@@_vcenter_bool},
%   we provide a helper function which puts its argument
%   \begin{arguments}
%     \item A list of \meta{tokens}
%   \end{arguments}
%   inside a \tn{vcenter} by means of \cs{moremath_vcenter:n}
%   or not depending on the current value of \cs{l_@@_vcenter_bool}.
%    \begin{macrocode}
\cs_new_protected_nopar:Nn \@@_maybe_vcenter:n
{
  \bool_if:NTF \l_@@_vcenter_bool
  {
    \moremath_vcenter:n {#1}
  }{
    #1
  }
}
%    \end{macrocode}
% \end{macro}
%
% \begin{macro}{\@@_gradient_operator_get:}
%   This function returns the gradient operator depending on the current setting
%   of the keys.
%    \begin{macrocode}
\cs_new_protected:Nn \@@_gradient_operator_get:
{
  \tl_if_empty:NTF \l_@@_grad_op_tl
  {
    \mathop
    {
%    \end{macrocode}
% Otherwise we first need to check if the operator shall have an arrow over it.
% Afterwards if the nabla shall be bold.
%    \begin{macrocode}
      \bool_if:NTF \l_@@_nabla_arrow_bool
      {
        \vec
        {
%    \end{macrocode}
% In case \cs{l_@@_nabla_arrow_bool} is true we should \emph{maybe}
% center the symbol.
%    \begin{macrocode}
         \@@_maybe_vcenter:n
          {
            \bool_if:NT \l_@@_nabla_bold_bool
            {
              \boldsymbol
            }
            \l_@@_nabla_tl
          }
        }
      }{
%    \end{macrocode}
% Like in the case above we should \emph{maybe} center the symbol
% if \cs{l_@@_nabla_bold_bool} is true.
%    \begin{macrocode}
        \bool_if:NTF \l_@@_nabla_bold_bool
        {
          \@@_maybe_vcenter:n
          {
            \boldsymbol
            \l_@@_nabla_tl
          }
        }{
          \l_@@_nabla_tl
        }
      }
    }% \mathop
    \nolimits
  }{
%    \end{macrocode}
% If the user provided its own implementation of the operator,
% we simply return it.
%    \begin{macrocode}
    \l_@@_grad_op_tl
  }
}
%    \end{macrocode}
% \end{macro}
%
% \begin{macro}{\@@_laplace_operator_get:}
%   Then we define a function for returning the laplace operator symbol,
%   depending on the currently set keys.
%   The function wraps the symbol for the operator inside
%   \tn{mathop} to provide the right spacing.
%    \begin{macrocode}
\cs_new_protected:Nn \@@_laplace_operator_get:
{
  \tl_if_empty:NTF \l_@@_laplacian_tl
  {
    \mathop
    {
      \bool_if:NTF \l_@@_laplacian_delta_bool
      {
        \Delta
      }{
        \bool_if:NTF \l_@@_laplacian_arrow_bool
        {
          \vec{
            \@@_maybe_vcenter:n
            {
              \bool_if:NT \l_@@_nabla_bold_bool {\boldsymbol}
              \l_@@_laplacian_symb_tl
            }
          }
        }{
          \bool_if:NTF \l_@@_nabla_bold_bool
          {
            \@@_maybe_vcenter:n
            {
              \boldsymbol
              \l_@@_laplacian_symb_tl
            }
          }{
            \l_@@_laplacian_symb_tl
          }
        }
      }
    }\nolimits
    \bool_if:NF \l_@@_laplacian_delta_bool
    {
      \c_math_superscript_token
      {
        2
      }
    }
  }{
    \l_@@_laplacian_tl
  }
}
%    \end{macrocode}
% \end{macro}
%
% \begin{macro}{
%   \@@_dalembert_operator_get:
% }
%   This function returns the \foreignlanguage{french}{d'Alembert} operator
%   depending on the currently set keys. The symbol for the
%   \foreignlanguage{french}{d'Alembert} operator is wrapped inside \tn{mathop}
%   to provide proper spacing.
%    \begin{macrocode}
\cs_new_protected:Nn \@@_dalembert_operator_get:
{
  \mathop
  {
    \l_@@_dalembert_symb_tl
  }% \mathop
  \nolimits
}
%    \end{macrocode}
%   \changes{v0.2.0}{\GetVersionDate{v0.2.0}}{%
%     Added function: \cs{@@_dalembert_operator_get:}
%   }
% \end{macro}
%
% \subsection{Macros Producing an Operator}
% \label{sec:op-macros}
%
% After we have defined the symbols it is now time to provide a function which
% produces the entire gradient operator
% \begin{macro}{
%   \moremath_gradient_operator:n,
%   \moremath_laplace_operator:n
%  }
% This function takes one arguments, the subscript to use with the operator.
%    \begin{macrocode}
\cs_new_protected_nopar:Nn \moremath_gradient_operator:n
{
  \@@_gradient_operator_get:
  \tl_if_empty:nF {#1} {\c_math_subscript_token {#1}}
}
%    \end{macrocode}
%
% Like for the gradient operator we do the same for the laplacian
%    \begin{macrocode}
\cs_new_protected_nopar:Nn \moremath_laplace_operator:n
{
  \@@_laplace_operator_get:
  \tl_if_empty:nF {#1} {\c_math_subscript_token {#1}}
}
%    \end{macrocode}
%   \changes{v0.1.0}{\GetVersionDate{v0.1.0}}{%
%     Added functions.
%   }
% \end{macro}
%
% \begin{macro}{
%   \moremath_divergence_operator:n,
%   \moremath_curl_operator:n
% }
%   Using the already defined gradient operator it is possible to define a
%   function which acts as an operator suitable for representing the divergence
%   operator and the curl operator.
%   Like the gradient operator this functions take one argument
%   \begin{arguments}
%     \item the subscript of the gradient operator.
%   \end{arguments}
%    \begin{macrocode}
\cs_new_protected_nopar:Nn \moremath_divergence_operator:n
{
%    \end{macrocode}
%   The braces around \cs{moremath_gradient_operator:n} are necessary to avoid
%   issues with the spacing between \tn{cdot} and the following \tn{mathopen}
%   from any braces.\footnote{%
%   See: \url{https://tex.stackexchange.com/a/223914}
%   }
%   Example:
%   \[
%   \text{With braces:}\quad \pdiv{f(x)} \text{,}
%   \qquad\text{without braces: }\quad \mathop{\nabla}\nolimits\cdot(f(x))
%   \]
%    \begin{macrocode}
  {
    \moremath_gradient_operator:n {#1}
  }
  \cdot
}

\cs_new_protected_nopar:Nn \moremath_curl_operator:n
{
  {
    \moremath_gradient_operator:n {#1}
  }
  \times
}
%    \end{macrocode}
%   \changes{v0.1.0}{\GetVersionDate{v0.1.0}}{%
%     Added functions.
%   }
% \end{macro}
%
% \begin{macro}{
%   \moremath_dalembert_operator:n
% }
%   This function produces the \foreignlanguage{french}{d'Alembert} operator,
%   including an optional subscript.
%   The function takes one argument:
%   \begin{arguments}
%     \item A \meta{tl} with the contents of the subscript of the operator.
%
%     This variable may be empty,
%     in this case no subscript (not even an empty one) is produced.
%   \end{arguments}
%    \begin{macrocode}
\cs_new_protected:Nn \moremath_dalembert_operator:n
{
  \@@_dalembert_operator_get:
  \tl_if_empty:nF {#1}
  {
    \c_math_subscript_token {#1}
  }
}
%    \end{macrocode}
%   \changes{v0.2.0}{\GetVersionDate{v0.2.0}}{%
%     Added new function \cs{moremath_dalembert_operator:n}.
%   }
% \end{macro}
%
%
% \subsection{Producing Delimited Vector Calculus Operators}
% \label{sec:delim-vec-ops}
%
% \begin{macro}{
%   \moremath_delim_nabla_op_noscale:NNnn,
%   \moremath_delim_nabla_op_autoscale:NNnn
% }
%   These functions produce a vector calculus operator with \meta{contents}
%   inside delimiters.
%   The "autoscale" variant scales the delimiters with the size of \meta{contents},
%   the "noscale" variant does no scaling at all.
% \begin{arguments}
%   \item The \meta{csname} of the operator to use.
%
%         This should have the same form as \cs{moremath_gradient_operator:n},
%         i.e.\ the function passed as \meta{csname} should accept one argument
%         to typeset as subscript.
%
%   \item The \meta{csname} of the \emph{paired delimiter} to use.
%
%         The paired delimiter has to be one declared using \pkg{mathtools}'
%         \cs[module=mathtools,replace=false]{DeclarePairedDelimiter} command.
%
%   \item A \meta{tl} with the subscript of the operator.
%
%   \item A \meta{tl} with the contents to typeset inside the delimiters
% \end{arguments}
% We begin with the "noscale" version.
%    \begin{macrocode}
\cs_new_protected_nopar:Nn \moremath_delim_nabla_op_noscale:NNnn
{
  #1 {#3} #2{#4}
}
%    \end{macrocode}
% Then we create the version with automatic scaling.
%    \begin{macrocode}
\cs_new_protected_nopar:Nn \moremath_delim_nabla_op_autoscale:NNnn
{
  #1 {#3} #2 * {#4}
}
%    \end{macrocode}
%   \changes{v0.1.0}{\GetVersionDate{v0.1.0}}{%
%     Added "noscale" and "autoscale" functions.
%   }
% \end{macro}
%
% \begin{macro}{
%   \moremath_delim_nabla_op_manuscale:NNnnn
% }
%   This function produces a vector calculus operator like
%   \cs{moremath_delim_nabla_op_noscale:NNnn}
%   and \cs{moremath_delim_nabla_op_autoscale:NNnn},
%   but with manual scaling.
%   It takes five arguments:
%   \begin{arguments}
%     \item The \meta{csname} of the operator to use.
%
%     \item The \meta{csname} of the paired delimiter.
%
%     \item A \meta{token~list} containing the \meta{scale~cmd} like
%           \cs{big}, \cs{Big}, \cs{bigg}, etc.
%
%     \item A \meta{tl} with the subscript of the operator.
%
%     \item A \meta{tl} with the contents to typeset inside the delimiters.
%   \end{arguments}
%    \begin{macrocode}
\cs_new_protected_nopar:Nn \moremath_delim_nabla_op_manuscale:NNnnn
{
  #1 {#4} #2 [#3] {#5}
}
%    \end{macrocode}
% \begin{macro}{
%   \moremath_delim_nabla_op_manuscale:NNVnn
% }
%   We also declare a variant for passing a variable with the \meta{scale~cmd}
%   instead of a \meta{token~list}.
%    \begin{macrocode}
\cs_generate_variant:Nn \moremath_delim_nabla_op_manuscale:NNnnn {NNVnn}
%    \end{macrocode}
%   \changes{v0.1.0}{\GetVersionDate{v0.1.0}}{%
%     Add variant.
%   }
% \end{macro}
%   \changes{v0.1.0}{\GetVersionDate{v0.1.0}}{%
%     Add function.
%   }
% \end{macro}
%
% To ease the definition of those macros we define a function for defining
% the delimited  versions.
%
% Of course we also need a way to declare a user interface for this functions.
% For this purpose we first need some helpers for setting
% necessary parameters.
% \begin{macro}{\@@_parse_kv_args:nN}
%   This helper macro takes two arguments,
%   \begin{arguments}
%     \item A token list of the key-value arguments.
%     \item The \meta{csname} of a token list to put the scale value in.
%   \end{arguments}
%   The function sets the given keys but before it does so it searches for a
%   key named \verb|scale| and puts it into the second argument.
%   For this to work it is necessary that all values have \verb|=| in them.
%    \begin{macrocode}
\cs_new_protected:Nn \@@_parse_kv_args:nN
{
%    \end{macrocode}
%   We first put the key-value arguments inside a property list.
%   Afterwards we check if the key \verb|scale| has been given.
%   If yes we pop it and assign it to the second argument.
%   Otherwise we do nothing.
%    \begin{macrocode}
  \prop_set_from_keyval:Nn \l_tmpa_prop {#1}

  \prop_pop:NnNT \l_tmpa_prop {scale} #2 {}

  \keys_set:ne {moremath} {\prop_to_keyval:N \l_tmpa_prop}
}
%    \end{macrocode}
% \end{macro}
%
% We also need a warning message for conflicting arguments,
% to inform the user that one of his options is going to be ignored.
%    \begin{macrocode}
\msg_new:nnn { moremath } { vector-calc / scale-star-conflict }
{
  Both~star~and~scaling~factor~given~to~'#1'.\\
  Automatic~scaling~will~be~preferred,~the~size~command~'#2'~will~be~
  ignored~\msg_line_context:.
}
%    \end{macrocode}
%
% \begin{macro}{
%   \@@_new_delim_nabla_doc_cmd:NNN,
%   \@@_new_delim_nabla_doc_cmd:cNN,
% }
%   This function takes three arguments:
%   \begin{arguments}
%     \item The \meta{csname} of the to be defined command
%     \item The \meta{csname} of the operator to use.
%
%           This should be a function like \cs{moremath_gradient_operator:n}.
%     \item The \meta{csname} of the delimiter function to use.
%
%     This should be a macro/command created with \pkg{mathtools}
%     \cs[module=mathtools,replace=false]{DeclarePairedDelimiter} command.
%   \end{arguments}
%   Its purpose is to create a new document level command,
%   for the delimited vector calculus operators.
%    \begin{macrocode}
\cs_new_protected:Nn \@@_new_delim_nabla_doc_cmd:NNN
{
  \cs_if_free:NTF #1
  {
    \exp_args:NNe \NewDocumentCommand #1
    {
      s ={scale} o E{ \char_generate:nn {`_}{8} }{ {} } m
    }
    { % command code
      \group_begin:
      % optional arguments given?
      \tl_if_novalue:nF {##2}
      {
        \@@_parse_kv_args:nN {##2} \l_tmpa_tl
      }
      % star given?
      \bool_if:nTF {##1}
      {
        % scale factor given?
        \tl_if_empty:NF \l_tmpa_tl
        {
          \msg_warning:nnnV { moremath } { vector-calc / scale-star-conflict }
            {#1} \l_tmpa_tl
        }
        \moremath_delim_nabla_op_autoscale:NNnn #2 #3 {##3} {##4}
      }{ % \bool_if:nTF {##1} FALSE BRANCH
        % scale factor given?
        \tl_if_empty:NTF \l_tmpa_tl
        {
          \moremath_delim_nabla_op_noscale:NNnn #2 #3 {##3} {##4}
        }{ % FALSE BRANCH
          \moremath_delim_nabla_op_manuscale:NNVnn #2 #3 \l_tmpa_tl {##3} {##4}
        }
      } % \bool_if:nTF {##1}
      \group_end:
    }
  }{ % \cs_if_free:NTF #1 FALSE BRANCH
    \msg_warning:nnn { moremath } { vector-calc / already-defined-skip } {#1}
  } % \cs_if_free:NTF #1
}
%
\cs_generate_variant:Nn \@@_new_delim_nabla_doc_cmd:NNN {cNN}
%    \end{macrocode}
% \end{macro}
%
% \begin{macro}{\@@_new_nabla_doc_cmds:nN}
%   This internal function creates five different document level commands at once,
%   using \cs{@@_new_delim_nabla_doc_cmd:cNN}.
%   It takes two arguments:
%   \begin{arguments}
%     \item The a \meta{suffix~tl} for constructing the command names.
%
%     The resulting commands will have the form \cs[no-index]{p\meta{suffix}},
%     \cs[no-index]{b\meta{suffix}}, \cs[no-index]{B\meta{suffix}},
%     \cs[no-index]{v\meta{suffix}} and \cs[no-index]{V\meta{suffix}}.
%
%     \item The \meta{csname} of the operator to use
%   \end{arguments}
%    \begin{macrocode}
\cs_new_protected:Nn \@@_new_nabla_doc_cmds:nN
{
  \@@_new_delim_nabla_doc_cmd:cNN { p #1 } #2 \@@_inparent:w
  \@@_new_delim_nabla_doc_cmd:cNN { b #1 } #2 \@@_inbrack:w
  \@@_new_delim_nabla_doc_cmd:cNN { B #1 } #2 \@@_inbrace:w
  \@@_new_delim_nabla_doc_cmd:cNN { v #1 } #2 \@@_in_vert:w
  \@@_new_delim_nabla_doc_cmd:cNN { V #1 } #2 \@@_in_Vert:w
}
%    \end{macrocode}
% \end{macro}
%
% \subsection{Document Level Commands}
% \label{sec:vec-doc-lvl-commands}
%
% The predefined macros for vector calculus are also guarded by a package
% option to be conditionally disabled by the user.
%    \begin{macrocode}
\bool_if:NTF \l_@@_predef_vector_op_bool
{
%    \end{macrocode}
% \subsubsection{Standalone Operators}
% \label{par:impl-vector-calc-standalone}
% The user might want to use also a non-delimited version
% of the vector calculus operators,
% we provide them with a standalone version of those.
%
% As the definition of a new document command can fail if the \meta{csname}
% clashes with some already defined macro,
% we define an error message to use when defining document level commands.
%    \begin{macrocode}
\msg_new:nnn { moremath } { vector-calc / already-defined-skip }
{
  Control~sequence~'#1'~is~already~defined.\\
  Skipping~definition~\msg_line_context:.
}
%    \end{macrocode}
%
% \begin{macro}{
%   \grad,
%   \divergence,
%   \curl,
%   \laplacian
% }
% Now we provide the user with document level commands for
% \cs[no-index]{moremath_\meta{op}_operator:n}.
%    \begin{macrocode}
\cs_if_free:NTF \grad
{
  \exp_args:NNe \NewDocumentCommand \grad { !o E{ \char_generate:nn {`_}{8} }{{}} }
  {
    \group_begin:
    \tl_if_novalue:nF {#1}
    {
      \keys_set:nn {moremath} {#1}
    }
    \moremath_gradient_operator:n {#2}
    \group_end:
  }
}{
  \msg_warning:nnn {moremath} { vector-calc / already-defined-skip } {\grad}
}

\cs_if_free:NTF \divergence
{
  \exp_args:NNe \NewDocumentCommand \divergence
  { !o E{ \char_generate:nn {`_} {8} }{{}} }
  {
    \group_begin:
    \tl_if_novalue:nF {#1}
    {
      \keys_set:nn {moremath} {#1}
    }
    \moremath_divergence_operator:n {#2}
    \group_end:
  }
}{
  \msg_warning:nnn {moremath} { vector-calc / already-defined-skip } {\divergence}
}

\cs_if_free:NTF \curl
{
  \exp_args:NNe \NewDocumentCommand \curl
    { !o E{ \char_generate:nn {`_}{8} }{{}} }
  {
    \group_begin:
    \tl_if_novalue:nF {#1}
    {
      \keys_set:nn {moremath} {#1}
    }
    \moremath_curl_operator:n {#2}
    \group_end:
  }
}{
  \msg_warning:nnn {moremath} {vector-calc / already-defined-skip} {\curl}
}

\cs_if_free:NTF \laplacian
{
  \exp_args:NNe \NewDocumentCommand \laplacian
  { !o E{ \char_generate:nn {`_}{8} }{{}} }
  {
    \group_begin:
    \tl_if_novalue:nF {#1}
    {
      \keys_set:nn {moremath} {#1}
    }
    \moremath_laplace_operator:n {#2}
    \group_end:
  }
}{
  \msg_warning:nnn {moremath} { vector-calc / already-defined-skip }
  {\laplacian}
}
%    \end{macrocode}
% \begin{macro}{\quabla}
%   We now also define a command to use as a standalone
%   \foreignlanguage{french}{d'Alembert} operator.
%   As the name \cs[no-index]{dalembertian} is a bit cumbersome to type out,
%   I've decided to use one of its alternate names \cs{quabla}
%    \begin{macrocode}
\cs_if_free:NTF \quabla
{
  \exp_args:NNe \NewDocumentCommand \quabla
    { !o E{ \char_generate:nn {`_} { 8 } }{{}} }
  {
    \group_begin:
    \tl_if_novalue:nF {#1}
    {
      \keys_set:nn { moremath } {#1}
    }
    \moremath_dalembert_operator:n {#2}
    \group_end:
  }
}{ % \cs_if_free:NTF \quabla FALSE BRANCH
  \msg_warning:nnn { moremath } { vector-calc / already-defined-skip }
    {\quabla}
}
%    \end{macrocode}
%   \changes{v0.2.0}{\GetVersionDate{v0.2.0}}{
%     Added \cs{quabla} command.
%   }
%   \changes{v0.3.0}{\GetVersionDate{v0.3.0}}{
%     Changed: Spaces between the \meta{csname}
%     and the optional argument are now disallowed.
%   }
% \end{macro}^^X  \quabla
%   \changes{v0.1.0}{\GetVersionDate{v0.1.0}}{
%     Added \cs{grad}, \cs{divergence}, \cs{curl} and \cs{laplacian}
%     commands.
%   }
%   \changes{v0.3.0}{\GetVersionDate{v0.3.0}}{
%     Changed: Spaces between the \meta{csname}
%     and the optional argument are now disallowed
%     for \cs{grad}, \cs{divergence}, \cs{curl} and \cs{laplacian}.
%   }
% \end{macro}
%
% \subsubsection{Operators with Delimiters}
% \label{par:impl-vec-calc-op-delim}
%
% \begin{macro}{
%   \pgrad,
%   \bgrad,
%   \Bgrad,
%   \vgrad,
%   \Vgrad,
% }
%   Now lets declare the delimited gradient operators.
%   We provide five versions using parenthesis, brackets, braces, single \tn{vert},
%   and double \tn{Vert}.
%    \begin{macrocode}
\@@_new_nabla_doc_cmds:nN {grad} \moremath_gradient_operator:n
%    \end{macrocode}
%   \changes{v0.1.0}{\GetVersionDate{v0.1.0}}{%
%     Added commands for delimited gradient operators.
%   }
% \end{macro}
%
% \begin{macro}{
%   \pdiv,
%   \bdiv,
%   \Bdiv,
%   \vdiv,
%   \Vdiv
% }
% Now we do the same for the divergence operator.
%    \begin{macrocode}
\@@_new_nabla_doc_cmds:nN {div} \moremath_divergence_operator:n
%    \end{macrocode}
%   \changes{v0.1.0}{\GetVersionDate{v0.1.0}}{%
%     Added commands for delimited divergence operators.
%   }
% \end{macro}
%
% \begin{macro}{
%   \pcurl,
%   \bcurl,
%   \Bcurl,
%   \vcurl,
%   \Vcurl
% }
% Now to the curl macros.
%    \begin{macrocode}
\@@_new_nabla_doc_cmds:nN {curl} \moremath_curl_operator:n
%    \end{macrocode}
%   \changes{v0.1.0}{\GetVersionDate{v0.1.0}}{%
%     Added commands for delimited curl operators.
%   }
% \end{macro}
%
% \begin{macro}{
%   \plaplacian,
%   \blaplacian,
%   \Blaplacian,
%   \vlaplacian,
%   \Vlaplacian
% }
% Next we take care of the laplacian
%    \begin{macrocode}
\@@_new_nabla_doc_cmds:nN {laplacian} \moremath_laplace_operator:n
%    \end{macrocode}
%   \changes{v0.1.0}{\GetVersionDate{v0.1.0}}{%
%     Added commands for delimited Laplace operators.
%   }
% \end{macro}
%
% \begin{macro}{
%   \pquabla,
%   \bquabla,
%   \Bquabla,
%   \vquabla,
%   \Vquabla
% }
%   Finally we define delimited commands for the
%   \foreignlanguage{french}{d'Alembert} operator.
%    \begin{macrocode}
\@@_new_nabla_doc_cmds:nN {quabla} \moremath_dalembert_operator:n
%    \end{macrocode}
%   \changes{v0.2.0}{\GetVersionDate{v0.2.0}}{
%     Added commands for a delimited \foreignlanguage{french}{d'Alembert}
%     operator.
%   }
% \end{macro}
%
% If the user issued \texttt{no-vector} as a package loading option,
% write this to the log file.
%    \begin{macrocode}
}{ % IF NOT \l_@@_predef_vector_op_bool
  \msg_info:nnnn {moremath} { load /disabling } {no-vector}
  {
    predefined~vector~calculus~macros
  }
} % END \l_@@_predef_vector_op_bool
%    \end{macrocode}
%
% \section{Macros Producing Matrices and Vectors}
% \label{sec:impl-matr-vec}
%
% \subsection{Producing Row and Column Vectors}
%
% The functions in this section are used to generate row and column vectors
% utilizing \pkg{mathtools}~\autocite{mathtools} \env{matrix*} and
% \env{smallmatrix*} environments.
%
% \begin{variable}{\l_@@_vector_entries_seq}
%   For generating row or column vectors it is necessary to store the entries
%   inside a variable. \cs{l_@@_vector_entries_seq} is used for this purpose.
%    \begin{macrocode}
\seq_new:N \l_@@_vector_entries_seq
%    \end{macrocode}
% \end{variable}
%
% Then we need a function for formatting the actual entries of the vector.
% We need two version one for the row vector and one for the column vector
% version.
% \begin{macro}{
%   \@@_seq_to_column_vector:N,
%   \@@_seq_to_row_vector:N
% }
%   These functions take one argument
%   \begin{arguments}
%     \item A sequence which should be converted to the contents of the single
%     column/row matrix.
%   \end{arguments}
%   They format the input suitable to be put inside a \env{matrix*} environment.
% We begin with the column vector version.
%    \begin{macrocode}
\cs_new_protected_nopar:Nn \@@_seq_to_column_vector:N
{
  \seq_use:Nn #1 {\\}
}
%    \end{macrocode}
% Then we get to the row vector.
%    \begin{macrocode}
\cs_new_protected_nopar:Nn \@@_seq_to_row_vector:N
{
  \seq_use:Nn #1 {&}
}
%    \end{macrocode}
% \end{macro}
%
% In the next step we construct the single row/column matrix from the user
% provided input.
% \begin{macro}{
%   \moremath_column_vector:nn,
%   \moremath_row_vector:nn,
% }
%   The two commands \cs{moremath_column_vector:n} and \cs{moremath_row_vector:n}
%   construct the matrix, they both take two arguments.
%   \begin{arguments}
%     \item The delimiter specifier.
%
%     This should be one of the prefixes of the \env{\meta{prefix}matrix*},
%     environments.
%
%     Another possibility is to issue |small| as this parameter to get
%     an inline matrix.
%
%     \item The comma separated contents of the matrix.
%   \end{arguments}
%     \begin{macrocode}
\cs_new_protected_nopar:Nn \moremath_column_vector:nn
{
  \seq_clear:N \l_@@_vector_entries_seq
  \seq_set_from_clist:Nn \l_@@_vector_entries_seq {#2}

  \exp_args:NnNV \begin{#1 matrix*} [ \l_@@_matrix_align_tl ]
    \@@_seq_to_column_vector:N \l_@@_vector_entries_seq
  \end{#1 matrix*}
}

\cs_new_protected_nopar:Nn \moremath_row_vector:nn
{
  \seq_clear:N \l_@@_vector_entries_seq
  \seq_set_from_clist:Nn \l_@@_vector_entries_seq {#2}

  \exp_args:NnNV \begin{#1matrix*} [ \l_@@_matrix_align_tl ]
    \@@_seq_to_row_vector:N \l_@@_vector_entries_seq
  \end{#1matrix*}
}
%    \end{macrocode}
%   \changes{v0.1.0}{\GetVersionDate{v0.1.0}}{%
%     Added functions producing row- and column vectors.
%   }
% \end{macro}
%
% \begin{macro}{
%   \moremath_column_smallvector:nn,
%   \moremath_row_smallvector:nn
% }
%   To make the code more readable, we define functions specifically for
%   creating row and column vectors using the \env{smallmatrix*} family of
%   environments. These functions take the same arguments as the non-small
%   versions above.
%    \begin{macrocode}
\cs_new_protected_nopar:Nn \moremath_column_smallvector:nn
{
  \moremath_column_vector:nn {#1 small} {#2}
}

\cs_new_protected_nopar:Nn \moremath_row_smallvector:nn
{
  \moremath_row_vector:nn {#1 small} {#2}
}
%    \end{macrocode}
%   \changes{v0.1.0}{\GetVersionDate{v0.1.0}}{%
%     Added functions producing inline math row- and column vectors.
%   }
% \end{macro}
%
% ^^X SUBSECTION: Shorthands for simple matrices
% \subsection{Shorthands for Simple Matrices}
% \label{sec:impl-matr-shorth}
%
% The construction of several simple matrices like diagonal matrices,
% can be simplified as there is no need to use the \env{matrix} environment.\footnote{%
% The code in this section is heavily inspired by the following answer on
% \TeX{}SE: \url{https://tex.stackexchange.com/a/539741}
% }
%
% \begin{variable}{
%   \l_@@_mat_diag_entries_seq,
%   \l_@@_mat_row_entries_seq,
% }
% We construct the matrices row by row, therefore we need to store the
% currently constructed row inside a variable.
% The same is true for the diagonal entries which also need to be stored
% somewhere.
% We therefore declare two sequence variables \cs{l_@@_mat_diag_entries_seq}
% and \cs{l_@@_mat_row_seq} for this purpose
%    \begin{macrocode}
\seq_clear_new:N \l_@@_mat_diag_entries_seq
\seq_clear_new:N \l_@@_mat_row_entries_seq
%    \end{macrocode}
% \end{variable}
%
% \subsubsection{(Anti-)diagonal matrices}
% We split the construction of the matrix into multiple parts,
% utilizing internal helper functions.
% \begin{macro}{
%   \@@_constr_diagmat_row:n,
%   \@@_constr_antidiagmat_row:n
% }
%   \cs{@@_constr_diagmat_row:n} and \cs{@@_constr_antidiagmat_row:n}
%   take one integer argument:
%   \begin{arguments}
%     \item The number of the current row.
%   \end{arguments}
%   They both construct a matrix row and place it inside the input stream.
%   They take the content of the (anti-)diagonal from the variable
%   \cs{l_@@_mat_diag_entries_seq}
%   and use the variable \cs{l_@@_mat_row_entries_seq}
%   to store the current row.
%    \begin{macrocode}
\cs_new_protected:Nn \@@_constr_diagmat_row:n
{
  \seq_clear:N \l_@@_mat_row_entries_seq
%    \end{macrocode}
%   As all diagonal matrices \(M \in A^{m \times n}\), where \(A\) is a field,
%   are quadratic i.e.\ \(A^{m \times n} \equiv A^{n \times n}\) the length of the
%   diagonal sequence equals the number of rows and columns of the matrix.
%   We exploit this fact here.
%    \begin{macrocode}
  \int_step_inline:nn {\seq_count:N \l_@@_mat_diag_entries_seq}
  {
    \int_compare:nTF { #1 == ##1 }
    {
      \seq_put_right:Nx \l_@@_mat_row_entries_seq
      {
        \seq_item:Nn \l_@@_mat_diag_entries_seq { #1 }
      }
    }{ % false branch
      \seq_put_right:NV \l_@@_mat_row_entries_seq \l_@@_matrix_fill_tl
    } % \int_compare:nTF { #1 == ##1 }
  }
  \seq_use:Nn \l_@@_mat_row_entries_seq { & } \\
}

% Anti-diagonal version
\cs_new_protected:Nn \@@_constr_antidiagmat_row:n
{
  \seq_clear:N \l_@@_mat_row_entries_seq
  \int_step_inline:nn {\seq_count:N \l_@@_mat_diag_entries_seq}
  {
    \int_compare:nTF { #1 == ##1 }
    {
      % as this is a anti diagonal matrix we put in the elements from the
      % left so that the first entry is the right most entry
      \seq_put_left:Nx \l_@@_mat_row_entries_seq
      {
        \seq_item:Nn  \l_@@_mat_diag_entries_seq { #1 }
      }
    }{ % false branch
      \seq_put_left:NV \l_@@_mat_row_entries_seq \l_@@_matrix_fill_tl
    } % \int_compare:nTF { #1 == ##1 }
  }
  \seq_use:Nn \l_@@_mat_row_entries_seq { & } \\
}
%    \end{macrocode}
% \end{macro}
%
% \begin{macro}{
%   \moremath_diagonal_matrix:nn,
%   \moremath_antidiagonal_matrix:nn,
%   \moremath_diagonal_smallmatrix:nn,
%   \moremath_antidiagonal_smallmatrix:nn,
% }
%   The \cs[no-index]{moremath_\meta{a or d}_matrix:nn} and
%   \cs[no-index]{moremath_\meta{a or d}_smallmatrix:nn} family of functions
%   produce a matrix based on \pkg{mathtools}~\autocite{mathtools}
%   \env{matrix*} environment.
%   The functions take two arguments.
%   \begin{arguments}
%     \item The delimiter specifier.
%
%     This should be one of the prefixes of the \env{\meta{prefix}matrix*}
%     environments.
%
%     \item The comma separated contents of the (anti-)diagonal.
%   \end{arguments}
%   These functions also use the values of the variables
%   \cs{l_@@_matrix_fill_tl} and \cs{l_@@_matrix_align_tl}.
%   And modifies the variable \cs{l_@@_mat_diag_entries_seq}.
%    \begin{macrocode}
\cs_new_protected:Nn \moremath_diagonal_matrix:nn
{
  \seq_set_from_clist:Nn \l_@@_mat_diag_entries_seq { #2 }
  \exp_args:NnNV \begin{#1 matrix*} [ \l_@@_matrix_align_tl ]
    \int_step_function:nN { \seq_count:N \l_@@_mat_diag_entries_seq }
      \@@_constr_diagmat_row:n
  \end{#1 matrix*}
}

% Anti-diagonal matrix
\cs_new_protected:Nn \moremath_antidiagonal_matrix:nn
{
  \seq_set_from_clist:Nn \l_@@_mat_diag_entries_seq { #2 }
  \exp_args:NnNV \begin{ #1 matrix* } [ \l_@@_matrix_align_tl ]
    \int_step_function:nN { \seq_count:N \l_@@_mat_diag_entries_seq }
      \@@_constr_antidiagmat_row:n
  \end{ #1 matrix* }
}

% Small versions
\cs_new_protected:Nn \moremath_diagonal_smallmatrix:nn
{
  \moremath_diagonal_matrix:nn {#1 small} {#2}
}

\cs_new_protected:Nn \moremath_antidiagonal_smallmatrix:nn
{
  \moremath_antidiagonal_matrix:nn {#1 small} {#2}
}
%    \end{macrocode}
% \begin{macro}{
%   \moremath_diagonal_matrix:nV,
%   \moremath_diagonal_matrix:Vn,
%   \moremath_diagonal_matrix:VV,
%   \moremath_antidiagonal_matrix:nV,
%   \moremath_antidiagonal_matrix:Vn,
%   \moremath_antidiagonal_matrix:VV,
%   \moremath_diagonal_smallmatrix:nV,
%   \moremath_diagonal_smallmatrix:Vn,
%   \moremath_diagonal_smallmatrix:VV,
%   \moremath_antidiagonal_smallmatrix:nV,
%   \moremath_antidiagonal_smallmatrix:Vn,
%   \moremath_antidiagonal_smallmatrix:VV,
% }
% For convenience we also define some variants of the above functions.
%    \begin{macrocode}
\cs_generate_variant:Nn \moremath_diagonal_matrix:nn { n V }
\cs_generate_variant:Nn \moremath_diagonal_matrix:nn { V n }
\cs_generate_variant:Nn \moremath_diagonal_matrix:nn { V V }
\cs_generate_variant:Nn \moremath_antidiagonal_matrix:nn { n V }
\cs_generate_variant:Nn \moremath_antidiagonal_matrix:nn { V n }
\cs_generate_variant:Nn \moremath_antidiagonal_matrix:nn { V V }
\cs_generate_variant:Nn \moremath_diagonal_smallmatrix:nn { n V }
\cs_generate_variant:Nn \moremath_diagonal_smallmatrix:nn { V n }
\cs_generate_variant:Nn \moremath_diagonal_smallmatrix:nn { V V }
\cs_generate_variant:Nn \moremath_antidiagonal_smallmatrix:nn { n V }
\cs_generate_variant:Nn \moremath_antidiagonal_smallmatrix:nn { V n }
\cs_generate_variant:Nn \moremath_antidiagonal_smallmatrix:nn { V V }
%    \end{macrocode}
%   \changes{v0.1.0}{\GetVersionDate{v0.1.0}}{Added variants.}
%   \changes{v0.2.0}{\GetVersionDate{v0.2.0}}{Added "nV" and "VV" variants.}
% \end{macro}
% \changes{v0.1.0}{\GetVersionDate{v0.1.0}}{%
% Added functions producing (anti-)diagonal matrices.
% Also added versions for inline math.
% }
% \end{macro}
%
%
% \subsubsection{Identity Matrices}
% As we already have functions available for producing diagonal matrices,
% it makes only sense to also provide a shorthand for producing an identity matrix,
% i.e.\ a diagonal matrix with \enquote{1} along the diagonal.
%
%
% \begin{macro}{
%   \@@_generate_one_filled_clist:Nn
% }
%   The function \cs{@@_generate_one_filled_clist:Nn} produces a \meta{clist},
%   consisting only of \enquote{1} as entries.
%   This function takes two arguments:
%   \begin{arguments}
%     \item The \emph{csname} of a \meta{clist~var} to store the \meta{clist}
%     into.
%
%     \item An \meta{int} to represent the number of entries to produce.
%   \end{arguments}
%    \begin{macrocode}
\cs_new_protected_nopar:Nn \@@_generate_one_filled_clist:Nn
{
  \seq_clear:N \l_tmpa_seq
  \int_step_inline:nn {#2}
  {
    \seq_put_right:NV \l_tmpa_seq \c_one_int
  }
  \clist_set_from_seq:NN #1 \l_tmpa_seq
}
%    \end{macrocode}
% \begin{macro}{
%   \@@_generate_one_filled_clist:NV
% }
% We also define a variant accepting an integer variable.
%    \begin{macrocode}
\cs_generate_variant:Nn \@@_generate_one_filled_clist:Nn { N V }
%    \end{macrocode}
% \end{macro}
%   \changes{v0.2.0}{\GetVersionDate{v0.2.0}}{%
%     Added function.
%   }
% \end{macro}
%
% \begin{variable}{
%   \l_@@_id_entries_clist
% }
%   As we want to utilize the \cs{moremath_diagonal_matrix:VV} and
%   \cs{moremath_diagonal_smallmatrix:VV} functions for creating the identity
%   matrix we declare an internal \meta{clist~var} called
%   \cs{l_@@_id_entries_clist} for passing the \meta{clist} around.
%    \begin{macrocode}
\clist_new:N \l_@@_id_entries_clist
%    \end{macrocode}
%   \changes{v0.2.0}{\GetVersionDate{v0.2.0}}{Added variable.}
% \end{variable}
%
% \begin{macro}{
%   \moremath_id_matrix:n,
%   \moremath_id_smallmatrix:n
% }
%   These functions are intended to produce an identity matrix from an integer
%   expression.
%   They take one argument.
%   \begin{arguments}
%     \item The number of diagonal entries.
%   \end{arguments}
%    \begin{macrocode}
\cs_new_protected_nopar:Nn \moremath_id_matrix:n
{
  \clist_clear:N \l_@@_id_entries_clist
  \@@_generate_one_filled_clist:Nn \l_@@_id_entries_clist {#1}
  \moremath_diagonal_matrix:VV \l_@@_matrix_delim_tl \l_@@_id_entries_clist
}
%    \end{macrocode}
%
%    \begin{macrocode}
\cs_new_protected_nopar:Nn \moremath_id_smallmatrix:n
{
  \clist_clear:N \l_@@_id_entries_clist
  \@@_generate_one_filled_clist:Nn \l_@@_id_entries_clist {#1}
  \moremath_diagonal_smallmatrix:VV \l_@@_matrix_delim_tl \l_@@_id_entries_clist
}
%    \end{macrocode}
%
% \begin{macro}{
%   \moremath_id_matrix:V,
%   \moremath_id_smallmatrix:V
% }
% We also provide variants, which accepts a "V"-type argument:
%    \begin{macrocode}
\cs_generate_variant:Nn \moremath_id_matrix:n { V }
\cs_generate_variant:Nn \moremath_id_smallmatrix:n { V }
%    \end{macrocode}
% \end{macro}
%
%   \changes{v0.2.0}{\GetVersionDate{v0.2.0}}{%
%   Added functions \cs{moremath_id_matrix:n} and \cs{moremath_id_smallmatrix:n}
%   and variants.
%   }
% \end{macro}
%
% \subsection{Document Level Commands}
% \label{sec:mat-doc-lvl-cmds}
%
% Now we define document level commands for the previously defined
% functions.
%
% But before we do so we define a message to be issued in case the targeted
% \meta{csname} is already defined elsewhere.
%    \begin{macrocode}
\msg_new:nnnn { moremath } { matrix / already-defined-doc-cmd-skip }
{
  Control~sequence~'#1'~is~already~defined.\\
  Skipping~definition~\msg_line_context:.
}
{
  The~control~sequence~'#1'~has~already\\
  been~defined~by~some~other~package.\\
  And~I~am~refusing~to~overwrite~the~existing~definition,\\
  therefore~I~am~skipping~the~definition~of~this~command.
}
%    \end{macrocode}
%
% \subsubsection{Row and Column Vectors}
% We begin with the row and column vector functions.
% As with the other document level commands, we guard the definitions
% with a key value option, so that the user can disable them.
%    \begin{macrocode}
\bool_if:nTF \l_@@_predef_crvector_bool
{
%    \end{macrocode}
% \begin{macro}{
%   \cvector,
%   \rvector,
%   \smallcvector,
%   \smallrvector
% }
% First we define the document level command for the bare column vector
%    \begin{macrocode}
\cs_if_free:NTF \cvector
{
  \NewDocumentCommand \cvector { o m }
  {
    \group_begin:
    \tl_if_novalue:nF {#1}
    {
      \keys_set:nn { moremath / matrix } {#1}
    }
    \moremath_column_vector:nn {\l_@@_matrix_delim_tl} {#2}
    \group_end:
  }
}{
  % issue a warning message if the csname is already taken.
  \msg_warning:nnn { moremath } { matrix / already-defined-doc-cmd-skip }
  {
    \cvector
  }
} % \cs_if_free:NTF \cvector
%    \end{macrocode}
% and the row vector.
%    \begin{macrocode}
\cs_if_free:NTF \rvector
{
  \NewDocumentCommand \rvector { o m }
  {
    \group_begin:
    \tl_if_novalue:nF {#1}
    {
      \keys_set:nn { moremath / matrix } {#1}
    }
    \moremath_row_vector:nn {\l_@@_matrix_delim_tl} {#2}
    \group_end:
  }
}{
  % warn if csname is already taken
  \msg_warning:nnn { moremath } { matrix / already-defined-doc-cmd-skip}
  {
    \rvector
  }
} % \cs_if_free:NTF \rvector
%    \end{macrocode}
% Then we define the smaller inline versions of those commands.
%    \begin{macrocode}
\cs_if_free:NTF \smallcvector
{
  \NewDocumentCommand \smallcvector { o m }
  {
    \group_begin:
    \tl_if_novalue:nF {#1}
    {
      \keys_set:nn {moremath / matrix} {#1}
    }
    \moremath_column_smallvector:nn {\l_@@_matrix_delim_tl} {#2}
    \group_end:
  }
}{
  % Issue a warning message if the csname is already taken
  \msg_warning:nnn { moremath } { matrix / already-defined-doc-cmd-skip }
  {
    \smallcvector
  }
} % \cs_if_free:NTF \smallcvector

\cs_if_free:NTF \smallrvector
{
  \NewDocumentCommand \smallrvector { o m }
  {
    \group_begin:
    \tl_if_novalue:nF {#1}
    {
      \keys_set:nn { moremath / matrix } {#1}
    }
    \moremath_row_smallvector:nn {\l_@@_matrix_delim_tl} {#2}
    \group_end:
  }
}{
  % warn if csname is taken
  \msg_warning:nnn { moremath } { matrix / already-defined-doc-cmd-skip }
  {
    \smallrvector
  }
}
%    \end{macrocode}
%   \changes{v0.1.0}{\GetVersionDate{v0.1.0}}{%
%   Added commands producing row and column vectors,
%   including inline math versions.
%   }
% \end{macro}
%
% \paragraph{Commands with Pre-defined Delimiters}
% Next we define several shorthands to for the commonly used delimiters,
% to avoid code duplication, we first create some helper functions which
% define those functions.
% \begin{macro}{
%   \@@_new_vector_shorth_doc_cmd:NNn
% }
%   The function \cs{@@_new_vector_shorth_doc_cmd} creates a new vector shorthand,
%   command. It takes three arguments:
%   \begin{arguments}
%     \item The \meta{csname} to be defined.
%     \item The \meta{function} to use for this shorthand.
%
%     This should be one of the
%     \cs[no-index]{moremath_\meta{type}_\meta{size}vector:nn}
%     like commands.
%
%     \item The \meta{delimiter} to use.
%
%      Usually one of |p|, |b|, |B|, |v|, |V|.
%   \end{arguments}
%    \begin{macrocode}
\cs_new_protected:Nn \@@_new_vector_shorth_doc_cmd:NNn
{
  \cs_if_free:NTF #1
  {
    \NewDocumentCommand #1 { o m }
    {
      \group_begin:
      % set the delimiter key pre-set for this function
      \keys_set:nn {moremath / matrix } {delimiter = #3}
      \tl_if_novalue:nF {##1}
      {
        \keys_set:nn {moremath / matrix } {##1}
      }
      #2 {\l_@@_matrix_delim_tl} {##2}
      \group_end:
    }
  }{
    % warn if csname is taken
    \msg_warning:nnn { moremath } { matrix / already-defined-doc-cmd-skip }
    {
      #1
    }
  }
}
%    \end{macrocode}
% \end{macro}
%
% \begin{macro}{
%   \pcvector,
%   \bcvector,
%   \Bcvector,
%   \vcvector,
%   \Vcvector,
%   \prvector,
%   \brvector,
%   \Brvector,
%   \vrvector,
%   \Vrvector,
% }
%   Now we define shorthands for all of the matrix types so that the user
%   does not have to specify |delimiter=|\meta{delim} every time.
%   We begin with the column vector.
%    \begin{macrocode}
% parenthesis
\@@_new_vector_shorth_doc_cmd:NNn \pcvector \moremath_column_vector:nn {p}
% brackets
\@@_new_vector_shorth_doc_cmd:NNn \bcvector \moremath_column_vector:nn {b}
% braces
\@@_new_vector_shorth_doc_cmd:NNn \Bcvector \moremath_column_vector:nn {B}
% single vert
\@@_new_vector_shorth_doc_cmd:NNn \vcvector \moremath_column_vector:nn {v}
% double vert
\@@_new_vector_shorth_doc_cmd:NNn \Vcvector \moremath_column_vector:nn {V}
%    \end{macrocode}
%   Now to the row vectors.
%    \begin{macrocode}
% parenthesis
\@@_new_vector_shorth_doc_cmd:NNn \prvector \moremath_row_vector:nn {p}
% brackets
\@@_new_vector_shorth_doc_cmd:NNn \brvector \moremath_row_vector:nn {b}
% braces
\@@_new_vector_shorth_doc_cmd:NNn \Brvector \moremath_row_vector:nn {B}
% single vert
\@@_new_vector_shorth_doc_cmd:NNn \vrvector \moremath_row_vector:nn {v}
% double vert
\@@_new_vector_shorth_doc_cmd:NNn \Vrvector \moremath_row_vector:nn {V}
%    \end{macrocode}
%   \changes{v0.1.0}{\GetVersionDate{v0.1.0}}{
%     Added commands for row and column vectors with predefined delimiters.
%   }
% \end{macro}
%
% \begin{macro}{
%   \psmallcvector,
%   \bsmallcvector,
%   \Bsmallcvector,
%   \vsmallcvector,
%   \Vsmallcvector,
%   \psmallrvector,
%   \bsmallrvector,
%   \Bsmallrvector,
%   \vsmallrvector,
%   \Vsmallrvector
% }
%   We also define shorthands for the \cs{shortcvector} and \cs{shortrvector}
%   versions.
%    \begin{macrocode}
% column vectors
% parenthesis
\@@_new_vector_shorth_doc_cmd:NNn \psmallcvector \moremath_column_smallvector:nn
  {p}
% brackets
\@@_new_vector_shorth_doc_cmd:NNn \bsmallcvector \moremath_column_smallvector:nn
  {b}
% braces
\@@_new_vector_shorth_doc_cmd:NNn \Bsmallcvector \moremath_column_smallvector:nn
  {B}
% single vert
\@@_new_vector_shorth_doc_cmd:NNn \vsmallcvector \moremath_column_smallvector:nn
  {v}
% double vert
\@@_new_vector_shorth_doc_cmd:NNn \Vsmallcvector \moremath_column_smallvector:nn
  {V}
%
% row vectors
% parenthesis
\@@_new_vector_shorth_doc_cmd:NNn \psmallrvector \moremath_row_smallvector:nn {p}
% brackets
\@@_new_vector_shorth_doc_cmd:NNn \bsmallrvector \moremath_row_smallvector:nn {b}
% braces
\@@_new_vector_shorth_doc_cmd:NNn \Bsmallrvector \moremath_row_smallvector:nn {B}
% single vert
\@@_new_vector_shorth_doc_cmd:NNn \vsmallrvector \moremath_row_smallvector:nn {v}
% double vert
\@@_new_vector_shorth_doc_cmd:NNn \Vsmallrvector \moremath_row_smallvector:nn {V}


}{ % \bool_if:nTF \l_@@_predef_crvector_bool FALSE PATH
  \msg_info:nnnn {moremath} {load / disabling} {no-crvector}
  {
    commands~producing~row~and~column~vectors
  }
} % \bool_if:nTF \l_@@_predef_crvector_bool
%    \end{macrocode}
%   \changes{v0.1.0}{\GetVersionDate{v0.1.0}}{
%   Added commands for inline math row and column vectors
%   with predefined delimiters.
%   }
% \end{macro}
%
%
% \subsubsection{(Anti-)diagonal Matrices}
% Now to the (anti-)diagonal matrix shorthands,
% these are also guarded by a key value option.
%    \begin{macrocode}
\bool_if:nTF \l_@@_predef_matrix_bool
{
%    \end{macrocode}
% \begin{macro}{
%   \diagmat,
%   \antidiagmat,
%   \smalldiagmat,
%   \smallantidiagmat,
% }
% \begin{NOTE}{MI}
%   Say something like we begin with some commands producing a matrix that
%   is not necessarily delimited.
% \end{NOTE}
%    \begin{macrocode}
\cs_if_free:NTF \diagmat
{
  \NewDocumentCommand \diagmat { o m }
  {
    \group_begin:
    \tl_if_novalue:nF {#1}
    {
      \keys_set:nn { moremath / matrix } {#1}
    }
    \moremath_diagonal_matrix:Vn \l_@@_matrix_delim_tl {#2}
    \group_end:
  }
}{
  \msg_warning:nnn {moremath} {matrix / already-defined-doc-cmd-skip}
  {
    \diagmat
  }
} % \cs_if_free:nTF \diagmat

\cs_if_free:NTF \antidiagmat
{
  \NewDocumentCommand \antidiagmat { o m }
  {
    \group_begin:
    \tl_if_novalue:nF {#1}
    {
      \keys_set:nn { moremath / matrix } {#1}
    }
    \moremath_antidiagonal_matrix:Vn \l_@@_matrix_delim_tl {#2}
    \group_end:
  }
}{
  \msg_warning:nnn { moremath } { matrix / already-defined-doc-cmd-skip }
  {
    \antidiagmat
  }
} % \cs_if_free:nTF \antidiagmat

\cs_if_free:NTF \smalldiagmat
{
  \NewDocumentCommand \smalldiagmat { o m }
  {
    \group_begin:
    \tl_if_novalue:nF {#1}
    {
      \keys_set:nn { moremath / matrix } {#1}
    }
    \moremath_diagonal_smallmatrix:Vn \l_@@_matrix_delim_tl {#2}
    \group_end:
  }
}{
  \msg_warning:nnn { moremath } { matrix / already-defined-doc-cmd-skip }
  {
    \smalldiagmat
  }
}

\cs_if_free:NTF \smallantidiagmat
{
  \NewDocumentCommand \smallantidiagmat { o m }
  {
    \group_begin:
    \tl_if_novalue:nF {#1}
    {
      \keys_set:nn { moremath / matrix } {#1}
    }
    \moremath_antidiagonal_smallmatrix:Vn \l_@@_matrix_delim_tl {#2}
    \group_end:
  }
}{
  \msg_warning:nnn { moremath } { matrix / already-defined-doc-cmd-skip }
  {
    \smallantidiagmat
  }
}
%    \end{macrocode}
%   \changes{v0.1.0}{\GetVersionDate{v0.1.0}}{%
%   Added commands producing (anti-)diagonal matrices, including inline math
%   versions.
%   }
% \end{macro}
%
% \paragraph{(Anti-)diagonal Matrices with Pre-defined Delimiters}
% As it is sort of cumbersome to always specify the delimiter key,
% we also provide commands with pre-defined delimiters.
% \begin{macro}{
%   \@@_new_matrix_shorth_doc_cmd:NNn
% }
%   To provide several shorthands for delimited matrices,
%   we use a helper function to avoid code duplication.
%   \cs{@@_new_matrix_shorth_doc_cmd:NNn} takes three arguments:
%   \begin{arguments}
%     \item The \meta{csname} to define
%     \item The \meta{csname} of the matrix function to use,
%           which should have the signature |Vn|.
%      \item The \enquote{predefined} delimiter of this version
%   \end{arguments}
%    \begin{macrocode}
\cs_new_protected:Nn \@@_new_matrix_shorth_doc_cmd:NNn
{
  \cs_if_free:NTF #1
  {
    \NewDocumentCommand #1 { o m }
    {
      \group_begin:
      \tl_if_empty:nF {#3}
      {
        \keys_set:nn { moremath / matrix }
        {
          delimiter = #3
        }
      } % \tl_if_empty:nF {#3}
      \tl_if_novalue:nF {##1}
      {
        \keys_set:nn { moremath / matrix } {##1}
      }
      #2 \l_@@_matrix_delim_tl {##2}
      \group_end:
    }
  }{
    \msg_warning:nnn { moremath } { matrix / already-defined-doc-cmd-skip }
    {
      #1
    }
  }
}
%    \end{macrocode}
% \end{macro}
%
% We now define the shorthand commands with predefined delimiters.
% \begin{macro}{
%   \pdiagmat,
%   \bdiagmat,
%   \Bdiagmat,
%   \vdiagmat,
%   \Vdiagmat,
% }
%   We begin with the regular diagonal matrix
%    \begin{macrocode}
\@@_new_matrix_shorth_doc_cmd:NNn \pdiagmat \moremath_diagonal_matrix:Vn {p}
\@@_new_matrix_shorth_doc_cmd:NNn \bdiagmat \moremath_diagonal_matrix:Vn {b}
\@@_new_matrix_shorth_doc_cmd:NNn \Bdiagmat \moremath_diagonal_matrix:Vn {B}
\@@_new_matrix_shorth_doc_cmd:NNn \vdiagmat \moremath_diagonal_matrix:Vn {v}
\@@_new_matrix_shorth_doc_cmd:NNn \Vdiagmat \moremath_diagonal_matrix:Vn {V}
%    \end{macrocode}
%   \changes{v0.1.0}{\GetVersionDate{v0.1.0}}{
%     Added commands for delimited diagonal matrices.
%   }
% \end{macro}
% \begin{macro}{
%   \pantidiagmat,
%   \bantidiagmat,
%   \Bantidiagmat,
%   \vantidiagmat,
%   \Vantidiagmat,
%  }
%   Now for the anti-diagonal matrix commands.
%    \begin{macrocode}
\@@_new_matrix_shorth_doc_cmd:NNn \pantidiagmat
  \moremath_antidiagonal_matrix:Vn {p}
\@@_new_matrix_shorth_doc_cmd:NNn \bantidiagmat
  \moremath_antidiagonal_matrix:Vn {b}
\@@_new_matrix_shorth_doc_cmd:NNn \Bantidiagmat
  \moremath_antidiagonal_matrix:Vn {B}
\@@_new_matrix_shorth_doc_cmd:NNn \vantidiagmat
  \moremath_antidiagonal_matrix:Vn {v}
\@@_new_matrix_shorth_doc_cmd:NNn \Vantidiagmat
  \moremath_antidiagonal_matrix:Vn {V}
%    \end{macrocode}
%   \changes{v0.1.0}{\GetVersionDate{v0.1.0}}{
%     Added commands for delimited antidiagonal matrices.
%   }
% \end{macro}
%
% \begin{macro}{
%   \psmalldiagmat,
%   \bsmalldiagmat,
%   \Bsmalldiagmat,
%   \vsmalldiagmat,
%   \Vsmalldiagmat,
% }
%   We continue with the inline math versions based on the \env{smallmatrix*}
%   environment.
%    \begin{macrocode}
\@@_new_matrix_shorth_doc_cmd:NNn \psmalldiagmat
  \moremath_diagonal_smallmatrix:Vn {p}
\@@_new_matrix_shorth_doc_cmd:NNn \bsmalldiagmat
  \moremath_diagonal_smallmatrix:Vn {b}
\@@_new_matrix_shorth_doc_cmd:NNn \Bsmalldiagmat
  \moremath_diagonal_smallmatrix:Vn {B}
\@@_new_matrix_shorth_doc_cmd:NNn \vsmalldiagmat
  \moremath_diagonal_smallmatrix:Vn {v}
\@@_new_matrix_shorth_doc_cmd:NNn \Vsmalldiagmat
  \moremath_diagonal_smallmatrix:Vn {V}
%    \end{macrocode}
%   \changes{v0.1.0}{\GetVersionDate{v0.1.0}}{
%     Added commands for delimited diagonal matrices suitable for inline math.
%   }
% \end{macro}
% \begin{macro}{
%   \psmallantidiagmat,
%   \bsmallantidiagmat,
%   \Bsmallantidiagmat,
%   \vsmallantidiagmat,
%   \Vsmallantidiagmat,
% }
% We provide also anti-diagonal versions of the small matrices.
%    \begin{macrocode}
\@@_new_matrix_shorth_doc_cmd:NNn \psmallantidiagmat
  \moremath_antidiagonal_smallmatrix:Vn {p}
\@@_new_matrix_shorth_doc_cmd:NNn \bsmallantidiagmat
  \moremath_antidiagonal_smallmatrix:Vn {b}
\@@_new_matrix_shorth_doc_cmd:NNn \Bsmallantidiagmat
  \moremath_antidiagonal_smallmatrix:Vn {B}
\@@_new_matrix_shorth_doc_cmd:NNn \vsmallantidiagmat
  \moremath_antidiagonal_smallmatrix:Vn {v}
\@@_new_matrix_shorth_doc_cmd:NNn \Vsmallantidiagmat
  \moremath_antidiagonal_smallmatrix:Vn {V}
%    \end{macrocode}
%   \changes{v0.1.0}{\GetVersionDate{v0.1.0}}{
%     Added commands for delimited anti-diagonal matrices suitable for inline math.
%   }
% \end{macro}
%
% \subsubsection{Identity Matrices}
% We also provide document level commands for producing an identity matrix.
% These commands are also guarded by the same variable as the other matrix commands
% (\cs{l_@@_predef_matrix_bool}).
%
% \begin{macro}{
%   \idmat,
%   \smallidmat,
% }
%   We provide two document level commands for producing the identiy matrix,
%   one for inline math mode and one for display math mode.
%
%   We start with the display math mode version.
%    \begin{macrocode}
\cs_if_free:NTF \idmat
{
  \NewDocumentCommand \idmat { o m }
  {
    \group_begin:
    \tl_if_novalue:nF {#1}
    {
      \keys_set:nn { moremath / matrix } {#1}
    }
    \moremath_id_matrix:n {#2}
    \group_end:
  }
}{ % \cs_if_free:NTF \idmat FALSE BRANCH
  \msg_warning:nnn { moremath } { matrix / already-defined-doc-cmd-skip }
    {\idmat}
}
%    \end{macrocode}
%   Afterwards we continue with the inline math mode version.
%    \begin{macrocode}
\cs_if_free:NTF \smallidmat
{
  \NewDocumentCommand \smallidmat { o m }
  {
    \group_begin:
    \tl_if_novalue:nF {#1}
    {
      \keys_set:nn { moremath / matrix } {#1}
    }
    \moremath_id_smallmatrix:n {#2}
    \group_end:
  }
}{ % \cs_if_free:NTF \smallidmat FALSE BRANCH
  \msg_warning:nnn { moremath } { matrix / already-defined-doc-cmd-skip }
    {\smallidmat}
}
%    \end{macrocode}
%   \changes{v0.2.0}{\GetVersionDate{v0.2.0}}{%
%     Added \cs{idmat} and \cs{smallidmat} document level commands.
%   }
% \end{macro}
%
% \paragraph{Identity Matrices with Pre-defined Delimiters}
% We also define shorthands for the commonly used delimiters around matrices,
% to avoid code duplication, we first declare a helper function for this.
% \begin{macro}{
%   \@@_new_id_matrix_doc_cmd:NNn
% }
%   This function creates a new document level command for an identity matrix like
%   command.
%   It allows pre-setting \meta{kv~opts}.
%   The function takes three arguments.
% \begin{arguments}
%   \item The \meta{csname} of the document level command to define
%
%   \item The \meta{csname} of the function to use
%
%   This is indented to be one of \cs{moremath_id_matrix:n} or
%   \cs{moremath_id_smallmatrix:n}
%
%   \item \meta{kv~opts} to preset in the "moremath / matrix" namespace
%   for this command
%
%     This is meant to be used for pre-setting the key "delimiter".
% \end{arguments}
%    \begin{macrocode}
\cs_new_protected:Nn \@@_new_id_matrix_doc_cmd:NNn
{
  \cs_if_free:NTF #1
  {
    \NewDocumentCommand #1 { o m }
    {
      \group_begin:
      \tl_if_empty:nF {#3}
      {
        \keys_set:nn { moremath / matrix } {#3}
      }
      \tl_if_novalue:nF {##1}
      {
        \keys_set:nn { moremath / matrix } {##1}
      }
      #2 {##2}
      \group_end:
    }
  }{ % \cs_if_free:NTF #1 FALSE BRANCH
    \msg_warning:nnn { moremath } { matrix / already-defined-doc-cmd-skip }
      {#1}
  }
}
%    \end{macrocode}
% \end{macro}
%
% \begin{macro}{
%   \pidmat,
%   \bidmat,
%   \Bidmat,
%   \vidmat,
%   \Vidmat
% }
%   We begin with the display math versions,
%   starting with the version delimited by parenthesis,
%    \begin{macrocode}
\@@_new_id_matrix_doc_cmd:NNn \pidmat \moremath_id_matrix:n { delimiter = p }
%    \end{macrocode}
%   continue with the bracketed version,
%    \begin{macrocode}
\@@_new_id_matrix_doc_cmd:NNn \bidmat \moremath_id_matrix:n { delimiter = b }
%    \end{macrocode}
%   the version using braces,
%    \begin{macrocode}
\@@_new_id_matrix_doc_cmd:NNn \Bidmat \moremath_id_matrix:n { delimiter = B }
%    \end{macrocode}
%   single vertical lines,
%    \begin{macrocode}
\@@_new_id_matrix_doc_cmd:NNn \vidmat \moremath_id_matrix:n { delimiter = v }
%    \end{macrocode}
%    and finally double vertical lines.
%    \begin{macrocode}
\@@_new_id_matrix_doc_cmd:NNn \Vidmat \moremath_id_matrix:n { delimiter = V }
%    \end{macrocode}
%
%   \changes{v0.2.0}{\GetVersionDate{v0.2.0}}{
%     Added document level commands \cs{pidmat}, \cs{bidmat}, \cs{Bidmat},
%     \cs{vidmat}, and \cs{Vidmat}.
%   }
% \end{macro}
%
% \begin{macro}{
%   \psmallidmat,
%   \bsmallidmat,
%   \Bsmallidmat,
%   \vsmallidmat,
%   \Vsmallidmat,
% }
% Now we also define shorthands for inline math mode.
% We start again defining the version using parenthesis,
%    \begin{macrocode}
\@@_new_id_matrix_doc_cmd:NNn \psmallidmat \moremath_id_smallmatrix:n
  { delimiter = p }
%    \end{macrocode}
%   then brackets,
%    \begin{macrocode}
\@@_new_id_matrix_doc_cmd:NNn \bsmallidmat \moremath_id_smallmatrix:n
  { delimiter = b }
%    \end{macrocode}
%   then braces,
%    \begin{macrocode}
\@@_new_id_matrix_doc_cmd:NNn \Bsmallidmat \moremath_id_smallmatrix:n
  { delimiter = B }
%    \end{macrocode}
%   followed by single vertical lines,
%    \begin{macrocode}
\@@_new_id_matrix_doc_cmd:NNn \vsmallidmat \moremath_id_smallmatrix:n
  { delimiter = v }
%    \end{macrocode}
%   and finally double vertical lines.
%    \begin{macrocode}
\@@_new_id_matrix_doc_cmd:NNn \Vsmallidmat \moremath_id_smallmatrix:n
  { delimiter = V }
%    \end{macrocode}
%   \changes{v0.2.0}{\GetVersionDate{v0.2.0}}{%
%   Added document level commands \cs{psmallidmat}, \cs{bsmallidmat},
%   \cs{Bsmallidmat}, \cs{vsmallidmat} and \cs{Vsmallidmat}.
%   }
% \end{macro}
%
%    \begin{macrocode}
}{ % \bool_if:nTF \l_@@_predef_matrix_bool FALSE BRANCH
  \msg_info:nnnn {moremath} { load / disabling } { no-matrix }
  {
    (anti-)diagonal~matrix~commands
  }
} % \bool_if:nTF \l_@@_predef_matrix_bool
%    \end{macrocode}
%
%
% \section{Shorthand Macros for Absolute Value and Norm}
% \label{sec:impl-abs-shorthands}
%
% We first declare another warning message to inform the user of the case that,
% the \meta{csnames} are already taken.
%    \begin{macrocode}
\msg_new:nnnn { moremath } { abs-shorth / csname-already-defined-skip }
{
  Control~sequence~'#1'~is~already~defined.\\
  Skipping~declaration~of~paired~delimiter~\msg_line_context:.\\
  Use~package~option~'no-abs-shorthands'~to~disable~the~paired\\
  delimiter~shorthands.
}{
  The~control~sequence~'#1'~has~already~been\\
  defined~by~something~else.\\
  I~am~refusing~to~overwrite~its~existing~definition~and~instead~avoid\\
  declaring~a~paired~delimiter.\\
}
%    \end{macrocode}
%
%
% As with the other parts these macros may be conditionally disabled.
%    \begin{macrocode}
\bool_if:NTF \l_@@_predef_abs_bool
{
%    \end{macrocode}
% \begin{macro}{\abs,\norm}
% These macros provide shorthands for \(\abs{\meta{content}}\)
% and \(\norm{\meta{content}}\).
%    \begin{macrocode}
\cs_if_free:NTF \abs
{
  \DeclarePairedDelimiter \abs {\lvert} {\rvert}
}{
  % warn if the csname is taken
  \msg_warning:nnn { moremath } { abs-shorth / csname-already-defined-skip }
    {\abs}
} % \cs_if_free:NTF \abs

\cs_if_free:NTF \norm
{
  \DeclarePairedDelimiter \norm {\lVert} {\rVert}
}{
  % warn if csname is already taken
  \msg_warning:nnn { moremath } { abs-shorth / csname-already-defined-skip }
    {\norm}
} % \cs_if_free:NTF
%    \end{macrocode}
%   \changes{v0.1.0}{\GetVersionDate{v0.1.0}}{%
%     Added shorthands for absolute value and norm.
%   }
% \end{macro}
%
%    \begin{macrocode}
}{
  \msg_info:nnnn {moremath} {load / disabling} {no-abs-shorthands}
  {
    '\abs'~and~'\norm'~macros
  }
} % End of the conditional
%    \end{macrocode}
%
% \iffalse debug-switch
%\debug_off:n {all}
% \fi
%    \begin{macrocode}
%</package>
%    \end{macrocode}
% \end{implementation}
%
% \section*{Copyright and License}
% \addcontentsline{toc}{section}{Copyright and License}
% \label{sec:license}
%
% The following copyright notice applies to the \pkg{moremath} package:
%
% \begin{quote}
%   Copyright \textcopyright{} 2024 Marcel Ilg
%
% This work may be distributed and/or modified under the
% conditions of the LaTeX Project Public License, either version 1.3
% of this license or (at your option) any later version.
% The latest version of this license is in
%      \url{https://www.latex-project.org/lppl.txt}
% and version 1.3 or later is part of all distributions of LaTeX
% version 2005/12/01 or later.
%
% This work has the LPPL maintenance status \enquote{maintained}.
%
% The Current Maintainer of this work is Marcel Ilg.
%
% This work consists of the files listed in \file{MANIFEST.md}.
% \end{quote}
%
% \noindent The file \file{MANIFEST.md} has to be distributed together with the
% package.
%
%
% \printbibliography[heading=bibintoc]
% \PrintChanges
% \PrintIndex
